\subsection{Tobacco, what's that smell?}\label{disc:tobacco}

A concern for Western tourists was the smell of tobacco in their room. Upon manual inspection of a sample of reviews with this keyword, we found that the room was often advertised as non-smoking, yet, the smell permeated the room and curtains. Another common complaint was that there were no non-smoking facilities available at all. The smell of smoke can completely ruin some customers' stay and give a bad impression to review writers, which can lower the number of future customers. 

However, in comparison, Chinese customers seem not to be bothered by this at all. We consulted studies involving the use of tobacco in different countries. Previous research states that 49 - 60\% of Chinese men (and 2.0 - 2.8\% of women) currently smoke or have smoked before, taken from a sample of \num[group-separator={,}]{170000} Chinese adults in 2013-2014, which is high compared to many English-speaking countries \cite[][]{zhang2019tobacco, who2015tobacco}.

Japan itself has a polarized view on smoking, and despite being one of the world's largest tobacco markets, its use has been decreasing in recent years. Smoking in public spaces is prohibited in some wards of Tokyo (namely Chiyoda, Shinjuku, and Shibuya). However, it is generally only urged and not mandatory to have smoking restrictions in restaurants, bars, hotels, and public areas. However, there are many places where ``smoking rooms" are available to keep the smoke to an enclosed area and avoid bothering others. Despite this, businesses, especially those who cater to certain kinds of customers, will generally be discouraged from having smoking restrictions if they want to keep their clientele. If Japanese hotels want to cater to all kinds of customers, Western and Asian alike, they must provide spaces without tobacco smell. After all, even if it does not bother a few customers, the lack of smell would make it an appropriate space for all kinds of customers. 
