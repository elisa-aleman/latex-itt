\begin{abstract}

  The Japanese spirit of hospitality and service \textit{Omotenashi} is known worldwide for its excellence. Recent years show a steady increase in international tourists coming to Japan. Chinese tourists, especially, have been steadily increasing. However, before the shift that has brought a global perspective in recent years, most tourist behavior studies were biased for the Western world. Previous research shows that different cultural backgrounds result in different expectations and, arguably, different satisfaction factors. Knowing this, a cross-cultural study of differences between Chinese and Western cultures after the current boom in the Chinese economy in the high standard Japanese hospitality environment is fascinating. Will the top-grade hospitality of Japan influence both populations equally, or will their cultural differences set them apart? Will they be satisfied with the soft attributes like service or be more concerned with hard attributes like location and facilities? We bring light to these questions and the differences in each population's satisfaction and dissatisfaction factors in different price ranges. Taking advantage of Web 2.0, we applied Shannon's entropy to extract these factors automatically and then use them in an SVM to classify a more extensive data set. We then used dependency parsing and part of speech tagging to extract which nouns were tied to praising adjectives. We found that Chinese tourists are less concerned with hospitality and more with room quality than Western tourists. The latter were delighted by the staff behavior. We also found that Chinese tourists are concerned with the lack of a Chinese friendly environment, and Western customers are unsatisfied with dirty rooms or the smell of cigarettes.

  Sentiment Analysis, Hotels and Lodging, Text Mining, Chinese, English, Satisfaction and Dissatisfaction Factors

\section{Introduction}\label{intro}

  Japan has been known historically for its hospitality being the highest grade. The spirit of Japanese hospitality is celebrated worldwide in a single Japanese word: \textit{Omotenashi}. With roots in Japanese history and tea ceremony, their hospitality is famous around the world \cite[][]{al2015characteristics}. Therefore it would stand to reason that tourists visiting Japan would have this hospitality as their first and foremost satisfaction factor. However, it is known that customers from different countries and cultures hold different expectations \cite[][]{engel1990}. Thus, it could be theorized that their satisfaction factors should be different. How will different cultures react and perceive hotels and their hospitality in this context? Our study attempts to bring light to this with two essential tourist populations that differ in culture to Japan: Chinese and Western tourists. 

  In the last couple of decades, the Japanese economy has been more and more affected by an increase in inbound international tourism \cite[][]{jones2009}. There was a Year-on-Year Growth Rate of 19.3\% in 2017, with a total of \num[group-separator={,}]{28691073} inbound tourists that year \cite[][]{jnto2003-2019}. From this total, the tourist population was mostly Asian (86.14\%), and approximately a fourth of the total (25.63\%) came from China. Western countries, counting English-speaking countries and Europe, make for 11.4\% of the total, with a 7.23\% of the total being countries where English is the official or the de facto national language. The effect of Chinese tourists on international economies is increasing. From that, the number of researchers interested in this phenomenon has been increasing as well. \cite[][]{sun2017}. With these and other multicultural tourist populations, the tourist market is more and more diverse. Diversity in customers' cultural backgrounds means that their expectations when staying at a hotel will also be varied. Therefore, Hotel management needs to cater to these needs and expectations to increase customer satisfaction, maintain a good reputation, and generate positive word-of-mouth.

  However, many tourist behavior analyses have been performed with Western subjects. As such, a gap in knowledge and discussion existed until recent decades. Those studies that do include Asian populations in their analysis most commonly study Chinese tourist behavior \cite[e.g.][]{liu2019, chang2010, dongyang2015}. The few that compare Asian to Western tourist behavior \cite[e.g.][]{choi2000} are commonly survey or interview-based studies with small samples. These, while valid, can have their limitations. This gap in research creates a need for cross-cultural studies for the increasing Asian and Western tourist populations. It could be said that Westerners make for a smaller portion of the tourist population compared to Asians. However, according to \cite{choi2000}, Westerners are known as ``long-haul" customers, spending more than 45\% of their budget on hotel lodging. In comparison, their Asian counterparts only spend 25\% of their budget on hotels. Therefore, it is essential to study Asian and Western tourist populations, their differences, and contrast with the existing literature results.

  With the advent of Web 2.0 and customer review websites, researchers realized the benefits of online reviews for research, and their importance for sales  \cite[][]{ye2009, basuroy2003}, customer consideration \cite[][]{vermeulen2009} and perception of services and products \cite[][]{browning2013}, among other effects of online interactions between customers \cite[e.g.][]{xiang2010, ren2019}. Consequentially, tourism research also began to use information collected online for data mining analysis, such as opinion mining \cite[e.g.][]{hu2017436}, predicting hotel demand from online traffic \cite[][]{yang2014}, recommender systems \cite[e.g.][]{loh2003}, and more. Data mining, machine learning, and big data methodologies can increase the number of manageable samples per study. The increase can be from the hundred samples manually analyzed by researchers to the hundreds of thousands automatically analyzed by machines. This technology can not only help confirm existing theories but also lead to finding new patterns and to knowledge discovery \cite[][]{fayyad1996data}. 

  In this study, we take advantage of the availability of enormous amounts of online reviews of Japanese hotels by both Mainland Chinese tourists posting in \textit{Ctrip} and Western English-speaking tourists populations posting in \textit{TripAdvisor}. With this data, we can confirm existing theories about their differences in behavior and explore the data to discover factors that could have been overlooked in the past. To do this, we use machine learning to automatically classify review sentences as positive or negative opinions of the hotel. We then perform a statistical extraction of the topics that concerns the customers of each population the most.

\section{Research objective}\label{research_objective}

  This study's objective is to determine the difference in factors driving satisfaction and dissatisfaction between Chinese and English-speaking tourists in the context of high-grade hospitality of Japanese hotels using text-mining techniques. We aim to contrast customer groups' satisfaction and dissatisfaction factors across several price ranges. We use machine learning to classify texts' sentiment and natural language processing to study commonly used word pairings. More importantly, we also intend to measure how hard and soft attributes influence customer groups' satisfaction and dissatisfaction. We define hard attributes as relating to physical aspects and environmental aspects, such as the hotel's facilities, the hotel's location, infrastructure, and real estate nearby. In contrast, soft attributes are the hotel's non-physical attributes related to services, staff, or management.

  Our proposal includes using large scale data from online hotel reviews in Chinese and English to study their differences in a statistical manner. In the past, survey-based studies have provided a theoretical background for a few specific tourist populations of a single culture or that travel with a single purpose. These studies' short scope often leads to difficulties in observing cultural and language differences in a single study.

  Our study attempts to uncover the difference in satisfaction and dissatisfaction factors between different cultures. These factors can become the focal point for improving the tourism and service industries and increasing customer satisfaction. Satisfied customers will then write more positive online reviews that will, in turn, increase sales and attract new customers.

\section{Theoretical background and hypothesis development}\label{theory_hypothesis}

  \subsection{Japanese hospitality and service: \textit{Omotenashi}}\label{theory_omotenashi}

    The spirit of Japanese hospitality, or \textit{Omotenashi}, has roots in the countries history. However, to this day, it is regarded as the highest standard \cite[][]{ikeda2013omotenashi, al2015characteristics}. There is even a famous phrase in customer service in Japan: \textit{okyaku-sama wa kami-sama desu}, or translated ``The customer is god''. Some say that \textit{omotenashi} originated from the old Japanese art of the tea ceremony in the 16th century. However, other scholars found that its roots come from even earlier, in the form of formal banquets in the 7th-century \cite[][]{aishima2015origin}. The practice of high standards in hospitality has survived throughout the years. Today, it permeates all business practices in Japan, from the cheapest convenience stores to the most expensive ones. Manners, service, and respect towards the customer are taught to workers in their training. High standards are always followed as to not fall behind in the competition. In Japanese businesses, hotels included, staff members are trained to speak in \textit{sonkeigo}, or ``respectful language", one of the most formal of the Japanese formality syntaxes. They are also trained to bow with different depths depending on the situation, where a light bow could be used to say: ``Please, allow me to guide you". Deep bows are also used to apologize for any inconvenience the customer could have, followed by a very respectful apology. In fact, despite the word \textit{omotenashi} being translated directly as ``hospitality'', it includes both the concepts of hospitality and service \cite[][]{Kuboyama2020}. This hospitality culture permeates every kind of business with customer interaction in Japan. A simple convenience shop could express all of these hospitality and service standards, which are not exclusive to hotels.  

    It stands to reason that this cultural aspect of hospitality would be a positive aspect that would be at the top of satisfaction for any customer. However, in many cases, other factors such as proximity to a convenience store, transport availability, or room quality might be more critical to a customer.  In this study, we cannot determine whether or not a hotel is practicing with the cultural standards of \textit{omotenashi} directly. Instead, we consider it as a cultural factor that influences all businesses in Japan. We then observe the customers' evaluations regarding service and hospitality factors in general, which can be compared to other places and business practices in the world. In summary, our study considers the influence of the cultural aspect of \textit{omotenashi} while analyzing the evaluations of service and hospitality factors that are universal to all hotels in any country.

    Therefore we pose a research question for our study:

    To what degree are Chinese and Western tourists satisfied with Japanese hospitality factors such as staff behavior or service?

    However, Japanese hospitality comes from Japanese culture. Different cultures interacting with it could have a different evaluation of it. While some might be impressed by it, some might consider other factors more important to their stay in a hotel. This point leads us to a derivative of the above research question:

    Do Western and Chinese tourists have a different evaluation of Japanese hospitality factors such as staff behavior or service?

  \subsection{Customer satisfaction and dissatisfaction towards individual factors during hotel stay}\label{theory_satisfaction}

    Customer satisfaction in tourism has been analyzed since decades past, \cite{hunt1975} having defined customer satisfaction as the realization or overcoming of expectations towards the service. \cite{oliver1981} defined it as an emotional response to the provided services in retail and other contexts, and \cite{oh1996} reviewed the psychological processes of customer satisfaction for the hospitality industry. It is generally agreed upon that satisfaction and dissatisfaction stem from the individual expectations of the customer. As such, \cite{engel1990} states that each customer's background, therefore, influences satisfaction and dissatisfaction. Previous studies on the dimensions of culture that influence differences in expectations have been performed in the past \cite{donthu1998cultural}. Western and Chinese customers can then have very different satisfaction and dissatisfaction factors since they have different backgrounds and cultures. These varying backgrounds will lead to varying expectations of the hotel services, the experiences they want to have while staying at a hotel, and the level of comfort that they will have. These expectations will be there from the moment that they choose the hotel throughout their stay. In turn, these different expectations will determine the distinct factors of satisfaction and dissatisfaction for each kind of customer and the order in which they prioritize them. 

    Because of their different origins, expectations, and cultures, it stands to reason Chinese and Western tourists could have completely different factors to one another. Therefore, it could be that some factors do not appear in the other reviews at all. For example, between different cultures, it can be that a single word can express some concept that would take more words in the other language. So we must measure their differences or similarities at their common ground as well.

    However, in this study, we study not overall customer satisfaction but the satisfaction and dissatisfaction that stem from individual-specific expectations, be they conscious or unconscious. For example, suppose a customer has a conscious expectation of a comfortable bed and a wide shower, and it is realized during their visit. In that case, they will be satisfied with this matter. However, suppose that same customer with a conscious expectation of a comfortable bed experienced loud noises at night. In that case, they can be dissatisfied with a different aspect, regardless of the satisfaction towards the bed. Then, the same customer might have packed their toiletries, thinking that the amenities might not include those. They can then be pleasantly surprised with good quality amenities and toiletries, satisfying an unconscious expectation. This definition of satisfaction does not allow us to examine overall customer satisfaction. However, it will allow us to examine the factors that a hotel can revise individually and how a population perceives them as a whole. In our study, we consider the definitions by \cite{hunt1975} that satisfaction is a realization of an expectation, and posit that customers can have different expectations towards different service aspects. Therefore, in our study, we define satisfaction as the emotional response to the realization or overcoming of conscious or unconscious expectations towards an individual aspect or factor of a service. On the other hand, dissatisfaction is the emotional response to the lack of a realization or under-performance of these conscious or unconscious expectations towards specific service aspects.

    Studies on customer satisfaction \cite[e.g.][]{truong2009, romao2014, wu2009} commonly use the Likert scale \cite[][]{likert1932technique} (e.g. 1 to 5 scale from strongly dissatisfied to strongly satisfied) to perform statistical analysis of which factors relate most to satisfaction on the same dimension as dissatisfaction \cite[e.g.][]{chan201518, choi2000}. The Likert scale's use leads to correlation analyses where one factor can lead to satisfaction, implying that the lack of it can lead to dissatisfaction. However, a binary distinction (satisfied or dissatisfied) could allow us to analyze the factors that correlate to satisfaction and explore factors that are solely linked to dissatisfaction. There are fewer examples of this approach, but studies have done this in the past \cite[e.g.][]{zhou2014}. This method can indeed decrease the extent to which we can analyze degrees of satisfaction or dissatisfaction. However, it has the benefit that it can be applied to a large sample of text data via automatic sentiment detection techniques using artificial intelligence. 

    Previous research has also focused more on soft attributes, with little focus on hard attributes, like location or infrastructure, mostly focusing only on facilities \cite[e.g.][]{shanka2004, choi2001}. However, hard factors, which are uncontrollable by the hotel staff, can play a part in the customers' choice behavior and satisfaction. Examples of these factors include the hotel's surroundings, location, language immersion of the country as a whole, or touristic destinations, and the hotel's integration with tours available nearby, among other factors. 

    This leads to another couple of research questions:

    To what degree does satisfaction and dissatisfaction stem from hard and soft attributes of the hotel?

    How differently do Chinese and Western customers perceive hard and soft attributes of the hotel?

    The resulting proportions of hard attributes to soft attributes for each population could measure how much the improvement of management in the hotel can increase future satisfaction in customers. 

  \subsection{Chinese and Western tourist behavior}\label{theory_zh_en}

    
    In the past, social science and tourism studies focused extensively on Western tourist behavior in other countries. Recently, however, with the rise of Chinese outbound tourism, both academic researchers and businesses have decided to study Chinese tourist behavior. Explaining this increase, \cite{sun2017} analyzed the number of studies related to Chinese tourists from 2001 to 2012 and found a steady increase until 2007, followed by the rapid growth of Chinese tourism studies. This increase in Chinese tourist behavior research resulted in several studies focusing on only the behavior of this subset of tourists. To this day, studies and analyses specifically comparing Asian and Western tourists are scarce, and even less the number of studies explicitly comparing Chinese and Western tourists. One example is a study by \cite{choi2000}, who found that Western tourists visiting Hong Kong are satisfied more with room quality, while Asians are satisfied with the value for money. Another study by \cite{bauer1993changing} found that Westerners prefer hotel health facilities. At the same time, Asian tourists were more inclined to enjoy the Karaoke facilities of hotels. Both groups tend to have high expectations for the overall facilities. Another study done by \cite{kim2000} found that American tourists were found to be individualistic and motivated by novelty. In contrast, Japanese tourists were collectivist and motivated by prestige and family, with an escape from routine and increased knowledge as a common motivator. 

    One thing to note with the above Asian vs. Western analyses is that they were performed before 2000 and that they are not Chinese specific but study Asian people in general. Meanwhile, the current Chinese economy boom is increasing the influx of tourists of this nation. The resulting increase in marketing and the creation of guided tours for Chinese tourists could have created a difference in tourists' perceptions and expectations. In turn, if we follow the definition of satisfaction by \cite{hunt1975}, that change in expectations could have influenced their satisfaction factors when traveling. Another note is that these studies were performed with questionnaires in places where it would be easy to locate tourists, i.e., airports. However, our study of online reviews takes the data that the hotel customers uploaded themselves. This data makes the analysis unique in exploring their behavior compared with Western tourists via factors that are not considered in most other studies. Furthermore, our study is unique in observing the customers in the specific environment of high-level hospitality in Japan.

    More recent studies have surfaced as well. A cross-country study \cite[][]{FRANCESCO201924} using posts from U.S.A. citizens, Italians, and Chinese tourists, determined using a text link analysis that customers from different countries indeed have a different perception and emphasis of a few predefined hotel attributes. According to their results, U.S.A. customers perceive cleanliness and quietness most positively. In contrast, Chinese customers perceive budget and restaurant above other attributes. Another couple of studies \cite[][]{JIA2020104071, HUANG2017117} analyze differences between Chinese and U.S. tourists using text mining techniques and more massive datasets, although in a restaurant context. 

    These last three articles focus on U.S.A. culture, while our study focuses on Western culture. Another difference with our study is that of the context of the study. The first study \cite[][]{FRANCESCO201924} was done with the context of tourists from three countries staying in hotels across the world. The second one chose restaurant reviews from the U.S.A. and Chinese tourists eating in three countries in Europe. The third is analyzing restaurants in Bejing.

    On the other hand, our study focuses on Western culture, instead of a single Western country, and Chinese culture clashing with the hospitality environment in Japan, specifically. Japan's importance in this analysis comes from the unique environment of high-grade hospitality that the country presents. In this environment, do customers hold their satisfaction to this hospitality regardless of their culture, or are other factors more relevant to the customers? Our study measures this at a large scale across different hotels in Japan. 

    
    Other studies have gone further and studied people from many countries in their samples and performed a more universal and holistic (not cross-culture) analysis. \cite{choi2001} analyzed hotel guest satisfaction determinants in Hong Kong with surveys in English, Chinese and Japanese translations, with people from many countries in their sample. \cite{choi2001} found that staff service quality, room quality, and value for money were the top satisfaction determinants. As another example, \cite{Uzama2012} produced a typology for foreigners coming to Japan for tourism, without making distinctions for their culture, but their motivation in traveling in Japan. In another study, \cite{zhou2014} analyzed hotel satisfaction using English and Mandarin online reviews from guests staying in Hangzhou, China coming from many different countries. The general satisfaction score was noticed to be different in those countries. However, a more in-depth cross-cultural analysis of the satisfaction factors was not performed. As a result of their research, \cite{zhou2014} thus found that customers are universally satisfied by welcome extras, dining environments, and special food services. 

    
    Regarding Western tourist behavior, a few examples can tell us what to expect when analyzing our data. \cite{kozak2002} found that British and German tourists' satisfaction determinants while visiting Spain and Turkey were hygiene and cleanliness, hospitality, the availability of facilities and activities, and accommodation services. \cite{shanka2004} found that English-speaking tourists in Perth, Australia were most satisfied with staff friendliness, the efficiency of check-in and check-out, restaurant and bar facilities, and lobby ambiance. 

    
    Regarding outbound Chinese tourists, academic studies about Chinese tourists have increased \cite[][]{sun2017}. Different researchers have found that Chinese tourist populations have several specific attributes. According to \cite{ryan2001} and their study of Chinese tourists in New Zealand, Chinese tourists prefer nature, cleanliness, and scenery in contrast to experiences and activities. \cite{dongyang2015} studied Chinese tourists in the Kansai region of Japan and found that Chinese tourists are satisfied mostly with exploring the food culture of their destination, cleanliness, and staff. Studying Chinese tourists in Vietnam, \cite{truong2009} found that Chinese tourists are highly concerned with value for money. According to \cite{liu2019}, Chinese tourists tend to have harsher criticism compared with other international tourists. Moreover, as stated by \cite{gao2017chinese}, who analyzed different generations of Chinese tourists and their connection to nature while traveling, Chinese tourists prefer nature overall. However, the younger generations seem to do so less than their older counterparts. 

    
    Although the studies focusing only on Chinese tourists or Western tourists have a narrow view, their theoretical contributions are valuable. We can see that depending on the study and the design of questionnaires and the destinations, the results can vary greatly. Not only that, but while there seems to be some overlap in most studies, some factors are completely ignored in one study but not in the other. Since our study uses data mining, each factor's definition is left for hotel customers to decide en masse via their reviews. This means that the factors will be selected through statistical methods alone, instead of being defined by the questionnaire. Our method allows us to find factors that we would not have contemplated. It also avoids enforcing a factor on the mind of study subjects by presenting them with a question that they did not think of by themselves. This large variety of opinions in a well-sized sample, added to the automatic findings of statistical text analysis methods, gives our study an advantage compared to others with smaller samples. This study could also help us analyze the satisfaction and dissatisfaction factors cross-culturally and compare them with the existing literature.

    
    Undoubtedly previous literature has examples of other cross-culture studies of tourist behavior and to further highlight our study and its merits. A contrast is shown in Table \ref{tab:lit-rev}. This table shows that older studies were conducted with surveys and had a different study topic. These are changes in demand \cite[][]{bauer1993changing}, tourist motivation \cite[][]{kim2000}, and closer to our study, satisfaction levels \cite[][]{choi2000}. However, our study topic is not the levels of satisfaction but the factors that drive it and dissatisfaction, which is overlooked in most studies. Newer studies with larger samples and similar methodologies have emerged, although two of these study restaurants instead of hotels \cite[][]{JIA2020104071, HUANG2017117}. One important difference is the geographical focus of their studies. While \cite{FRANCESCO201924} , \cite{JIA2020104071} and \cite{HUANG2017117} have a multi-national focus, we instead focus on Japan. The focus on Japan is important because of its top rank in hospitality across all types of businesses. This raises the question: in such an environment, will the customers be universally satisfied with this factor, or will they have differing views within their cultures? Our study brings light to the changes, or lack thereof, in different touristic environments where an attribute can be considered excellent. The number of samples in other text-mining studies is also smaller than ours in comparison. Apart from that, every study has a different text mining method.

  \subsection{Data mining, machine learning, knowledge discovery and sentiment analysis}\label{theory_data}

    
    In the current world, data is presented to us in larger and larger quantities. Today's data sizes were commonly only seen in very specialized large laboratories with supercomputers a couple of decades ago. However, they are now standard for market and managerial studies, independent university students, and any scientist connecting to the Internet. Such quantities of data are available to study now more than ever. Nevertheless, it would be impossible for researchers to parse all of this data by themselves. As \cite{fayyad1996data} summarizes, data by itself is unusable until it goes through a process of selection, preprocessing, transformation, mining, and evaluation. Only then can it be established as knowledge. With the tools available to us in the era of information science, algorithms can be used to detect patterns that would take researchers too long to recognize. These patterns can, later on, be evaluated to generate knowledge. This process is called Knowledge Discovery in Databases. 

    
    Now, there are, of course, many sources of numerical data to be explored.  However, perhaps what is most available and interesting to managerial purposes is the resource of customers' opinions in text form. Since the introduction of Web 2.0, a never before seen quantity of valuable information is posted to the Internet at a staggering speed. Text mining has then been proposed more than a decade ago to utilize this data \cite[e.g.][]{rajman1998text,nahm2002text}. Using Natural Language Processing, one can parse language in a way that translates to numbers so that a computer can analyze it. Since then, text mining techniques have improved over the years. This has been used in the field of hospitality as well for many purposes, including satisfaction analysis from reviews \cite[e.g][]{berezina2016, xu2016, xiang2015, hargreaves2015, balbi2018}, social media's influence on travelers \cite[e.g.][]{xiang2010}, review summarization \cite[e.g.][]{hu2017436}, perceived value of reviews \cite[e.g][]{FANG2016498}, and even predicting hotel demand using web traffic data \cite[e.g][]{yang2014}.

    More than only analyzing patterns within the text, researchers have found how to determine the sentiment behind a statement based on speech patterns, statistical patterns, and other methodologies. This method is called sentiment analysis or opinion mining. A precursor of this method was attempted decades ago \cite[][]{stone1966general}. With sentiment analysis, one could use patterns in the text to determine whether a sentence was being said with a positive opinion, a critical opinion. This methodology could even determine other ranges of emotions, depending on the thoroughness of the algorithm. Examples of sentiment analysis include ranking products through online reviews \cite[e.g][]{liu2017149, zhang2011}, predicting political poll results through opinions in Twitter \cite[][]{oconnor2010}, and so on. In the hospitality field, it has been used to classify reviewers' opinions of hotels in online reviews \cite[e.g.][]{kim2017362, alsmadi2018}. 

    The algorithm used for sentiment analysis in our study is called a Support Vector Machine. It is a form of supervised machine learning used for binary classification. This means a sample of labeled training data is given to the algorithm to detect patterns in the data and use those patterns to establish a method for classifying other unlabeled data automatically. Machine learning is a general term used for algorithms that, when given data, will automatically use that data to "learn" from its patterns and apply them for improving upon a task. Learning machines can be supervised, as in our study, where the algorithm has manually labeled data to know the correct task result template. Machine learning can also be unsupervised, where without any pre-labeled data. In this latter case, the machine will analyze the structure and patterns of the data and perform a task based on its conclusions. Our study calls for a supervised machine since text analysis can be intricate. Many patterns might occur, but we are only interested in satisfaction and dissatisfaction labels. Consequently, we teach the machine through previously labeled text samples. 

    Machine learning and data mining are two fields with a significant overlap since they can use each other's methods to achieve the task at hand. Machine learning methods focus on predicting new data based on known properties and patterns of the given data. Data mining, on the other hand, is discovering new information and new properties of the data. Our machine learning approach will learn the sentiment patterns of our sample texts showing satisfaction and dissatisfaction and using these to label the rest of the data. We are not exploring new patterns in the sentiment data. However, we are using sentiment predictions for knowledge discovery in our database. Thus our study is a data mining experiment based on machine learning.

    Because the methodology for finding patterns in the data is automatic and statistical, it is both reliable and unpredictable. Reliable in that the algorithm will find a pattern by its nature. Unpredictable in that since it has no intervention from the researchers in making questionnaires, it can result in anything that the researchers could expect. These qualities determine why, much like actual mining, data mining is mostly exploratory. One can never be sure that one will find a specific something. However, we can make predictions and estimates about finding knowledge and what kind of knowledge we can uncover. The exploration of large opinion datasets with these methods is essential. The reason being we can discover knowledge that could be missed by looking at a localized sample rather than a holistic view of every users' opinion. In other words, a machine algorithm can find the needles in a haystack that we did not know were there from taking small bundles of hay at a time.

    In this study, we can predict that several things might occur. Our data could show satisfaction and dissatisfaction factors that are universal, and it could also find strictly cultural factors. However, we expect that both of these options will present themselves. We can also assert that we could arrive at very similar results to previous literature if they are correct in their findings. However, we are using a database of several orders of magnitude larger. We can also expect to discover patterns that researchers previously had not noticed because of the lack of questionnaire design and users' freedom to record their pleasures and grievances.


\section{Methodology}\label{method}

  We have extracted a large number of text reviews from a Chinese portal site \textit{Ctrip} and the travel site \textit{TripAdvisor}. We then determined the most commonly used words that contribute to positive and negative opinions in a review. We did this using Shannon's entropy to extract keywords from their vocabulary. These positive and negative keywords allow us to perform a Support Vector Machine (SVM) based emotional classification of the reviews in large quantities, saving time and resources for the researchers. We classified the sentences in the extracted reviews as emotionally positive or negative, using an optimized Support Vector Classifier (SVC). We then applied a dependency parsing to the reviews and a Part of Speech tagging (POS tagging) to observe the relationship between adjective keywords and other nouns used in the reviews. We split the dataset into price ranges to observe the satisfaction factors and their differences between lower-class and higher-class hotels. We observed the frequency of the terms in the dataset to extract the most utilized words in either review. We show an overview of this methodology in Figure \ref{fig:method-overview}, which is an updated version of the methodology used by \cite{Aleman2018ICAROB}. Finally, we also observed if the satisfaction factors were soft or hard attributes of the hotel. Soft attributes are non-physical aspects of the hotel, those that regard service, staff behavior, or hotel management, which can be changed by the staff or management. Hard attributes are related to physical and environmental aspects, such as facilities,  infrastructure, and the hotel's surroundings, which are impossible to change by the staff alone.

  \subsection{Data collection}\label{datacollection}

    In the data collection stage for Chinese reviews in \textit{Ctrip}, a total of \num[group-separator={,}]{5774} review pages of hotels in Japan were collected. From these pages, we extracted a total of \num[group-separator={,}]{245919} reviews, from which \num[group-separator={,}]{211932} were detected to be standard Mandarin Chinese from mainland China. Since a single review can have sentences with different sentiments, we separated sentences using punctuation marks. The Chinese reviews were comprised of \num[group-separator={,}]{187348} separate sentences. 

    In the \textit{TripAdvisor} data collection, we collected data from \num[group-separator={,}]{21380} different hotels. In total, we collected \num[group-separator={,}]{295931} reviews, from which \num[group-separator={,}]{295503} were detected to be in English. Similarly to the Chinese data, we then separated these English reviews into \num[group-separator={,}]{2694261} sentences using the \textit{gensim} python library. For the language detection in both cases we used the \textit{langdetect} python library.

    However, we needed to make the data and comparisons we draw from each of these datasets fair. For that purpose, we filtered both databases only to contain reviews from hotels in both datasets, using their English names to do a search match. We also filtered them to be in the same date range and cut off reviews outside of each other's date ranges. In addition, we selected only the hotels that had pricing information available. We extracted the lowest price possible for a room or bed for one night, and the highest price possible for one night as well. The difference in pricing can be from better room settings, such as double or twin rooms or suites of several classes, depending on the hotel. Regardless of the reason, the highest-priced room can be an indicator of the hotel's class indirectly, giving us insight into the kind of service offered. After filtering, we found that the number of hotels in common in the data collected was \num[group-separator={,}]{557}. The overlapping date range for reviews was from July 2014 to July 2017. Within these hotels, from \textit{Ctrip} there was \num[group-separator={,}]{48070} reviews comprised of \num[group-separator={,}]{101963} sentences, and from \textit{TripAdvisor} there was \num[group-separator={,}]{41137} reviews comprised of \num[group-separator={,}]{348039} sentences. After filtering the data, we found that the number of reviews was similar for both English and Chinese reviews, but that English reviews tend to be longer in general.

    The price for a night in these hotels ranges from low priced capsule hotels at 2000 yen per night to high-end hotels \num[group-separator={,}]{188000} yen a night as the far ends of the bell curve. Customers' expectations can vary greatly depending on the pricing of the hotel room they stay at. Therefore, we made observations on the distribution of pricing in our database's hotels and binned the data by price ranges, decided by consideration of the objective of stay. We show these distributions in Figure \ref{fig:price_dist}. The structure of the data after division by price is shown in Table \ref{tab:exp_notes}. This table also includes the results of emotional classification after applying our SVC, as explained in \ref{sentimentanalysis}. The first three price ranges (0 to 2500 yen, 2500 to 5000 yen, 5000 to 10,000 yen) would correspond to low-class hotels or even hostels on the lower end and cheap business hotels on the higher end. Further on, there are business hotels in the next range (10,000 to 15,000 yen). After that, the stays could be at Japanese style \textit{ryokan} when traveling in groups, high-class business hotels, luxury love hotels, or higher class hotels (15,000 to 20,000 yen, 20,000 to 30,000 yen). Further than that is more likely to be \textit{ryokan} or high class resorts or five-star hotels (30,000 to 50,000 yen, 50,000 to 100,000 yen, 100,000 to 200,000 yen). Note that because of choosing the highest price per one night in each hotel, the cheapest two price ranges (0 to 2500 yen, 2500 to 5000 yen) are empty, despite some rooms being priced at 2000 yen per night. Because of this, other tables will omit these two price ranges.

  \subsection{Text processing}\label{textprocessing}

    Chinese text, unlike English, does not have spaces between each word to separate them. Besides, we also needed to analyze the grammatical relationship between words, be it English or Chinese, to understand connections between adjectives and nouns. For all these processes, we used the Stanford CoreNLP pipeline developed by the Natural Language Processing Group at Stanford University \cite[][]{manning-EtAl:2014:P14-5}. In order to separate Chinese words for analysis, we used the Stanford Word Segmenter \cite[][]{chang2008}. In the case of texts in English, however, only using spaces is not enough to correctly collect concepts. The English language is full of variations and conjugations of words depending on the context and tense. Thus, a better segmentation is achieved by using lemmatization, which returns each word's dictionary form. For this purpose, we used the \textit{gensim} library with the English texts.

    A dependency parser analyzes the grammatical structure, detecting connections between words, and describing the action and direction of those connections. We show an example of these dependencies in Figure \ref{fig:depparse}. This study uses the Stanford NLP Dependency Parser, as described by \cite{chen-EMNLP:2014}. A list of dependencies used by this parser is detailed by \cite{marneffe_manning_2016_depparse_manual}. In more recent versions, they use an updated dependency tag list from Universal Dependencies \cite[][]{zeman2018conll}. In our study, this step was necessary to extract adjective modifiers and their subject. We did that by parsing the entire database and extracting instances of a few determined dependency codes. One of these dependency codes is ``amod'', which stands for ``adjectival modifier''. This is used when an adjective modifies a noun directly (e.g., A big apple). The other dependency code we used was ``nsubj'', or nominal subject, the class's syntactic subject. We used this one for cases where the adjective is modifying the noun indirectly through other words (e.g., The apple is big). This dependency does not necessarily only include a combination of adjectives and nouns. However, it can also be connected with copular verbs, nouns, or other adjectives. We saw it necessary also to perform a Part of Speech (POS) tagging of these clauses.

    A Part of Speech (POS) tagger is a program that assigns word tokens with tags identifying the part of speech. An example is shown in Figure \ref{fig:postag}. A Part of Speech is a category of lexical items that serve similar grammatical purposes, for example, nouns, adjectives, verbs, or conjunctions. In our study, we used the Stanford NLP POS tagger software, described by \cite{toutanova2000enriching} and \cite{toutanova2003feature}, which uses the Penn Chinese Treebank tags \cite[][]{xia_penntreebank}.

    In this study, we were interested in identifying combinations of adjectives, some verbs, and nouns. We also needed to filter away bad combinations that were brought by the versatility of nominal subject dependencies. For this purpose, we identified the tags for nouns, verbs, and adjectives in Chinese and English, with the English tags being a bit more varied. What would be called adjectives in English corresponds more to stative verbs in Chinese, so we needed to extract those as well. We show a detailed description of the chosen tags in Table \ref{tab:target_postag}. We also show a detailed description of the tags we needed to filter. We selected these tags heuristically by observing commonly found undesired pairs in Table \ref{tab:filter_postag}.

    Once we had these adjective + noun or verb + noun pairs, we could determine what the customers referred to in their reviews. With what frequency they use those pairings positively or negatively.

  \subsection{Sentiment analysis using a Support Vector Classifier}\label{sentimentanalysis}

    The sentiment analysis was performed using the methodology described by \cite{Aleman2018ICAROB}. Keywords are determined by a comparison of Shannon's entropy \cite[][]{shannon1948} between two classes by a factor of \(\alpha\) for one class and \(\alpha'\) for the other, and then they are used in an SVM \cite[][]{cortes1995}, optimizing keywords to select the best performing classifier using the \(F_1\)-measure \cite[][]{powers2011}. The selected SVM keywords would then clearly represent the user driving factors leading to positive and negative emotions. We also performed experiments to choose the best value of the parameter C used in the SVM. C is a constant that affects the optimization process when minimizing the error of the separating hyperplane. Low values of C give some freedom of error, which minimizes false positives. However, depending on the data, it can increase false negatives. Inversely, high C values will likely result in minimal false negatives but a possibility of false positives. SVM performance results are displayed in  Tables \ref{tab:svm_f1_zh} and \ref{tab:svm_f1_en}. Examples of tagged sentences are shown in Table \ref{tab:training_examples}. 

    Shannon's entropy can be used to observe the probability distribution of each word inside the corpus. A word included in many documents will have a high entropy value for that set of documents. Opposite to this, a word appearing in only one document will have an entropy value of zero. 

    An SVM is trained to classify data based on previously labeled data, generalizing the data's features by defining a separating (p-1)-dimensional hyperplane in p-dimensional space. Each dimension is a feature of the data in this space. The separating hyperplane, along with the support vectors, divides the multi-dimensional space and minimizes classification error. 

    Our study used the SVM classification process's linear kernel, defined by the formula (\ref{eq:svm1}) below. Each training sentence is a data point, a row in the vector \(x\). Each column represents a feature; in our case, the quantities of each of the keywords in that particular sentence. The labels of previously known classifications (1 for positive, 0 for negative) for each sentence comprise the \(f(x)\) vector. The Weight Vector \(w\) is comprised of the influences each point has had in the training process to define the hyperplane angle. The bias coefficient \(b\) determines its position.

    During the SVM learning algorithm, each data point classified incorrectly causes a change in the weight vector to correctly classify new data. These changes to the weight vector are greater for features close to the separating hyperplane. These features have stronger changes because they needed to be taken into account to classify with a minimal error. Sequentially, the weight vector can be interpreted as a numerical representation of each feature's effect on each class's classification process. Below we show the formula for the weight vector \(w\) (\ref{eq:svm_weight}), where \(x\) is the training data and each vectorized sentence \(x_i\) in the data is labeled \(y_i\). Each cycle of the algorithm alters the value of \(w\) by \(\alpha\) to reduce the number of wrong classifications. This equation shows the last value of \(\alpha\) after the end of the cycle.

    We tagged 159 Chinese sentences and \num[group-separator={,}]{2357} English sentences as positive or negative for our training data. The entropy comparison factors \(\alpha\) and \(\alpha'\) were tested from 1.25 to 6 in intervals of 0.25. We applied this SVC to classify the rest of our data collection. Subsequently, the positive and negative sentence counts shown in Table \ref{tab:exp_notes} result from applying our SVC for classification.

\section{Data Analysis}\label{dataanalysis}

  \subsection{Frequent keywords in differently priced hotels}\label{svmresults}

    To understand Chinese tourists and English-speaking tourists' satisfaction and dissatisfaction factors when lodging in Japan, we study both the frequency of the words they use. Following that, to know the relevance of a keyword as a preference for each group, we observed each entropy-based keyword's frequencies in our complete data set and in each price range. The frequency of the keywords in the database shows the level of priority it has for customers.

    We observed the top 10 words with the highest frequencies for keywords linked by entropy to satisfaction and dissatisfaction in emotionally positive and negative statements to study. The keywords are the quantitative rank of the needs of Chinese and English speaking customers. We show the top 10 positive keywords for each price range comparing English and Chinese in Table \ref{tab:freq_res_pos}. For the negative keywords, we show the results in Table \ref{tab:freq_res_neg}.

    We can observe that the most used keywords for most price ranges in the same language are similar, with a few changes in priority for the keywords involved. For example, in Chinese, we can see that the customers praise cleanliness first in cheaper hotels, whereas the size of the room or bed is praised more in hotels of higher class. Another example is that in negative English reviews, complaints about price appear only after 10,000 yen hotels. After this, it climbs in importance following the increase in the hotel's price.


  \subsection{Frequently used adjectives and their pairs}\label{adjresults}

    Some keywords in these lists are adjectives, such as the word ``\begin{CJK}{UTF8}{gbsn}大\end{CJK} (big)'' mentioned before. To understand those, we performed the dependency parsing, and part of speech tagging explained in section \ref{textprocessing}. While many of these connections, we only considered the top 4 used keyword connections per adjective per price range. We show the most used Chinese adjectives in positive keywords in Table \ref{tab:adj_zh_pos}, and for negative Chinese adjective keywords in Table \ref{tab:adj_zh_neg}. Similarly, for English adjectives used in positive sentences we show the most common examples in Table \ref{tab:adj_en_pos}, and for adjectives used in negative sentences in Table \ref{tab:adj_en_neg}.

  \subsection{Determining hard and soft attribute usage}\label{det_hard_soft}

    To further understand the differences in satisfaction and dissatisfaction in Chinese and Western customers of Japanese hotels, we classified these factors as either hard or soft attributes of a hotel. We define hard attributes as matters regarding the hotel's physical or environmental aspects, such as facilities, location, or infrastructure. Some of these aspects would be impractical for the hotel to change, such as its surroundings and location. Others can be expensive to change, such as matters requiring construction costs, which are possible but regarding infrastructure investment. On the other hand, soft attributes are the non-physical attributes of the hotel service and staff behavior that are practical to change through management. For example, the hotel's services or the cleanliness of the rooms are soft attributes. For our purposes, amenities, clean or good quality bedsheets or curtains, and other physical attributes that are part of the service and not of the hotel's physical structure are also considered soft attributes. Thus, we can observe the top 10 satisfaction and dissatisfaction keywords and determine whether they are soft or hard attributes.

    We manually labeled each language's top keywords into either hard or soft by considering how the word would be used when writing a review. If the word is describing unchangeable physical factors by the staff or management, we consider them hard. If the word implies an issue that could be solved or managed by the hotel staff or management, we consider it soft. For adjectives, we looked at the top 4 adjective and noun pairings used in the entire dataset. We counted the percentage of usage in each context. If it is not clear from the word or the pairing alone, we declare it undefined. Then, we added the counts of these words in each category. A single word with no pairing is always 100\% in the category it corresponds to. We add the partial percentages for each category when an adjective includes various contexts. The interpretation of these keywords is shown in the Tables \ref{tab:zh_hard_soft_keywords} and \ref{tab:en_hard_soft_keywords}. We can see the summarized results for the hard and soft percentages of positive and negative Chinese keywords in Figure \ref{fig:hard_soft_zh}. For the English keywords, see Figure \ref{fig:hard_soft_en}.

\section{Results}\label{results}

  \subsection{Experiment results and answering research questions}

    Our research questions were about two things. In research questions \ref{rsq:hospitality} and \ref{rsq:hospitality_both}, we decide that the objective of this study to determine how Chinese and Western tourists interact with the \textit{omotenashi} culture influenced hospitality and service in Japan, and how are they different in their perceptions in this matter. We observed the top-ranking positive factors for Chinese tourists across different price ranges in Table \ref{tab:freq_res_pos}, and specifically the word ``\begin{CJK}{UTF8}{gbsn}不错\end{CJK} (not bad)'' and its pairings in Table \ref{tab:adj_zh_pos}. From these observations, we can infer that, while service, cleanliness, and breakfast are praised in most hotels, location is usually placed above it in importance on the pairings. When we see the rest of the factors lower on the list, we see that the list is more populated with hard attributes like location and transportation availability across different price ranges. From the negative keyword usages in Table \ref{tab:freq_res_neg}, there are complaints about the lack of a Chinese friendly environment. However, most complaints are also about hard attributes such as the building's age and the distance from other convenient spots. Nevertheless, the most complained about aspect is the price of the hotel. Surprisingly, all of the price ranges have this negative keyword at the top of the list, suggesting that it is the main concern to Chinese customers with different travel purposes.

    On the other hand, the word ``staff'' is the second or third in the lists of satisfaction factors in English-written reviews in all the price ranges. This word is followed by a few other keywords lower in the top 10 list, such as ``helpful'' or ``friendly''. When we look at the pairings of the top-ranked keyword ``good'' in Table \ref{tab:adj_en_pos}, we find that customers mostly praise the location, service, breakfast, or English availability. When we look at the negative keyword ``poor'' and its pairings in Table \ref{tab:adj_en_neg}, we see that it is also service-related concepts that the Western tourists are disappointed with. With these results, we can observe that both Chinese and English-speaking tourists in Japan have different priorities. However, both populations consider the hotel's location and transport availability (subways and trains) nearby as secondary but still essential points in their satisfaction with a hotel. The Chinese customers are primarily satisfied with the room quality in spaciousness and cleanliness and the service of breakfast.

    In contrast, the English-speaking customers are easily upset by any lack of cleanliness and smoke smell from cigarettes. Surprisingly, the cigarette smell is an issue even in the middle to high-class hotels above 30,000 yen per night. However, above 50,000 yen per night, this problem seems to disappear from the list of top 10 concerns. Old and dated buildings seem to be a concern for both populations. On the positive side, for all price ranges considered, English-speaking tourists value staff friendliness over room quality when considering their satisfaction. In contrast, Chinese tourists consider location and transportation more often.

    We also can observe some keywords that are not considered by their counterparts. For example, English-speaking customers mentioned tobacco smell in many reviews. However, it was not statistically identified as a problem for their Chinese counterparts. On the other hand, while it appears in both English and Chinese lists, references to ``\begin{CJK}{UTF8}{gbsn}购物\end{CJK} (shopping)'' are more common in the Chinese lists across hotels of 15,000 yen to 200,000 yen per night. Meanwhile, the term ``shopping'' only appears in the 20,000 to 30,000 yen per night top 10 positive keywords list for English-speakers.

    In our research questions \ref{rsq:hard_soft} and \ref{rsq:hard_soft_diff}, we ponder how customers of both cultural backgrounds evaluate hard and soft attributes of the hotel and how they differ in those evaluations. Here we define hard attributes as those relating to the hotel's physical, structural, or environmental aspects. These are often impossible or impractical to change by the hotel management and staff, such as facilities, infrastructure, the surroundings, view, or location convenience. On the other hand, hotel staff and management can change soft attributes, for example, by improving the hotel's services via training or hiring specialized staff, improving the quantity or quality of amenities, bedsheets, or general cleanliness. Our study found that Chinese tourists are mostly positively reacting more to the hotel's hard attributes. There is a slightly hard leaning (53\%) concern with hard attributes in negative sentences, albeit this is more uniform than the positive evaluations. English-speaking tourists, on the other hand, are both positively and negatively more responsive to soft attributes. In the case of negative keywords, English-speaking tourists are overwhelmingly more concerned with the hotel's soft attributes dissatisfaction somehow.

    One factor that both populations have in common is, when perceiving the hotel negatively, ``\begin{CJK}{UTF8}{gbsn}老\end{CJK} (old)'', ``dated'', ``outdated'', or ``\begin{CJK}{UTF8}{gbsn}陈旧\end{CJK} (obsolete)'' aspects of the room or the hotel are being criticized across, surprisingly, most price ranges. This is, however, a hard attribute and is unlikely to change for most hotels.
    
  \subsection{Chinese tourists - A big and clean space}\label{disc:zh}

    We found that mainland Chinese tourists are satisfied mostly by Japanese hotels' big and clean spaces. From the adjectival pairings that we extracted with dependency parsing and POS tagging in Table \ref{tab:adj_zh_pos}, we can observe that mostly they mean big and clean rooms. Other mentions are also big markets nearby or a big bed. We can observe that across different price ranges, the usage of the word ``\begin{CJK}{UTF8}{gbsn}大\end{CJK} (big)'' increases as the hotel increases in price. However, we can see that they still react positively in a significant manner in cheaper hotels. Inspecting closer by taking random samples of the pairs of ``\begin{CJK}{UTF8}{gbsn}大 空间\end{CJK} (big space)'' or ``\begin{CJK}{UTF8}{gbsn}大 面积\end{CJK} (large area)'', we can see that there are also many references to the public bathing facilities in the hotel. We can also see them mentioned as a word pairing ``\begin{CJK}{UTF8}{gbsn}棒 温泉\end{CJK} (great hot spring)''. In Japan, there is what is called ``\begin{CJK}{UTF8}{min}銭湯\end{CJK} (\textit{sent\=o})'', which are artificially made public bathing facilities, on occasions including saunas and baths with unique qualities. On the other hand, there are natural hot springs, called ``\begin{CJK}{UTF8}{gbsn}温泉\end{CJK} (\textit{onsen})''. They can either be bathing in the natural source of the water or using the hot springs in artificially made bath facilities. It is a Japanese custom and culture that all customers use the facilities after cleaning themselves in a shower and go into the baths without any clothes. It can be a cultural shock for many tourists, but this is a fundamental attraction for many. 

    However, the size of the room or the bed is a hard attribute. Without considering rebuilding the hotel, it is not trivial to improve on. On the other hand, cleanliness is mostly relating to soft attributes when we observe its adjectival pairings. We can observe pairs such as ``\begin{CJK}{UTF8}{gbsn}干净 房间\end{CJK} (clean room)'' at the top rank of all price ranges, and then variably ``\begin{CJK}{UTF8}{gbsn}干净 酒店\end{CJK} (clean hotel)'', ``\begin{CJK}{UTF8}{gbsn}干净 总体\end{CJK} (clean overall)'', ``\begin{CJK}{UTF8}{gbsn}干净 环境\end{CJK} (clean environment)'', and ``\begin{CJK}{UTF8}{gbsn}干净 设施\end{CJK} (clean facilities)'', among other examples. In negative reviews, there is a mention of criticizing the ``\begin{CJK}{UTF8}{gbsn}一般 卫生\end{CJK} (general hygiene)'' of the hotel, although it is an uncommon pair. Therefore, we can assert that cleanliness is an important soft attribute for Chinese customers and that they are mostly pleased with their expectations being met. 

    One key component we found in Chinese customer satisfaction soft factors is the inclusion of breakfast within the hotel. While other food-related words were extracted, most of them were general, like ``food'' or ``eating,'' which were lower ranking. In contrast, the word ``\begin{CJK}{UTF8}{gbsn}早餐\end{CJK} (breakfast)'' refers possibly to its inclusion in the hotel commodities, was frequently used in positive texts compared to other food-related words. The word ``\begin{CJK}{UTF8}{gbsn}早餐\end{CJK} (breakfast)'' is also observed across all price ranges, although at different priorities in each of them. However, we assert that it is an important factor. Observing word pairs from the positive Chinese keywords in Table \ref{tab:adj_zh_pos}, we can also see that ``\begin{CJK}{UTF8}{gbsn}不错\end{CJK} (not bad)'' is paired as ``\begin{CJK}{UTF8}{gbsn}不错 早餐\end{CJK} (nice breakfast)'' in four of the seven price ranges with reviews available as part of the top 4 pairings. It is only slightly lower on other categories, although it is not shown on the table. Thus we consider that a recommended strategy for hotel management is to invest in the inclusion or betterment of hotel breakfast to increase good reviews.

  \subsection{Western tourists - A friendly face, and absolutely clean}\label{disc:en}

    From the satisfaction factors of English-speaking tourists, we can see that at least three words relate directly to staff friendliness and services, being ``staff'', ``helpful'' and ``friendliness'' in the general database. The word ``staff'' is the second most frequently used word for satisfied customers across most price ranges, and only third in one of them. Adding to that, ``helpful'' and ``friendly'' follow it lower in the list in most price ranges. The word ``good'' is mostly about the location, the service, breakfast, or English availability in Table \ref{tab:adj_en_pos}. Like Chinese customers, Western customers also seem to enjoy the included breakfasts regarding their satisfaction keyword pairings. However, the word does not appear directly in the top 10 list as in their Chinese counterparts. The words ``helpful'' and ``friendly'' are mostly paired with ``staff'', ``concierge'', ``desk'', and ``service''. When we look at the negative keyword ‘poor’ and its pairings in Table \ref{tab:adj_en_neg}, we see that it is also service-related concepts that the Western tourists are disappointed with when they react negatively.

    Another soft attribute that is high on the list for most of the price ranges is the word ``clean''. Since it is an adjective, we have explored the word pairings as well. Customers are mostly praising ``clean rooms'' and ``clean bathrooms'', while also referring to the hotel in general. It seems that when observing the negative keyword frequencies for English-speakers, we can find words such as ``dirty'', ``carpet'', and from the word pairings ``dirty carpet'', ``dirty room'', and ``dirty bathroom''. Along with complaints about off-putting smells, we can conclude that Western tourists have high expectations about cleanliness when traveling in Japan.

    An interesting detail of the keyword ranking is that the word ``comfortable'' is high on the satisfaction factors and ``uncomfortable'' high on the dissatisfaction factors. The words are paired with nouns like ``bed'', or ``room'', ``pillow'' or ``mattress'', generally referring to their sleep conditions in the hotel.
    It seems that Western tourists are highly sensitive to comfort levels in the hotels and whether it reaches their expectations. The ranking for the negative keyword ``uncomfortable'' is similar across most price ranges, except the two most expensive ones, where this keyword disappears from the top 10 list.

    While less high in priority, the price range of 15,000 to 20,000 yen hotels also mentions ``free'' as one of the top 10 positive keywords, paired mostly with ``wifi''. This price range is mostly for business hotels, where we infer users would be expecting this feature the most. Western tourists are highly sensitive to comfort levels in the hotels and whether it reaches their expectations.


  \subsection{Tobacco, what's that smell?}\label{disc:tobacco}

    A concern for Western tourists was the smell of tobacco in their room, which can be considered a soft attribute. Tobacco was found not only as a standalone word ``cigarette'', but also as word pairs in Table \ref{tab:adj_en_neg}. We can find other related word pairs such as ``funny smell''. Upon manual inspection of a sample of reviews with this keyword, we found that the room was often advertised as non-smoking, yet, the smell permeated the room and curtains. Another common complaint was that there were no non-smoking facilities available at all in the first place. The smell of smoke can completely ruin some customers’ stay and give a bad impression to review writers, lowering the number of future customers.

    However, in comparison, Chinese customers seem not to be bothered by this at all. We consulted studies involving the use of tobacco in different countries. Previous research states that 49 - 60\% of Chinese men (and 2.0 - 2.8\% of women) currently smoke or have smoked before. This was taken from a sample of 170,000 Chinese adults in 2013-2014, which is high compared to many English-speaking countries \cite[][]{zhang2019tobacco,who2015tobacco}.

    Japan has a polarized view on the topic of smoking. Despite being one of the world’s largest tobacco markets, its use has decreased in recent years. Smoking in public spaces is prohibited in some wards of Tokyo (namely Chiyoda, Shinjuku, and Shibuya). However, it is generally only urged and not mandatory to have smoking restrictions in restaurants, bars, hotels, and public areas. However, many places have designated smoking rooms are available to keep the smoke in an enclosed area and avoid bothering others. Despite this, businesses, especially those who cater to certain customers, will generally be discouraged from having smoking restrictions if they want to keep their clientele. If Japanese hotels want to cater to all kinds of customers, Western and Asian alike, they must provide spaces without tobacco smell. After all, even if it does not bother a few customers, the lack of smell would make it an appropriate space for all customers.


  \subsection{Location, location, location}\label{disc:location}

    The hotel's location, closeness to the subway and public transport, and nearby shops' availability were observed to be of importance to both Chinese and English-speaking tourists. In positive word pairings Tables \ref{tab:adj_zh_pos} and \ref{tab:adj_en_pos}, we can find pairs such as ``\begin{CJK}{UTF8}{gbsn}不错 位置\end{CJK} (nice location)'', ``\begin{CJK}{UTF8}{gbsn}近 地铁站\end{CJK} (near subway station)'', ``\begin{CJK}{UTF8}{gbsn}近 地铁\end{CJK} (near subway)'' in Chinese texts and ``good location'', ``great location'', and ``great view'', as well as single keywords ``location'' and ``shopping'' for English-speakers, and ``\begin{CJK}{UTF8}{gbsn}交通\end{CJK} (traffic)'', ``\begin{CJK}{UTF8}{gbsn}购物\end{CJK} (shopping)'', ``\begin{CJK}{UTF8}{gbsn}地铁\end{CJK} (subway)'', and ``\begin{CJK}{UTF8}{gbsn}环境\end{CJK} (environment or surroundings)'' for Chinese speakers. All of these keywords and their location in each population's priorities across the price ranges signal that while it was not the priority for either of them, the hotel's location is a secondary but still important point in the hotel's satisfaction. However, since this is a hard attribute, unchangeable to the hotel's management, it is not often considered in the literature. Upon inspection of examples from the data, we found that most customers were satisfied if the hotel was near to at least two other subjects: subway, train, and convenience stores. 

    Japan is a country with a peculiar public transport system. The rush hour makes for a subway filled to the brim with people in suits making their commute, and trains and subway stations in Tokyo create a confusing public transport map for a visitor. Buses are also available, although less used than the rail systems in the big cities. These three are unusually affordable in price. Then there are the more expensive transports, such as the bullet train \textit{shinkansen} for traveling across the country, and taxis. Taxis in Japan are a luxury compared to other countries. In less developed countries, a taxi is the cheap method of transport of choice. In Japan, taxis are made to provide a high-quality experience with a matching price. This means that for tourists, subway availability and maps or GPS applications and a plan to travel the city are of utmost necessity. 

    Japanese convenience stores, on the other hand, are also famous worldwide. Japanese convenience stores are a haven for the traveler in need. It offers anything, from drinks and snacks to full meals, copy and scanning machines, alcohol, cleaning supplies, personal hygiene items, underwear, towels, international ATMs, among other things. If some trouble occurred, or a traveler forgot to pack a particular item, it is almost sure that they can find it. 

    Therefore, considering that both transport systems and nearby shops are points of interest for Chinese and Western tourists, Japanese hotels have to carefully choose their location from the moment they are constructed. While not a top priority, this is a universal factor for both customer groups, and it can be an instant way to generate positive reviews.

\section{Discussion}\label{discussion}

  Below we explore the possible interactions with Japanese hospitality, the differences between Chinese and Western tourists, the possible cause for them, how they vary across different price ranges, and what they imply for the industry. We also discuss the differences between the hotel's hard and soft attributes and how they contribute to customers' satisfaction.

  \subsection{Western and Chinese tourists in the Japanese hospitality environment}\label{disc:omotenashi}

    To this day, scholars continue to correct our historical bias towards the west. Studies have determined that different cultural backgrounds lead to different expectations, which influences tourists' satisfaction. Meaning, tourists of a particular culture will have different leading satisfaction factors across different destinations. However, Japan presents a particular environment. The spirit of hospitality and service, \textit{omotenashi}, excels and is considered the highest standard across the world. Can such an environment affect different cultures equally? Or is it only attractive to certain cultures? Our study brings light to these questions.

    Our results indicate that out of the two, Western tourists are the most satisfied with soft attributes, such as friendly and helpful staff in Japan. As explained earlier in this paper, Japan is famous for its customer service all over the world. Respectful language and bowing are not exclusive to high priced hotels or businesses. These can even be found in convenience stores. The level of hospitality in even the cheapest of convenience stores is starkly different from Westerner experiences. While it could be a culture shock to some, it is mostly seen positively. After all, the Japanese staff respectfully treats all customers. However, for some customers, this could be the best way they have been treated until that moment. Now, in higher priced hotels, the adjectives used to praise the service go from normal descriptors like ``good'' to higher levels of praise like ``wonderful staff'', ``wonderful experience'', ``excellent service'', and ``excellent staff''. We can also see that \cite{kozak2002} and \cite{shanka2004} have also found that hospitality and staff friendliness is a vital determinant in Western tourists' satisfaction.

    However, we can see from the negative English keywords that a big part of the dissatisfaction with Japanese hotels stems from a lack of hygiene and room cleanliness. Although Chinese customers only had positive keywords about cleanliness, English-speaking customers have found many places unacceptable to their standards. This is particularly true at hotels below 50,000 yen per night. The most common complaint regarding cleanliness was about the carpet, followed by complaints about cigarette stench and general dirtiness. \cite{kozak2002} also found that hygiene and cleanliness were essential satisfaction determinants for Western tourists. However, in the previous literature, this was linked merely to satisfaction. In comparison, our research uncovered that words relating to cleanliness are mostly linked to dissatisfaction. Westerners could be said to have a high standard of room cleanliness when compared to their Chinese counterparts.

    According to previous research, we can see that Western tourists are already inclined to appreciate hospitality for their satisfaction. When presented with Japanese hospitality, this expectation is met and overcome. In contrast, we can see from our results that Chinese tourists had less focus on hospitality, staff, or service and were more concerned with room quality. However, when analyzing the word pairs for ``\begin{CJK}{UTF8}{gbsn}不错\end{CJK} (not bad)'' and for ``\begin{CJK}{UTF8}{gbsn}棒\end{CJK} (great)'', we can see that they do praise staff, service and breakfast. Observing the percentage of hard to soft attributes in Figure \ref{fig:hard_soft_zh}, however, we know that Chinese customers are satisfied more with hard attributes, compared to the Western tourists who seem to be meeting more than their expectations.

    It could be that Chinese culture does not expect high-level service initially. When an expectation that is not held is met, the satisfaction that stems from this is less than if it was expected. On the other hand, we have the phenomenon of a ``nice surprise'': When an unknown need is unexpectedly met, there is more satisfaction. It is necessary to note the difference between these two phenomenons. The ``nice surprise'' fulfills a need unexpectedly. Perhaps the hospitality grade in Japan does not fulfill a high enough need for the Chinese population, resulting in less satisfaction. For greater satisfaction, the existence of a need being met is necessary. However, the word ``not bad'' is at the top of the list at most price ranges, and one of the uses is related to service. Thus, we cannot say that they are not satisfied with this matter. Rather, they hold other factors at a higher priority, considering the keyword frequency is higher for other pairings.

    Another possibility presents itself when we observe the Chinese tourists’ dissatisfaction factors. Chinese tourists may have expectations about the Chinese visitors' treatment that are not being met, even in this high standard hospitality environment. Japan is known worldwide for their hospitality, but they are also known historically to be monolingual and have a relatively large language barrier \cite[][]{heinrich2012making,coulmas2002japan}. While the Japanese effort to accommodate English speakers is slowly taking shape, Chinese accommodations can be lagging. Chinese language pamphlets, Chinese texts on instructions for the hotel room, and its appliances and features (e.g., T.V. channels, Wi-Fi setup), or just the treatment towards Chinese people could be examples. It is natural to be dissatisfied since traveling in a strange land without knowing the language can be a daunting experience. \cite{ryan2001} also found that communication difficulty was one of the main reasons Chinese customers would state for not visiting again. It seems like this is a problem that is not singular to Japan.

    Our initial question was whether the environment of high-grade hospitality would affect both cultures equally. This study brought us closer to the answer. On the one hand, there is a possibility that Chinese customers did have high-grade hospitality and did not get equally satisfied with Westerners. In that case, it appears that the difference stems from a psychological source. Expectation leads to satisfaction and a lack of expectation results in lesser satisfaction. On the other hand, there is also a possibility that Chinese customers are not receiving the highest grade of hospitality because of cultural friction between Japan and China.

    It is unclear from our results which of these could be the case. One thing is clear for hotel managers, however. Competing in hospitality and service does include language services, especially in the international tourism industry. Better multilingual support can only improve that already high standard in Japan. Considering that most of the tourists in Japan come from other countries in Asia, this is an endeavor that truly can bring benefits to their investment. Proposals for this endeavor include hiring Chinese speaking staff, preparing pamphlets in Chinese, or have a translator application readily available with staff trained in interacting through an electronic translator.

  \subsection{Hard vs. soft satisfaction factors}\label{disc:hard_soft}

    As we stated in section 3.2, previous research is focused mostly on the hotel's soft attributes and their influence on customer satisfaction. Examples of soft attributes include staff behavior, commodities, amenities, and appliances that can be improved within the hotel \cite[e.g.][]{shanka2004,choi2001}. However, hard attributes are not usually analyzed in satisfaction studies. Examples of hard attributes include the hotel's location relative to public transport and shops, language immersion of the country, noise pollution, or weather. Because our study left the satisfaction factors to be decided statistically via customers’ online reviews, we can see the importance of those hard or soft attributes in their priorities.

    Figure \ref{fig:hard_soft_zh} shows that in regards to Chinese customer satisfaction, in general, 68\% of the top 10 keywords are hard factors. In contrast, only 20\% are soft factors. The rates are similar for most price ranges, excepting the highest-priced hotels, where 35\% of the keywords are undefined. However, the soft attributes are still similar at 18\%. However, two of these managerial words are all concentrated at the top of the list (``\begin{CJK}{UTF8}{gbsn}不错\end{CJK} (not bad)'', ``\begin{CJK}{UTF8}{gbsn}干净\end{CJK} (clean)''), plus the adjective pairs relating to soft attributes of ``\begin{CJK}{UTF8}{gbsn}不错\end{CJK} (not bad)'' which are at the top in most price ranges as well. Chinese tourists could expect spaciousness and cleanliness when coming to Japan. That expectation could be caused by reputation, previous experiences, or cultural backgrounds. Some scholars argue that different cultures have different room size perceptions \cite[][]{Saulton2017}. Although the study subjects are German and South Korean, the study presents the results as differences influenced by Asian and Western cultures. We argue that one country is not representative of others’ cultures, so there can be differences between South Korea and China in room size perception. However, an interesting point appears. It could be that a different room size perception affects the satisfaction of Chinese tourists in contrast with Westerners. Westerners only start placing a priority on praising room size as the price of the hotel goes up. We can compare these results with previous literature, where traveling Chinese tourists choose their destination based on several factors, including cleanliness, nature, architecture, and scenery \cite[][]{ryan2001}. These other few factors found in previous literature could be linked to the keyword ``\begin{CJK}{UTF8}{gbsn}环境\end{CJK} (environment or surroundings)'' as well. This keyword is present in hotels priced at more than 20,000 yen per night. 

    In comparison, English speakers are mostly satisfied with the hotel's soft attributes. Figure \ref{fig:hard_soft_en} shows that soft attributes are above 48\% in all price ranges, the highest being 65\% in the 15,000 to 20,000 yen per night price range. This price range corresponds to affordable business hotels, for example. English-speaking customers also have soft attributes at the top of their list. The exception is the hard attribute that is the hotel's location, which is consistently around the middle of the top 10 lists for all price ranges. If one considers both Chinese and Western tourists’ satisfaction, a hotel can improve to attract more customers in the future. If it was the other way around, and the satisfaction was related more with hard attributes overall for 1020 both cultures, hotels would have to compete solely on their location.

    For both customer groups, the main reason for dissatisfaction is pricing, which can be interpreted as a concern about value for money. However, it is interesting to note that while English-speaking customers complain about price with a lower rank in lower-priced hotels. In contrast, the Chinese customers consistently have ``\begin{CJK}{UTF8}{gbsn}价格\end{CJK} (price)'' as a top or second-most concern across all price ranges. A paper studying Chinese tourists found that they had this concern \cite[][]{truong2009}. However, our results indicate that this is less of a cultural attribute in Japanese hotels and has more to do with the pricing of hotels overall. The tourists coming to Japan could be both experienced travelers or first-time travelers. However, the fact is that their expectation of the price for hotels was lower than what they found in Japan. In general, Japan is an expensive place to visit, impacting this placement in the ranking. Space is scarce in Japan, and capsule hotels with cramped spaces of 2 x 1 meters cost around 3,000 to 6,000 yen per night. Bigger business hotel rooms are relatively expensive, ranging from 5,000 to 12,000 yen per night. For comparison, hotels in the USA with a similar quality can be half the price.

    Around half of the dissatisfaction factors for both Chinese and Western customers are caused by issues that could be solved with improved management. The previous is true for all price ranges. Of course, the improvements could be staff training (perhaps in language), hiring professional cleaning services for rooms with cigarette smoke smells, or improving the bedding. All of these options can be costly. However, this paper provides a good guideline for which factors to consider first and which ones will be best suited to each customer group. Hotels can use the price range categorization in order to choose the appropriate strategy as well. However, once the hotel's location and construction are set for Chinese customers, not much else can be done to satisfy them further. As mentioned before, Chinese language availability is another soft attribute that can be improved with staff and training investment.

    On the other hand, Western tourists are all around dissatisfied with mostly soft attributes. They show this by having a low of 35\% in the highest price range where undefined factors are the majority and 78\% at most in the price range from 30,000 to 50,000 yen per night in a hotel. The room for improvement for Western tourists is more extensive than their Chinese counterparts. As such, it presents a bigger investment opportunity. As mentioned earlier in this paper, Westerners are known as ``long-haul'' customers, spending more than 45\% of their budget on hotel lodging.
    On the other hand, their Asian counterparts only spend 25\% of their budget on hotels \cite[][]{choi2000}. With bigger returns on managerial improvements, it seems like we can recommend investing in improving attributes that dissatisfy Western customers, such as cleanliness and removing tobacco smell. Making more hotel facilities tobacco-free and deodorizing the rooms can be a low-cost investment that could increase returns many times over.

    However, the opposite argument could also be made that Chinese customers provide a more significant number of customers, even though they tend to spend less on lodging. Attracting a large number of Chinese customers can be a viable strategy for hotels. However, as mentioned before, they tend to focus more on hard attributes, leaving language barrier-breaking as one of the few strategies to accomplish this.

    The basic premise of this study is that different cultures lead to different expectations and satisfaction factors. This premise also plays a role in the differentiation between the preference of hard or soft attributes.

    In \cite{donthu1998cultural}, subjects from 10 different countries were compared in their expectations of service quality and analyzed through the lens of Hofstede's typology of culture \cite[][]{hofstede1984culture}. That study states that although culture has no one specific index, five dimensions of culture can be used to analyze or categorize a country in comparison to others. These are \textit{power distance}, \textit{uncertainty avoidance}, \textit{individualism-collectivism}, \textit{masculinity-femininity}, and \textit{long-term versus short-term orientation}. In each of these dimensions, at least one element of service expectations was found to be significantly different for countries grouped under contrasting attributes (e.g., individualistic countries vs. collectivist countries, high uncertainty avoidance countries vs. low uncertainty avoidance countries). However, Hofstede's typology has received criticism from academics, particularly the fifth dimension that Hofstede proposed, which was added afterward with the alternate name \textit{Confucian dynamic}. Academics with a Chinese background criticized Hofstede for being misinformed on the philosophical aspects of Confucianism, as well as being a difficult dimension to measure \cite[][]{fang2003critique}. Other models, such as the GLOBE model, also take issue with some of Hofstede's dimensions and replace them with others, making a total of 9 dimensions \cite[][]{house1999cultural}. The \textit{masculinity-femininity} dimension, for example, is proposed to be instead two dimensions: \textit{gender egalitarianism} and \textit{assertiveness}. This addition of dimensions avoids assuming that assertiveness is either masculine or feminine, which stems from outdated gender stereotypes. Gender stereotypes such as these have also been the subject of critique for Hofstede's model\cite[][]{jeknic2014gender}. Our study agrees with these critiques and, therefore, will avoid considering these for our discussion.
    
    The backgrounds of collectivism in China and individualism in Western countries have been studied before \cite[][]{gao2017chinese}. These backgrounds and the differences in these cultural dimensions could be the underlying cause for differences in expectations. Regardless of the cause, however, measures in the past have proven that these differences in expectations exist \cite[][]{armstrong1997importance}. 

    For our purposes in contrasting Western vs. Chinese satisfaction stemming from expectations, these dimensions could explain why Chinese customers are generally satisfied more often with hard factors while Westerners are satisfied or dissatisfied with soft factors. Perhaps the cultural background of Chinese tourists emphasizes their surroundings and their place in nature and the environment. Chinese historical backgrounds of Confucianism, Taoism, and Buddhism permeate the thought processes of Chinese populations. However, scholars argue that the changes in generations and their economic and recent history gives less importance to these concepts in their lives \cite[][]{gao2017chinese}. 

    Nevertheless, one could argue that a Chinese cultural attribute emphasizes the environment and the place one is in towards satisfaction, rather than the way one is treated. According to previous research, Chinese tourists are collectivist, while Westerners are individualists \cite[][]{kim2000}. A more anthropocentric and individualistic Western culture could result in more of their expectations and priorities be related to how one is treated in social circumstances, rather than the environment one is in. According to \cite{donthu1998cultural}, highly individualistic customers have a higher expectation of empathy and assurance from the provider than collectivist customers. Empathy and assurance from the provider are aspects of service, a soft attribute of a hotel. 

    Among other dimensions in both models, we can consider uncertainty avoidance. High uncertainty avoidance customers would carefully plan their travel and therefore have higher expectations towards service. On the other hand, lower uncertainty avoidance customers would have certain room for risk in their decisions, and therefore face less disappointment with different expectations. However, according to \cite{xiumei2011cultural}, the difference between China and the USA in uncertainty avoidance is not so clear when measuring with the Hofstede typology and the GLOBE typology. While the USA is not representative of Western society, this dimension might not be the one causing the difference in hard-soft attribute satisfaction between Chinese and Western cultures. Another factor, power distance, was also different when measured by Hofstede's method compared to the GLOBAL method, so we decided against making this comparison.

  \subsection{Satisfaction across different price ranges}\label{disc:price}

    In previous sections of this paper, we have mentioned the differences reflected in hotel price ranges. Nevertheless, it is interesting to discuss this further. The most visible change in satisfaction factors across differently priced hotels is the change in voice to describe their satisfaction with the same topics. We can know this by observing the adjective + noun pairs and finding pairs with different adjectives for the same nouns. For example, in English, words describing nouns such as ``location'' or ``hotel'' are ``good'' or ``nice'' in lower-priced hotels. In contrast, the adjectives that pair with the same nouns for more highly-priced hotels are ``wonderful'' and ``excellent''. In Chinese, the change goes from ``\begin{CJK}{UTF8}{gbsn}不错\end{CJK} (not bad)'' to ``\begin{CJK}{UTF8}{gbsn}棒\end{CJK} (great)'' or ``\begin{CJK}{UTF8}{gbsn}赞\end{CJK} (awesome)''. We can infer that the level of satisfaction is high and influences how customers write their reviews. However, when we look at the negative keywords, the change is from ``annoying'' or ``worst'', to ``disappointing''. Here we can see how expectations influence satisfaction and dissatisfaction in different ways. 

    In this paper, we follow the definition of satisfaction by \cite{hunt1975}, where meeting or exceeding expectations produces satisfaction. Therefore, the lack of meeting expectations would cause dissatisfaction. In the cases above, we can infer that a customer that pays more for a higher class of experience has higher expectations. This is true in dissatisfaction, where their expectation is higher in a more expensive hotel. As such, any lack of cleanliness can lead to disappointment or outrage. In the case of English-speaking customers in the 30,000 to 50,000 yen per night price range, cigarette smell is particularly disappointing. However, we consistently see customers with high expectations for high-class hotels react even more positively when satisfied. In the positive case, expectations appear to be exceeded in most cases, judging their reactions. We argue that these are two different kinds of interactions with expectations. We can observe logical expectations. Customers set a standard in their mind that the service must not fall below or be disappointed — for example, a customer being disappointed with dirty rooms or cigarette smell.

    In contrast, we can observe emotional expectations, where customers have a vague idea of having a positive experience. However, they do not measure it against any standard. For example, having a pleasant customer service experience or being treated hospitably by the staff at a high-class hotel. Regardless of their knowledge beforehand of the service to be provided, positive emotions give them a perception of exceeded expectations and high satisfaction. This is where hospitality and service come into play and enhances the experience of the customers. 

    There are interesting differences between Chinese and English-speaking tourists in their change in satisfaction factors to differently priced hotels. For example, we can observe that the Chinese tourists have ``\begin{CJK}{UTF8}{gbsn}购物\end{CJK} (shopping)'' as a top keyword in all the price ranges. In contrast, English-speaking tourists only mention it as a top keyword in the 20,000 to 30,000 yen price range. It is common knowledge in Japan that Chinese tourists coming to Japan with the express intention of shopping are common. \cite{tsujimoto2017purchasing} analyzed the souvenir purchasing behavior of Chinese tourists in Japan. The study shows that common products besides food and drink are: electronics, cameras, cosmetics, medicine, among other more traditional \textit{souvenir} items, such as objects representative of the culture or places they visit \cite{japan2014consumption}. There is an understanding that tourists choose to shop in Japan has more to do with the quality of the items than their relation to the tourist attractions. Our results suggest that Western tourists are engaging more in tourist attractions in comparison with shopping activities. Another interesting difference is that English-speaking tourists start using negative keywords about the hotel's price only after it concerns hotels of 15,000 yen or more, and it rises in its ranking the more expensive the hotel is. In contrast, Chinese customers have this keyword as their top keyword across all price ranges. Previous research suggests that value for money is a key concern for Chinese and Asian tourists \cite[][]{choi2000,choi2001,truong2009}, while Western customers are more concerned with hospitality \cite[][]{kozak2002}.  

    While some aspects of satisfaction and dissatisfaction change depending on the hotel's price range, some other factors stay mostly constant for each culture's customers. For example, appreciation for staff from English-speaking tourists is ranked close to the top satisfaction factor in all the price ranges. Satisfaction for cleanliness by both cultures constantly stays part of the top 10 keywords, except for the most expensive one, where other keywords take that place in the ranking. However, it is still high on the list. Chinese tourists have a high ranking for the word ``\begin{CJK}{UTF8}{gbsn}早餐\end{CJK} (breakfast)'' across all price ranges as well. As discussed in section \ref{disc:location}, transport and location are also important for hotels of all classes and prices. While the ranking of attributes might differ between price ranges, hard and soft attribute proportions also appear to be constant within at most a 13\% margin of error per attribute, often being lower. This suggests that culturally the customers have a particular bias to consider some attributes more than others.

  \subsection{Implications for hotel managers}\label{disc:implications}

    Our study presents two important conclusions: one about hospitality and cultural differences, and another about managerial decisions towards two different populations. As a whole, Chinese tourists are not showing the most satisfaction towards hospitality and service factors. Instead, they focus on the hard attributes of a hotel. Either they do not get as much satisfaction from hospitality as Western tourists or feel that basic language and communication needs are not being met, so they receive a lesser impression. On the other hand, Western tourists are elated with Japanese hospitality, preferring soft attributes to hard-set ones. The other conclusion is that managerial decisions will mostly benefit Western tourists, except that Chinese language improvements and breakfast inclusion can satisfy more Chinese customers. Japan is recently seeing an increase in Chinese students as well as Western students of universities. Hiring students as part-time workers could increase the language services of a hotel.

    To satisfy both customer types, hotel managers need to invest in cleanliness, deodorizing, and making hotel rooms tobacco-free. It could also be recommended to invest in breakfast inclusion and multilingual services and staff preparedness to deal with Chinese and English speakers. Western tourists were also observed to have high comfort standards, which can be improved upon managerially for better reviews. Perhaps it could be suggested to perform surveys of the bedding that is most comfortable for Western tourists. However, not all hotels can invest in all of these factors simultaneously. Our results suggest that satisfying cleanliness needs can satisfy both customer types. A low-cost investment could be to make the facilities tobacco-free. Our results are also divided by price ranges, so a hotel manager can consider which analysis suits their hotel the most.

    While not manageable after a hotel has finished its construction, hard attributes are essential to consider for managers. As previously stated, transport systems and nearby shops are points of interest for both Chinese and Western tourists. Japanese hotel managers have to consider the location and surroundings before the hotel is constructed. A suggestion could be to purchase land and start the construction after public plans to make new subway lines are made. 

    It is left to the managers to consider their business model for the next strategy. One option could be attracting more Chinese customers in number with their observed low budgeting. Another could be attracting more high budget Western customers is on par with their business model. For example, investing more in cleanliness could improve Western customers looking for high-quality lodging satisfaction, even though the price per night would increase. On the other hand, hotels might be considered costly by Chinese customers wherever such an investment is made.

\section{Limitations and Future Work}\label{limitations}

  This paper is not without its limitations. We analyzed satisfaction and dissatisfaction keywords based on whether they appeared on satisfied reviews or dissatisfied ones. Following that, we attempted to understand the context in which these words were used by using a dependency parser and observing the related nouns. However, the study is limited in that it only analyzes the words directly related to each keyword and does not follow the upstream or downstream path down further connections. This means that if the words are used in combination with other keywords, we did not trace the effects of multiple contradicting statements. For example, in the sentence ``The room is good, but the food is lacking'', we would extract ``good food'' and ``lacking food'', but do not consider the fact that both occurred in the same sentence.

  This study analyzed the differences in customers' expectations at different levels of hospitality and service factors by dividing our data into price ranges. However, in the same price range, for example, the highest one, we can find both a western-style five-star resort and a high-end Japanese-style \textit{ryokan}. Services offered in these hotels are very high quality, although very different. However, most of our database is focused on the middle range priced hotels, which is comparably less varied in service. However, there is still a divide between western and Japanese style hotels.

  An essential aspect of this study is that we focus on the satisfaction and dissatisfaction towards expectations of individual aspects of the hotels. This gives us insight into which factors can focus on by hotel managers in applying this knowledge. However, our study was limited in that the overall satisfaction of each customer was not measured. This could be done by rating the volume of text used to describe satisfaction factors against the text volume used for dissatisfaction. However, this imposes a few difficulties that are out of the scope of this study. 

  Another limitation is that a large portion of the Asian tourists coming to Japan is Taiwanese and Korean. We could not analyze these populations because our team members do not know those languages. Aside from that, further typology analysis could not be made because of the nature of the data collected (for example, Chinese men and women of different ages or their Westerner counterparts).

  In future work, we plan to investigate further into this topic. We plan to extend our data to research for different trends for different regions of Japan and different kinds of hotels and between customers traveling alone or in groups, for fun or for work. Another point of interest in this study's future is to use word clusters with similar meanings instead of single words. 

\section{Conclusion}\label{conclusion}

  In this study, our objective was to analyze the differences in satisfaction and dissatisfaction between Chinese and English-speaking customers of Japanese hotels, particularly in the context of Japanese hospitality, \textit{omotenashi}. To answer our research questions \ref{rsq:hospitality} and \ref{rsq:hospitality_both}, we extracted keywords from their online reviews uploaded to the portal sites \textit{Ctrip} and \textit{TripAdvisor} using Shannon's entropy calculations. We used these keywords for sentiment classification via an SVC. We then used dependency parsing and part of speech tagging to extract commonly found pairs of adjectives and nouns, as well as single words. We divided this data by sentiment and hotel price range, considering the most expensive room for one night. 

  In the context of Japanese hospitality, we found that Western tourists had the most satisfaction with staff behavior, cleanliness, and other attributes relating to the hotel's services and hospitality. However, we found that Chinese customers had other concerns than hospitality when studying their satisfaction, more inclined to praise the room, location, or hotel's convenience. We found that both cultures have a different reaction to this hospitality environment. Both cultures have a different way of reacting to different prices. From this, we discussed two possible theories on why Chinese tourists respond differently to Westerners in this environment of \textit{omotenashi}. One theory is that while they are being treated well and react positively, the environment is not compatible with them because of language or culture barriers, which lessens their experience. The second possible theory is that they react differently to hospitality since they do not have the same expectations to be satisfied in the same way. We theorize that a lack of expectations could result in lessened satisfaction, even if the same service is presented. On the other hand, even when they hold high expectations in a highly-priced hotel, Western tourists show that Japanese hospitality exceeds their expectations, judging by their vocabulary for expressing their satisfaction. We consider that Western tourists are more reactive to hospitality and service factors than their Chinese counterparts.

  Lastly, we measured the satisfaction and dissatisfaction factors, referring to a hotel's hard and soft attributes. Soft attributes can be changed via management and staff by an improvement in services. On the other hand, hard attributes are physical and impractical elements to change, such as the size of a room that has already been constructed, the location of a hotel, closeness to convenient spots, or elements out of the control of the hotel managers. We found that for satisfaction, Western tourists favor soft attributes. In contrast, Chinese tourists are more interested in the hard attributes of hotels, consistently across price ranges. For dissatisfaction, Western tourists are also highly inclined to criticize soft attributes, such as cleanliness or cigarette smell in rooms. In contrast, Chinese tourists' dissatisfaction comes evenly from both hard and soft attributes.

  One possible approach for hotel managers is to improve the satisfaction levels of Chinese tourists, who dedicate less percentage of their budget to hotels but are more abundant in number. They are less satisfied with soft attributes but have an identifiable method of improving satisfaction by lessening language barriers and providing a satisfactory breakfast. Another approach we discussed was focusing on the cleanliness and comfort that Western tourists expect and making the hotels tobacco-free. We favor ``long-haul'' Western tourists who spend almost half of their budget on hotels with this strategy. While Westerners are less in number than Chinese tourists, it could prove to have more substantial returns. This is because Chinese customers also favor cleanliness as a satisfaction factor, and both populations could be pleased. This paper provides results and discussion that can be utilized as a guideline for managerial decisions when considering Chinese and Western tourists in Japan. We can observe their stark differences and shared attributes. 

\begin{acknowledgements}

  During our research, we received the commentary and discussion by our dear colleagues necessary to understand particular cultural aspects that could influence the data's interpretation. We would like to show gratitude to Mr. Liangyuan Zhou, Ms. Min Fan, and Ms. Eerdengqiqige for this.

  We would also like to show gratitude to Ms. Aleksandra Jajus, from whom we also received notes on the editing and commentary on the content of our manuscript.

  Funding: This work was supported by the Japan Construction Information Center Foundation (JACIC).

  Conflict of interest: none
