\subsection{Japanese hospitality and service: \textit{Omotenashi}}\label{theory_omotenashi}

    The spirit of Japanese hospitality, or \textit{Omotenashi}, has roots in the country’s history, and to this day, it is regarded as the highest standard \cite[][]{ikeda2013omotenashi, al2015characteristics}. There is a famous phrase in customer service in Japan: \textit{okyaku-sama wa kami-sama desu}, meaning ``The customer is god.'' Some scholars say that \textit{omotenashi} originated from the old Japanese art of the tea ceremony in the 16th century, while others found that it originates in the form of formal banquets in the 7th-century \cite[][]{aishima2015origin}. The practice of high standards in hospitality has survived throughout the years. Presently, it permeates all business practices in Japan, from the cheapest convenience stores to the most expensive ones. Manners, service, and respect towards the customer are taught to workers in their training. High standards are always followed to not fall behind in the competition. In Japanese businesses, including hotels, staff members are trained to speak in \textit{sonkeigo}, or ``respectful language,'' one of the most formal of the Japanese formality syntaxes. They are also trained to bow differently depending on the situation, where a light bow could be used to say ``Please, allow me to guide you.'' Deep bows are used to apologize for any inconvenience the customer could have faced, followed by a very respectful apology. Although the word \textit{omotenashi} can be translated directly as ``hospitality,'' it includes both the concepts of hospitality and service \cite[][]{Kuboyama2020}. This hospitality culture permeates every type of business with customer interaction in Japan. A simple convenience shop could express all of these hospitality and service standards, which are not exclusive to hotels.  

    It stands to reason that this cultural aspect of hospitality would positively influence customer satisfaction. However, in many cases, other factors such as proximity to a convenience store, transport availability, or room quality might be more critical to a customer.  In this study, we cannot directly determine whether a hotel is practicing the cultural standards of \textit{omotenashi}. Instead, we consider it as a cultural factor that influences all businesses in Japan. We then observe the customers' evaluations regarding service and hospitality factors and compare them to other places and business practices in the world. In summary, we consider the influence of the cultural aspect of \textit{omotenashi} while analyzing the evaluations on service and hospitality factors that are universal to all hotels in any country.

    Therefore, we pose the following research question:

    \begin{subrsq}
    \begin{rsq}
    \label{rsq:hospitality}
    To what degree are Chinese and Western tourists satisfied with Japanese hospitality factors such as staff behavior or service?
    \end{rsq}

    However, Japanese hospitality is based on the Japanese culture. Different cultures interacting with it could provide a different evaluation of it. Some might be impressed by it, whereas some might consider other factors more important to their stay in a hotel. This point leads us to a derivative of the aforementioned research question:

    \begin{rsq}
    \label{rsq:hospitality_both}
    Do Western and Chinese tourists have a different evaluation of Japanese hospitality factors such as staff behavior or service?
    \end{rsq}
    \end{subrsq}