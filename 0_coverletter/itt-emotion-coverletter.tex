\documentclass{letter}
\makeatletter
  \def\@texttop{}
\makeatother

\signature{Elisa Claire Alemán Carreón}
\address{
\textbf{Corresponding Author:}\\
Elisa Claire Alemán Carreón \\
Nagaoka University of Technology, Nagaoka, Japan \\
P.C. 940-2033, Ribbon Nagaoka B104, \\
1128-3 Kaminozoki-machi, Nagaoka, Niigata, Japan \\
E-mail: elisa.claire.aleman.carreon@gmail.com
}

\addtolength{\headheight}{-11em}
\addtolength{\textheight}{10em}
\addtolength{\textwidth}{8em}
\addtolength{\oddsidemargin}{-6em}

\begin{document}

\begin{letter}{
Information Technology \& Tourism Editorial Board \\
Editor-in-Chief:  Zheng Xiang\\
Virginia Tech, USA}

\opening{Dear Zheng Xiang and ITT Editorial Board:}

We wish to submit a revision for an original research paper JITT-D-19-00095, which is currently under the title "Differences in Chinese and Western tourists faced with Japanese hospitality: A natural language processing approach" for further review by Information Technology \& Tourism.

We appreciate the consideration by your journal, and took the comment in your decision to add to the theoretical reasoning behind the research problem. Therefore, in this revision, we made a few changes to the structure of section "3. Theoretical background and hypothesis development", adding more literature we reviewed, and added to the Abstract, Introduction, and Research objective sections as well. 

In the abstract, we added a concise background of why we are interested in a cross-culture study using a large data-base. In the introduction, we add citations to clarify the influence of culture on expectations, perceptions of quality and satisfaction, and the importance to analyze the difference between Chinese and Western tourists specifically in Japan (l.17-l.27, l.37-l.41). We also added on the necessity to do this analysis on a large scale with automatic methodologies (l.60-l.67). We also added to the research objective section to clarify the reasoning behind the objective (l.92-l.95). 

In the Theoretical background section, we moved the section "3.1 Japanese hospitality and service: \textit{Omotenashi}" to 3.3, and talked about the theory behind satisfaction, expectations, and their relationship to culture in the new sections 3.1 and 3.2, which add content and citations to what previously was section "3.2 Customer satisfaction and dissatisfaction towards individual factors during hotel stay". It was split into two sections to avoid going off topic, since one of them explains the theory of expectations leading to satisfaction, while the other one explains the theory behind our approach to studying the customers' satisfaction. Lastly, we added a section "3.4 Customer expectations beyond service and hospitality" to explain the reasoning behind studying hard and soft attributes better.

We believe this was an appropriate response to the major revision, and hope we continue to be in your consideration.

\closing{Sincerely,}


\end{letter}

\end{document}