\documentclass{letter}
\makeatletter
  \def\@texttop{}
\makeatother

\signature{Elisa Claire Alemán Carreón}
\address{
\textbf{Corresponding Author:}\\
Elisa Claire Alemán Carreón \\
Nagaoka University of Technology, Nagaoka, Japan \\
P.C. 940-2033, Ribbon Nagaoka B104, \\
1128-3 Kaminozoki-machi, Nagaoka, Niigata, Japan \\
E-mail: elisa.claire.aleman.carreon@gmail.com
}

\addtolength{\headheight}{-11em}
\addtolength{\textheight}{10em}
\addtolength{\textwidth}{8em}
\addtolength{\oddsidemargin}{-6em}

\begin{document}

\begin{letter}{
Information Technology \& Tourism Editorial Board \\
Editor-in-Chief:  Zheng Xiang\\
Virginia Tech, USA}

\opening{Dear Zheng Xiang and ITT Editorial Board:}

We wish to submit an original research paper titled "Analyzing differences in satisfaction and dissatisfaction between Chinese and English-speaking customers of Japanese hotels with machine learning" for consideration by Information Technology \& Tourism.

We confirm that this work is original and has not been published elsewhere, nor is it currently under consideration for publication elsewhere. 

In this paper, we report on our findings on the quantitative analysis of differences in hotel preferences between Chinese and English-speaking tourists in Japan using their online reviews. We found that Chinese customers are satisfied with big and clean spaces, but are unsatisfied with the level of Chinese friendliness, in that Chinese translations might be lacking. On the other hand Western customers are satisfied mostly with staff friendliness, but are unsatisfied with dirty rooms and the smell of cigarette. Both customer groups value the location of the hotel as a secondary factor for satisfaction, and both have concerns about the pricing of the hotels (value for money). We also tested a hypothesis that their satisfaction and dissatisfaction would stem from both managerial and environmental attributes of the hotel, and found that the most important factors are all managerial in nature, and as such, could be improved in the future.

We believe that this manuscript is appropriate for publication by Information Technology \& Tourism because we used text mining and machine learning techniques, as well as a similarity measure RBO used in the information field, to quantitatively measure the satisfaction of customer groups that are important for the tourism industry in Japan.

We have no conflicts of interest to disclose. Please address all correspondence concerning this manuscript to me at the e-mail provided above.

\closing{Sincerely,}


\end{letter}

\end{document}