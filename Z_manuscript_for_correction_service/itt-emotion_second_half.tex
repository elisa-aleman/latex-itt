\begin{document}

\section{Results}\label{results}

  \subsection{Experiment results and answering research questions}

    Our research questions were about two things. In research questions \ref{rsq:hospitality} and \ref{rsq:hospitality_both}, we decide that the objective of this study to determine how Chinese and Western tourists interact with the \textit{omotenashi} culture influenced hospitality and service in Japan, and how are they different in their perceptions in this matter. We observed the top-ranking positive factors for Chinese tourists across different price ranges in Table \ref{tab:freq_res_pos}, and specifically the word ``\begin{CJK}{UTF8}{gbsn}不错\end{CJK} (not bad)'' and its pairings in Table \ref{tab:adj_zh_pos}. From these observations, we can infer that, while service, cleanliness, and breakfast are praised in most hotels, location is usually placed above it in importance on the pairings. When we see the rest of the factors lower on the list, we see that the list is more populated with hard attributes like location and transportation availability across different price ranges. From the negative keyword usages in Table \ref{tab:freq_res_neg}, there are complaints about the lack of a Chinese friendly environment. However, most complaints are also about hard attributes such as the building's age and the distance from other convenient spots. Nevertheless, the most complained about aspect is the price of the hotel. Surprisingly, all of the price ranges have this negative keyword at the top of the list, suggesting that it is the main concern to Chinese customers with different travel purposes.

    On the other hand, the word ``staff'' is the second or third in the lists of satisfaction factors in English-written reviews in all the price ranges. This word is followed by a few other keywords lower in the top 10 list, such as ``helpful'' or ``friendly''. When we look at the pairings of the top-ranked keyword ``good'' in Table \ref{tab:adj_en_pos}, we find that customers mostly praise the location, service, breakfast, or English availability. When we look at the negative keyword ``poor'' and its pairings in Table \ref{tab:adj_en_neg}, we see that it is also service-related concepts that the Western tourists are disappointed with. With these results, we can observe that both Chinese and English-speaking tourists in Japan have different priorities. However, both populations consider the hotel's location and transport availability (subways and trains) nearby as secondary but still essential points in their satisfaction with a hotel. The Chinese customers are primarily satisfied with the room quality in spaciousness and cleanliness and the service of breakfast.

    In contrast, the English-speaking customers are easily upset by any lack of cleanliness and smoke smell from cigarettes. Surprisingly, the cigarette smell is an issue even in the middle to high-class hotels above 30,000 yen per night. However, above 50,000 yen per night, this problem seems to disappear from the list of top 10 concerns. Old and dated buildings seem to be a concern for both populations. On the positive side, for all price ranges considered, English-speaking tourists value staff friendliness over room quality when considering their satisfaction. In contrast, Chinese tourists consider location and transportation more often.

    We also can observe some keywords that are not considered by their counterparts. For example, English-speaking customers mentioned tobacco smell in many reviews. However, it was not statistically identified as a problem for their Chinese counterparts. On the other hand, while it appears in both English and Chinese lists, references to ``\begin{CJK}{UTF8}{gbsn}购物\end{CJK} (shopping)'' are more common in the Chinese lists across hotels of 15,000 yen to 200,000 yen per night. Meanwhile, the term ``shopping'' only appears in the 20,000 to 30,000 yen per night top 10 positive keywords list for English-speakers.

    In our research questions \ref{rsq:hard_soft} and \ref{rsq:hard_soft_diff}, we ponder how customers of both cultural backgrounds evaluate hard and soft attributes of the hotel and how they differ in those evaluations. Here we define hard attributes as those relating to the hotel's physical, structural, or environmental aspects. These are often impossible or impractical to change by the hotel management and staff, such as facilities, infrastructure, the surroundings, view, or location convenience. On the other hand, hotel staff and management can change soft attributes, for example, by improving the hotel's services via training or hiring specialized staff, improving the quantity or quality of amenities, bedsheets, or general cleanliness. Our study found that Chinese tourists are mostly positively reacting more to the hotel's hard attributes. There is a slightly hard leaning (53\%) concern with hard attributes in negative sentences, albeit this is more uniform than the positive evaluations. English-speaking tourists, on the other hand, are both positively and negatively more responsive to soft attributes. In the case of negative keywords, English-speaking tourists are overwhelmingly more concerned with the hotel's soft attributes dissatisfaction somehow.

    One factor that both populations have in common is, when perceiving the hotel negatively, ``\begin{CJK}{UTF8}{gbsn}老\end{CJK} (old)'', ``dated'', ``outdated'', or ``\begin{CJK}{UTF8}{gbsn}陈旧\end{CJK} (obsolete)'' aspects of the room or the hotel are being criticized across, surprisingly, most price ranges. This is, however, a hard attribute and is unlikely to change for most hotels.
    
  \subsection{Chinese tourists - A big and clean space}\label{disc:zh}

    We found that mainland Chinese tourists are satisfied mostly by Japanese hotels' big and clean spaces. From the adjectival pairings that we extracted with dependency parsing and POS tagging in Table \ref{tab:adj_zh_pos}, we can observe that mostly they mean big and clean rooms. Other mentions are also big markets nearby or a big bed. We can observe that across different price ranges, the usage of the word ``\begin{CJK}{UTF8}{gbsn}大\end{CJK} (big)'' increases as the hotel increases in price. However, we can see that they still react positively in a significant manner in cheaper hotels. Inspecting closer by taking random samples of the pairs of ``\begin{CJK}{UTF8}{gbsn}大 空间\end{CJK} (big space)'' or ``\begin{CJK}{UTF8}{gbsn}大 面积\end{CJK} (large area)'', we can see that there are also many references to the public bathing facilities in the hotel. We can also see them mentioned as a word pairing ``\begin{CJK}{UTF8}{gbsn}棒 温泉\end{CJK} (great hot spring)''. In Japan, there is what is called ``\begin{CJK}{UTF8}{min}銭湯\end{CJK} (\textit{sent\=o})'', which are artificially made public bathing facilities, on occasions including saunas and baths with unique qualities. On the other hand, there are natural hot springs, called ``\begin{CJK}{UTF8}{gbsn}温泉\end{CJK} (\textit{onsen})''. They can either be bathing in the natural source of the water or using the hot springs in artificially made bath facilities. It is a Japanese custom and culture that all customers use the facilities after cleaning themselves in a shower and go into the baths without any clothes. It can be a cultural shock for many tourists, but this is a fundamental attraction for many. 

    However, the size of the room or the bed is a hard attribute. Without considering rebuilding the hotel, it is not trivial to improve on. On the other hand, cleanliness is mostly relating to soft attributes when we observe its adjectival pairings. We can observe pairs such as ``\begin{CJK}{UTF8}{gbsn}干净 房间\end{CJK} (clean room)'' at the top rank of all price ranges, and then variably ``\begin{CJK}{UTF8}{gbsn}干净 酒店\end{CJK} (clean hotel)'', ``\begin{CJK}{UTF8}{gbsn}干净 总体\end{CJK} (clean overall)'', ``\begin{CJK}{UTF8}{gbsn}干净 环境\end{CJK} (clean environment)'', and ``\begin{CJK}{UTF8}{gbsn}干净 设施\end{CJK} (clean facilities)'', among other examples. In negative reviews, there is a mention of criticizing the ``\begin{CJK}{UTF8}{gbsn}一般 卫生\end{CJK} (general hygiene)'' of the hotel, although it is an uncommon pair. Therefore, we can assert that cleanliness is an important soft attribute for Chinese customers and that they are mostly pleased with their expectations being met. 

    One key component we found in Chinese customer satisfaction soft factors is the inclusion of breakfast within the hotel. While other food-related words were extracted, most of them were general, like ``food'' or ``eating,'' which were lower ranking. In contrast, the word ``\begin{CJK}{UTF8}{gbsn}早餐\end{CJK} (breakfast)'' refers possibly to its inclusion in the hotel commodities, was frequently used in positive texts compared to other food-related words. The word ``\begin{CJK}{UTF8}{gbsn}早餐\end{CJK} (breakfast)'' is also observed across all price ranges, although at different priorities in each of them. However, we assert that it is an important factor. Observing word pairs from the positive Chinese keywords in Table \ref{tab:adj_zh_pos}, we can also see that ``\begin{CJK}{UTF8}{gbsn}不错\end{CJK} (not bad)'' is paired as ``\begin{CJK}{UTF8}{gbsn}不错 早餐\end{CJK} (nice breakfast)'' in four of the seven price ranges with reviews available as part of the top 4 pairings. It is only slightly lower on other categories, although it is not shown on the table. Thus we consider that a recommended strategy for hotel management is to invest in the inclusion or betterment of hotel breakfast to increase good reviews.

  \subsection{Western tourists - A friendly face, and absolutely clean}\label{disc:en}

    From the satisfaction factors of English-speaking tourists, we can see that at least three words relate directly to staff friendliness and services, being ``staff'', ``helpful'' and ``friendliness'' in the general database. The word ``staff'' is the second most frequently used word for satisfied customers across most price ranges, and only third in one of them. Adding to that, ``helpful'' and ``friendly'' follow it lower in the list in most price ranges. The word ``good'' is mostly about the location, the service, breakfast, or English availability in Table \ref{tab:adj_en_pos}. Like Chinese customers, Western customers also seem to enjoy the included breakfasts regarding their satisfaction keyword pairings. However, the word does not appear directly in the top 10 list as in their Chinese counterparts. The words ``helpful'' and ``friendly'' are mostly paired with ``staff'', ``concierge'', ``desk'', and ``service''. When we look at the negative keyword ‘poor’ and its pairings in Table \ref{tab:adj_en_neg}, we see that it is also service-related concepts that the Western tourists are disappointed with when they react negatively.

    Another soft attribute that is high on the list for most of the price ranges is the word ``clean''. Since it is an adjective, we have explored the word pairings as well. Customers are mostly praising ``clean rooms'' and ``clean bathrooms'', while also referring to the hotel in general. It seems that when observing the negative keyword frequencies for English-speakers, we can find words such as ``dirty'', ``carpet'', and from the word pairings ``dirty carpet'', ``dirty room'', and ``dirty bathroom''. Along with complaints about off-putting smells, we can conclude that Western tourists have high expectations about cleanliness when traveling in Japan.

    An interesting detail of the keyword ranking is that the word ``comfortable'' is high on the satisfaction factors and ``uncomfortable'' high on the dissatisfaction factors. The words are paired with nouns like ``bed'', or ``room'', ``pillow'' or ``mattress'', generally referring to their sleep conditions in the hotel.
    It seems that Western tourists are highly sensitive to comfort levels in the hotels and whether it reaches their expectations. The ranking for the negative keyword ``uncomfortable'' is similar across most price ranges, except the two most expensive ones, where this keyword disappears from the top 10 list.

    While less high in priority, the price range of 15,000 to 20,000 yen hotels also mentions ``free'' as one of the top 10 positive keywords, paired mostly with ``wifi''. This price range is mostly for business hotels, where we infer users would be expecting this feature the most. Western tourists are highly sensitive to comfort levels in the hotels and whether it reaches their expectations.


  \subsection{Tobacco, what's that smell?}\label{disc:tobacco}

    A concern for Western tourists was the smell of tobacco in their room, which can be considered a soft attribute. Tobacco was found not only as a standalone word ``cigarette'', but also as word pairs in Table \ref{tab:adj_en_neg}. We can find other related word pairs such as ``funny smell''. Upon manual inspection of a sample of reviews with this keyword, we found that the room was often advertised as non-smoking, yet, the smell permeated the room and curtains. Another common complaint was that there were no non-smoking facilities available at all in the first place. The smell of smoke can completely ruin some customers’ stay and give a bad impression to review writers, lowering the number of future customers.

    However, in comparison, Chinese customers seem not to be bothered by this at all. We consulted studies involving the use of tobacco in different countries. Previous research states that 49 - 60\% of Chinese men (and 2.0 - 2.8\% of women) currently smoke or have smoked before. This was taken from a sample of 170,000 Chinese adults in 2013-2014, which is high compared to many English-speaking countries \cite[][]{zhang2019tobacco,who2015tobacco}.

    Japan has a polarized view on the topic of smoking. Despite being one of the world’s largest tobacco markets, its use has decreased in recent years. Smoking in public spaces is prohibited in some wards of Tokyo (namely Chiyoda, Shinjuku, and Shibuya). However, it is generally only urged and not mandatory to have smoking restrictions in restaurants, bars, hotels, and public areas. However, many places have designated smoking rooms are available to keep the smoke in an enclosed area and avoid bothering others. Despite this, businesses, especially those who cater to certain customers, will generally be discouraged from having smoking restrictions if they want to keep their clientele. If Japanese hotels want to cater to all kinds of customers, Western and Asian alike, they must provide spaces without tobacco smell. After all, even if it does not bother a few customers, the lack of smell would make it an appropriate space for all customers.


  \subsection{Location, location, location}\label{disc:location}

    The hotel's location, closeness to the subway and public transport, and nearby shops' availability were observed to be of importance to both Chinese and English-speaking tourists. In positive word pairings Tables \ref{tab:adj_zh_pos} and \ref{tab:adj_en_pos}, we can find pairs such as ``\begin{CJK}{UTF8}{gbsn}不错 位置\end{CJK} (nice location)'', ``\begin{CJK}{UTF8}{gbsn}近 地铁站\end{CJK} (near subway station)'', ``\begin{CJK}{UTF8}{gbsn}近 地铁\end{CJK} (near subway)'' in Chinese texts and ``good location'', ``great location'', and ``great view'', as well as single keywords ``location'' and ``shopping'' for English-speakers, and ``\begin{CJK}{UTF8}{gbsn}交通\end{CJK} (traffic)'', ``\begin{CJK}{UTF8}{gbsn}购物\end{CJK} (shopping)'', ``\begin{CJK}{UTF8}{gbsn}地铁\end{CJK} (subway)'', and ``\begin{CJK}{UTF8}{gbsn}环境\end{CJK} (environment or surroundings)'' for Chinese speakers. All of these keywords and their location in each population's priorities across the price ranges signal that while it was not the priority for either of them, the hotel's location is a secondary but still important point in the hotel's satisfaction. However, since this is a hard attribute, unchangeable to the hotel's management, it is not often considered in the literature. Upon inspection of examples from the data, we found that most customers were satisfied if the hotel was near to at least two other subjects: subway, train, and convenience stores. 

    Japan is a country with a peculiar public transport system. The rush hour makes for a subway filled to the brim with people in suits making their commute, and trains and subway stations in Tokyo create a confusing public transport map for a visitor. Buses are also available, although less used than the rail systems in the big cities. These three are unusually affordable in price. Then there are the more expensive transports, such as the bullet train \textit{shinkansen} for traveling across the country, and taxis. Taxis in Japan are a luxury compared to other countries. In less developed countries, a taxi is the cheap method of transport of choice. In Japan, taxis are made to provide a high-quality experience with a matching price. This means that for tourists, subway availability and maps or GPS applications and a plan to travel the city are of utmost necessity. 

    Japanese convenience stores, on the other hand, are also famous worldwide. Japanese convenience stores are a haven for the traveler in need. It offers anything, from drinks and snacks to full meals, copy and scanning machines, alcohol, cleaning supplies, personal hygiene items, underwear, towels, international ATMs, among other things. If some trouble occurred, or a traveler forgot to pack a particular item, it is almost sure that they can find it. 

    Therefore, considering that both transport systems and nearby shops are points of interest for Chinese and Western tourists, Japanese hotels have to carefully choose their location from the moment they are constructed. While not a top priority, this is a universal factor for both customer groups, and it can be an instant way to generate positive reviews.

\section{Discussion}\label{discussion}

  Below we explore the possible interactions with Japanese hospitality, the differences between Chinese and Western tourists, the possible cause for them, how they vary across different price ranges, and what they imply for the industry. We also discuss the differences between the hotel's hard and soft attributes and how they contribute to customers' satisfaction.

  \subsection{Western and Chinese tourists in the Japanese hospitality environment}\label{disc:omotenashi}

    To this day, scholars continue to correct our historical bias towards the west. Studies have determined that different cultural backgrounds lead to different expectations, which influences tourists' satisfaction. Meaning, tourists of a particular culture will have different leading satisfaction factors across different destinations. However, Japan presents a particular environment. The spirit of hospitality and service, \textit{omotenashi}, excels and is considered the highest standard across the world. Can such an environment affect different cultures equally? Or is it only attractive to certain cultures? Our study brings light to these questions.

    Our results indicate that out of the two, Western tourists are the most satisfied with soft attributes, such as friendly and helpful staff in Japan. As explained earlier in this paper, Japan is famous for its customer service all over the world. Respectful language and bowing are not exclusive to high priced hotels or businesses. These can even be found in convenience stores. The level of hospitality in even the cheapest of convenience stores is starkly different from Westerner experiences. While it could be a culture shock to some, it is mostly seen positively. After all, the Japanese staff respectfully treats all customers. However, for some customers, this could be the best way they have been treated until that moment. Now, in higher priced hotels, the adjectives used to praise the service go from normal descriptors like ``good'' to higher levels of praise like ``wonderful staff'', ``wonderful experience'', ``excellent service'', and ``excellent staff''. We can also see that \cite{kozak2002} and \cite{shanka2004} have also found that hospitality and staff friendliness is a vital determinant in Western tourists' satisfaction.

    However, we can see from the negative English keywords that a big part of the dissatisfaction with Japanese hotels stems from a lack of hygiene and room cleanliness. Although Chinese customers only had positive keywords about cleanliness, English-speaking customers have found many places unacceptable to their standards. This is particularly true at hotels below 50,000 yen per night. The most common complaint regarding cleanliness was about the carpet, followed by complaints about cigarette stench and general dirtiness. \cite{kozak2002} also found that hygiene and cleanliness were essential satisfaction determinants for Western tourists. However, in the previous literature, this was linked merely to satisfaction. In comparison, our research uncovered that words relating to cleanliness are mostly linked to dissatisfaction. Westerners could be said to have a high standard of room cleanliness when compared to their Chinese counterparts.

    According to previous research, we can see that Western tourists are already inclined to appreciate hospitality for their satisfaction. When presented with Japanese hospitality, this expectation is met and overcome. In contrast, we can see from our results that Chinese tourists had less focus on hospitality, staff, or service and were more concerned with room quality. However, when analyzing the word pairs for ``\begin{CJK}{UTF8}{gbsn}不错\end{CJK} (not bad)'' and for ``\begin{CJK}{UTF8}{gbsn}棒\end{CJK} (great)'', we can see that they do praise staff, service and breakfast. Observing the percentage of hard to soft attributes in Figure \ref{fig:hard_soft_zh}, however, we know that Chinese customers are satisfied more with hard attributes, compared to the Western tourists who seem to be meeting more than their expectations.

    It could be that Chinese culture does not expect high-level service initially. When an expectation that is not held is met, the satisfaction that stems from this is less than if it was expected. On the other hand, we have the phenomenon of a ``nice surprise'': When an unknown need is unexpectedly met, there is more satisfaction. It is necessary to note the difference between these two phenomenons. The ``nice surprise'' fulfills a need unexpectedly. Perhaps the hospitality grade in Japan does not fulfill a high enough need for the Chinese population, resulting in less satisfaction. For greater satisfaction, the existence of a need being met is necessary. However, the word ``not bad'' is at the top of the list at most price ranges, and one of the uses is related to service. Thus, we cannot say that they are not satisfied with this matter. Rather, they hold other factors at a higher priority, considering the keyword frequency is higher for other pairings.

    Another possibility presents itself when we observe the Chinese tourists’ dissatisfaction factors. Chinese tourists may have expectations about the Chinese visitors' treatment that are not being met, even in this high standard hospitality environment. Japan is known worldwide for their hospitality, but they are also known historically to be monolingual and have a relatively large language barrier \cite[][]{heinrich2012making,coulmas2002japan}. While the Japanese effort to accommodate English speakers is slowly taking shape, Chinese accommodations can be lagging. Chinese language pamphlets, Chinese texts on instructions for the hotel room, and its appliances and features (e.g., T.V. channels, Wi-Fi setup), or just the treatment towards Chinese people could be examples. It is natural to be dissatisfied since traveling in a strange land without knowing the language can be a daunting experience. \cite{ryan2001} also found that communication difficulty was one of the main reasons Chinese customers would state for not visiting again. It seems like this is a problem that is not singular to Japan.

    Our initial question was whether the environment of high-grade hospitality would affect both cultures equally. This study brought us closer to the answer. On the one hand, there is a possibility that Chinese customers did have high-grade hospitality and did not get equally satisfied with Westerners. In that case, it appears that the difference stems from a psychological source. Expectation leads to satisfaction and a lack of expectation results in lesser satisfaction. On the other hand, there is also a possibility that Chinese customers are not receiving the highest grade of hospitality because of cultural friction between Japan and China.

    It is unclear from our results which of these could be the case. One thing is clear for hotel managers, however. Competing in hospitality and service does include language services, especially in the international tourism industry. Better multilingual support can only improve that already high standard in Japan. Considering that most of the tourists in Japan come from other countries in Asia, this is an endeavor that truly can bring benefits to their investment. Proposals for this endeavor include hiring Chinese speaking staff, preparing pamphlets in Chinese, or have a translator application readily available with staff trained in interacting through an electronic translator.

  \subsection{Hard vs. soft satisfaction factors}\label{disc:hard_soft}

    As we stated in section \ref{theory_satisfaction}, previous research is focused mostly on the hotel's soft attributes and their influence on customer satisfaction. Examples of soft attributes include staff behavior, commodities, amenities, and appliances that can be improved within the hotel \cite[e.g.][]{shanka2004,choi2001}. However, hard attributes are not usually analyzed in satisfaction studies. Examples of hard attributes include the hotel's location relative to public transport and shops, language immersion of the country, noise pollution, or weather. Because our study left the satisfaction factors to be decided statistically via customers’ online reviews, we can see the importance of those hard or soft attributes in their priorities.

    Figure \ref{fig:hard_soft_zh} shows that in regards to Chinese customer satisfaction, in general, 68\% of the top 10 keywords are hard factors. In contrast, only 20\% are soft factors. The rates are similar for most price ranges, excepting the highest-priced hotels, where 35\% of the keywords are undefined. However, the soft attributes are still similar at 18\%. However, two of these managerial words are all concentrated at the top of the list (``\begin{CJK}{UTF8}{gbsn}不错\end{CJK} (not bad)'', ``\begin{CJK}{UTF8}{gbsn}干净\end{CJK} (clean)''), plus the adjective pairs relating to soft attributes of ``\begin{CJK}{UTF8}{gbsn}不错\end{CJK} (not bad)'' which are at the top in most price ranges as well. Chinese tourists could expect spaciousness and cleanliness when coming to Japan. That expectation could be caused by reputation, previous experiences, or cultural backgrounds. Some scholars argue that different cultures have different room size perceptions \cite[][]{Saulton2017}. Although the study subjects are German and South Korean, the study presents the results as differences influenced by Asian and Western cultures. We argue that one country is not representative of others’ cultures, so there can be differences between South Korea and China in room size perception. However, an interesting point appears. It could be that a different room size perception affects the satisfaction of Chinese tourists in contrast with Westerners. Westerners only start placing a priority on praising room size as the price of the hotel goes up. We can compare these results with previous literature, where traveling Chinese tourists choose their destination based on several factors, including cleanliness, nature, architecture, and scenery \cite[][]{ryan2001}. These other few factors found in previous literature could be linked to the keyword ``\begin{CJK}{UTF8}{gbsn}环境\end{CJK} (environment or surroundings)'' as well. This keyword is present in hotels priced at more than 20,000 yen per night. 

    In comparison, English speakers are mostly satisfied with the hotel's soft attributes. Figure \ref{fig:hard_soft_en} shows that soft attributes are above 48\% in all price ranges, the highest being 65\% in the 15,000 to 20,000 yen per night price range. This price range corresponds to affordable business hotels, for example. English-speaking customers also have soft attributes at the top of their list. The exception is the hard attribute that is the hotel's location, which is consistently around the middle of the top 10 lists for all price ranges. If one considers both Chinese and Western tourists’ satisfaction, a hotel can improve to attract more customers in the future. If it was the other way around, and the satisfaction was related more with hard attributes overall for 1020 both cultures, hotels would have to compete solely on their location.

    For both customer groups, the main reason for dissatisfaction is pricing, which can be interpreted as a concern about value for money. However, it is interesting to note that while English-speaking customers complain about price with a lower rank in lower-priced hotels. In contrast, the Chinese customers consistently have ``\begin{CJK}{UTF8}{gbsn}价格\end{CJK} (price)'' as a top or second-most concern across all price ranges. A paper studying Chinese tourists found that they had this concern \cite[][]{truong2009}. However, our results indicate that this is less of a cultural attribute in Japanese hotels and has more to do with the pricing of hotels overall. The tourists coming to Japan could be both experienced travelers or first-time travelers. However, the fact is that their expectation of the price for hotels was lower than what they found in Japan. In general, Japan is an expensive place to visit, impacting this placement in the ranking. Space is scarce in Japan, and capsule hotels with cramped spaces of 2 x 1 meters cost around 3,000 to 6,000 yen per night. Bigger business hotel rooms are relatively expensive, ranging from 5,000 to 12,000 yen per night. For comparison, hotels in the USA with a similar quality can be half the price.

    Around half of the dissatisfaction factors for both Chinese and Western customers are caused by issues that could be solved with improved management. The previous is true for all price ranges. Of course, the improvements could be staff training (perhaps in language), hiring professional cleaning services for rooms with cigarette smoke smells, or improving the bedding. All of these options can be costly. However, this paper provides a good guideline for which factors to consider first and which ones will be best suited to each customer group. Hotels can use the price range categorization in order to choose the appropriate strategy as well. However, once the hotel's location and construction are set for Chinese customers, not much else can be done to satisfy them further. As mentioned before, Chinese language availability is another soft attribute that can be improved with staff and training investment.

    On the other hand, Western tourists are all around dissatisfied with mostly soft attributes. They show this by having a low of 35\% in the highest price range where undefined factors are the majority and 78\% at most in the price range from 30,000 to 50,000 yen per night in a hotel. The room for improvement for Western tourists is more extensive than their Chinese counterparts. As such, it presents a bigger investment opportunity. As mentioned earlier in this paper, Westerners are known as ``long-haul'' customers, spending more than 45\% of their budget on hotel lodging.
    On the other hand, their Asian counterparts only spend 25\% of their budget on hotels \cite[][]{choi2000}. With bigger returns on managerial improvements, it seems like we can recommend investing in improving attributes that dissatisfy Western customers, such as cleanliness and removing tobacco smell. Making more hotel facilities tobacco-free and deodorizing the rooms can be a low-cost investment that could increase returns many times over.

    However, the opposite argument could also be made that Chinese customers provide a more significant number of customers, even though they tend to spend less on lodging. Attracting a large number of Chinese customers can be a viable strategy for hotels. However, as mentioned before, they tend to focus more on hard attributes, leaving language barrier-breaking as one of the few strategies to accomplish this.

    The basic premise of this study is that different cultures lead to different expectations and satisfaction factors. This premise also plays a role in the differentiation between the preference of hard or soft attributes.

    In \cite{donthu1998cultural}, subjects from 10 different countries were compared in their expectations of service quality and analyzed through the lens of Hofstede's typology of culture \cite[][]{hofstede1984culture}. That study states that although culture has no one specific index, five dimensions of culture can be used to analyze or categorize a country in comparison to others. These are \textit{power distance}, \textit{uncertainty avoidance}, \textit{individualism-collectivism}, \textit{masculinity-femininity}, and \textit{long-term versus short-term orientation}. In each of these dimensions, at least one element of service expectations was found to be significantly different for countries grouped under contrasting attributes (e.g., individualistic countries vs. collectivist countries, high uncertainty avoidance countries vs. low uncertainty avoidance countries). However, Hofstede's typology has received criticism from academics, particularly the fifth dimension that Hofstede proposed, which was added afterward with the alternate name \textit{Confucian dynamic}. Academics with a Chinese background criticized Hofstede for being misinformed on the philosophical aspects of Confucianism, as well as being a difficult dimension to measure \cite[][]{fang2003critique}. Other models, such as the GLOBE model, also take issue with some of Hofstede's dimensions and replace them with others, making a total of 9 dimensions \cite[][]{house1999cultural}. The \textit{masculinity-femininity} dimension, for example, is proposed to be instead two dimensions: \textit{gender egalitarianism} and \textit{assertiveness}. This addition of dimensions avoids assuming that assertiveness is either masculine or feminine, which stems from outdated gender stereotypes. Gender stereotypes such as these have also been the subject of critique for Hofstede's model\cite[][]{jeknic2014gender}. Our study agrees with these critiques and, therefore, will avoid considering these for our discussion.
    
    The backgrounds of collectivism in China and individualism in Western countries have been studied before \cite[][]{gao2017chinese}. These backgrounds and the differences in these cultural dimensions could be the underlying cause for differences in expectations. Regardless of the cause, however, measures in the past have proven that these differences in expectations exist \cite[][]{armstrong1997importance}. 

    For our purposes in contrasting Western vs. Chinese satisfaction stemming from expectations, these dimensions could explain why Chinese customers are generally satisfied more often with hard factors while Westerners are satisfied or dissatisfied with soft factors. Perhaps the cultural background of Chinese tourists emphasizes their surroundings and their place in nature and the environment. Chinese historical backgrounds of Confucianism, Taoism, and Buddhism permeate the thought processes of Chinese populations. However, scholars argue that the changes in generations and their economic and recent history gives less importance to these concepts in their lives \cite[][]{gao2017chinese}. 

    Nevertheless, one could argue that a Chinese cultural attribute emphasizes the environment and the place one is in towards satisfaction, rather than the way one is treated. According to previous research, Chinese tourists are collectivist, while Westerners are individualists \cite[][]{kim2000}. A more anthropocentric and individualistic Western culture could result in more of their expectations and priorities be related to how one is treated in social circumstances, rather than the environment one is in. According to \cite{donthu1998cultural}, highly individualistic customers have a higher expectation of empathy and assurance from the provider than collectivist customers. Empathy and assurance from the provider are aspects of service, a soft attribute of a hotel. 

    Among other dimensions in both models, we can consider uncertainty avoidance. High uncertainty avoidance customers would carefully plan their travel and therefore have higher expectations towards service. On the other hand, lower uncertainty avoidance customers would have certain room for risk in their decisions, and therefore face less disappointment with different expectations. However, according to \cite{xiumei2011cultural}, the difference between China and the USA in uncertainty avoidance is not so clear when measuring with the Hofstede typology and the GLOBE typology. While the USA is not representative of Western society, this dimension might not be the one causing the difference in hard-soft attribute satisfaction between Chinese and Western cultures. Another factor, power distance, was also different when measured by Hofstede's method compared to the GLOBAL method, so we decided against making this comparison.

  \subsection{Satisfaction across different price ranges}\label{disc:price}

    In previous sections of this paper, we have mentioned the differences reflected in hotel price ranges. Nevertheless, it is interesting to discuss this further. The most visible change in satisfaction factors across differently priced hotels is the change in voice to describe their satisfaction with the same topics. We can know this by observing the adjective + noun pairs and finding pairs with different adjectives for the same nouns. For example, in English, words describing nouns such as ``location'' or ``hotel'' are ``good'' or ``nice'' in lower-priced hotels. In contrast, the adjectives that pair with the same nouns for more highly-priced hotels are ``wonderful'' and ``excellent''. In Chinese, the change goes from ``\begin{CJK}{UTF8}{gbsn}不错\end{CJK} (not bad)'' to ``\begin{CJK}{UTF8}{gbsn}棒\end{CJK} (great)'' or ``\begin{CJK}{UTF8}{gbsn}赞\end{CJK} (awesome)''. We can infer that the level of satisfaction is high and influences how customers write their reviews. However, when we look at the negative keywords, the change is from ``annoying'' or ``worst'', to ``disappointing''. Here we can see how expectations influence satisfaction and dissatisfaction in different ways. 

    In this paper, we follow the definition of satisfaction by \cite{hunt1975}, where meeting or exceeding expectations produces satisfaction. Therefore, the lack of meeting expectations would cause dissatisfaction. In the cases above, we can infer that a customer that pays more for a higher class of experience has higher expectations. This is true in dissatisfaction, where their expectation is higher in a more expensive hotel. As such, any lack of cleanliness can lead to disappointment or outrage. In the case of English-speaking customers in the 30,000 to 50,000 yen per night price range, cigarette smell is particularly disappointing. However, we consistently see customers with high expectations for high-class hotels react even more positively when satisfied. In the positive case, expectations appear to be exceeded in most cases, judging their reactions. We argue that these are two different kinds of interactions with expectations. We can observe logical expectations. Customers set a standard in their mind that the service must not fall below or be disappointed — for example, a customer being disappointed with dirty rooms or cigarette smell.

    In contrast, we can observe emotional expectations, where customers have a vague idea of having a positive experience. However, they do not measure it against any standard. For example, having a pleasant customer service experience or being treated hospitably by the staff at a high-class hotel. Regardless of their knowledge beforehand of the service to be provided, positive emotions give them a perception of exceeded expectations and high satisfaction. This is where hospitality and service come into play and enhances the experience of the customers. 

    There are interesting differences between Chinese and English-speaking tourists in their change in satisfaction factors to differently priced hotels. For example, we can observe that the Chinese tourists have ``\begin{CJK}{UTF8}{gbsn}购物\end{CJK} (shopping)'' as a top keyword in all the price ranges. In contrast, English-speaking tourists only mention it as a top keyword in the 20,000 to 30,000 yen price range. It is common knowledge in Japan that Chinese tourists coming to Japan with the express intention of shopping are common. \cite{tsujimoto2017purchasing} analyzed the souvenir purchasing behavior of Chinese tourists in Japan. The study shows that common products besides food and drink are: electronics, cameras, cosmetics, medicine, among other more traditional \textit{souvenir} items, such as objects representative of the culture or places they visit \cite{japan2014consumption}. There is an understanding that tourists choose to shop in Japan has more to do with the quality of the items than their relation to the tourist attractions. Our results suggest that Western tourists are engaging more in tourist attractions in comparison with shopping activities. Another interesting difference is that English-speaking tourists start using negative keywords about the hotel's price only after it concerns hotels of 15,000 yen or more, and it rises in its ranking the more expensive the hotel is. In contrast, Chinese customers have this keyword as their top keyword across all price ranges. Previous research suggests that value for money is a key concern for Chinese and Asian tourists \cite[][]{choi2000,choi2001,truong2009}, while Western customers are more concerned with hospitality \cite[][]{kozak2002}.  

    While some aspects of satisfaction and dissatisfaction change depending on the hotel's price range, some other factors stay mostly constant for each culture's customers. For example, appreciation for staff from English-speaking tourists is ranked close to the top satisfaction factor in all the price ranges. Satisfaction for cleanliness by both cultures constantly stays part of the top 10 keywords, except for the most expensive one, where other keywords take that place in the ranking. However, it is still high on the list. Chinese tourists have a high ranking for the word ``\begin{CJK}{UTF8}{gbsn}早餐\end{CJK} (breakfast)'' across all price ranges as well. As discussed in section \ref{disc:location}, transport and location are also important for hotels of all classes and prices. While the ranking of attributes might differ between price ranges, hard and soft attribute proportions also appear to be constant within at most a 13\% margin of error per attribute, often being lower. This suggests that culturally the customers have a particular bias to consider some attributes more than others.

  \subsection{Implications for hotel managers}\label{disc:implications}

    Our study presents two important conclusions: one about hospitality and cultural differences, and another about managerial decisions towards two different populations. As a whole, Chinese tourists are not showing the most satisfaction towards hospitality and service factors. Instead, they focus on the hard attributes of a hotel. Either they do not get as much satisfaction from hospitality as Western tourists or feel that basic language and communication needs are not being met, so they receive a lesser impression. On the other hand, Western tourists are elated with Japanese hospitality, preferring soft attributes to hard-set ones. The other conclusion is that managerial decisions will mostly benefit Western tourists, except that Chinese language improvements and breakfast inclusion can satisfy more Chinese customers. Japan is recently seeing an increase in Chinese students as well as Western students of universities. Hiring students as part-time workers could increase the language services of a hotel.

    To satisfy both customer types, hotel managers need to invest in cleanliness, deodorizing, and making hotel rooms tobacco-free. It could also be recommended to invest in breakfast inclusion and multilingual services and staff preparedness to deal with Chinese and English speakers. Western tourists were also observed to have high comfort standards, which can be improved upon managerially for better reviews. Perhaps it could be suggested to perform surveys of the bedding that is most comfortable for Western tourists. However, not all hotels can invest in all of these factors simultaneously. Our results suggest that satisfying cleanliness needs can satisfy both customer types. A low-cost investment could be to make the facilities tobacco-free. Our results are also divided by price ranges, so a hotel manager can consider which analysis suits their hotel the most.

    While not manageable after a hotel has finished its construction, hard attributes are essential to consider for managers. As previously stated, transport systems and nearby shops are points of interest for both Chinese and Western tourists. Japanese hotel managers have to consider the location and surroundings before the hotel is constructed. A suggestion could be to purchase land and start the construction after public plans to make new subway lines are made. 

    It is left to the managers to consider their business model for the next strategy. One option could be attracting more Chinese customers in number with their observed low budgeting. Another could be attracting more high budget Western customers is on par with their business model. For example, investing more in cleanliness could improve Western customers looking for high-quality lodging satisfaction, even though the price per night would increase. On the other hand, hotels might be considered costly by Chinese customers wherever such an investment is made.

\section{Limitations and Future Work}\label{limitations}

  This paper is not without its limitations. We analyzed satisfaction and dissatisfaction keywords based on whether they appeared on satisfied reviews or dissatisfied ones. Following that, we attempted to understand the context in which these words were used by using a dependency parser and observing the related nouns. However, the study is limited in that it only analyzes the words directly related to each keyword and does not follow the upstream or downstream path down further connections. This means that if the words are used in combination with other keywords, we did not trace the effects of multiple contradicting statements. For example, in the sentence ``The room is good, but the food is lacking'', we would extract ``good food'' and ``lacking food'', but do not consider the fact that both occurred in the same sentence.

  This study analyzed the differences in customers' expectations at different levels of hospitality and service factors by dividing our data into price ranges. However, in the same price range, for example, the highest one, we can find both a western-style five-star resort and a high-end Japanese-style \textit{ryokan}. Services offered in these hotels are very high quality, although very different. However, most of our database is focused on the middle range priced hotels, which is comparably less varied in service. However, there is still a divide between western and Japanese style hotels.

  An essential aspect of this study is that we focus on the satisfaction and dissatisfaction towards expectations of individual aspects of the hotels. This gives us insight into which factors can focus on by hotel managers in applying this knowledge. However, our study was limited in that the overall satisfaction of each customer was not measured. This could be done by rating the volume of text used to describe satisfaction factors against the text volume used for dissatisfaction. However, this imposes a few difficulties that are out of the scope of this study. 

  Another limitation is that a large portion of the Asian tourists coming to Japan is Taiwanese and Korean. We could not analyze these populations because our team members do not know those languages. Aside from that, further typology analysis could not be made because of the nature of the data collected (for example, Chinese men and women of different ages or their Westerner counterparts).

  In future work, we plan to investigate further into this topic. We plan to extend our data to research for different trends for different regions of Japan and different kinds of hotels and between customers traveling alone or in groups, for fun or for work. Another point of interest in this study's future is to use word clusters with similar meanings instead of single words. 

\section{Conclusion}\label{conclusion}

  In this study, our objective was to analyze the differences in satisfaction and dissatisfaction between Chinese and English-speaking customers of Japanese hotels, particularly in the context of Japanese hospitality, \textit{omotenashi}. To answer our research questions \ref{rsq:hospitality} and \ref{rsq:hospitality_both}, we extracted keywords from their online reviews uploaded to the portal sites \textit{Ctrip} and \textit{TripAdvisor} using Shannon's entropy calculations. We used these keywords for sentiment classification via an SVC. We then used dependency parsing and part of speech tagging to extract commonly found pairs of adjectives and nouns, as well as single words. We divided this data by sentiment and hotel price range, considering the most expensive room for one night. 

  In the context of Japanese hospitality, we found that Western tourists had the most satisfaction with staff behavior, cleanliness, and other attributes relating to the hotel's services and hospitality. However, we found that Chinese customers had other concerns than hospitality when studying their satisfaction, more inclined to praise the room, location, or hotel's convenience. We found that both cultures have a different reaction to this hospitality environment. Both cultures have a different way of reacting to different prices. From this, we discussed two possible theories on why Chinese tourists respond differently to Westerners in this environment of \textit{omotenashi}. One theory is that while they are being treated well and react positively, the environment is not compatible with them because of language or culture barriers, which lessens their experience. The second possible theory is that they react differently to hospitality since they do not have the same expectations to be satisfied in the same way. We theorize that a lack of expectations could result in lessened satisfaction, even if the same service is presented. On the other hand, even when they hold high expectations in a highly-priced hotel, Western tourists show that Japanese hospitality exceeds their expectations, judging by their vocabulary for expressing their satisfaction. We consider that Western tourists are more reactive to hospitality and service factors than their Chinese counterparts.

  Lastly, we measured the satisfaction and dissatisfaction factors, referring to a hotel's hard and soft attributes. Soft attributes can be changed via management and staff by an improvement in services. On the other hand, hard attributes are physical and impractical elements to change, such as the size of a room that has already been constructed, the location of a hotel, closeness to convenient spots, or elements out of the control of the hotel managers. We found that for satisfaction, Western tourists favor soft attributes. In contrast, Chinese tourists are more interested in the hard attributes of hotels, consistently across price ranges. For dissatisfaction, Western tourists are also highly inclined to criticize soft attributes, such as cleanliness or cigarette smell in rooms. In contrast, Chinese tourists' dissatisfaction comes evenly from both hard and soft attributes.

  One possible approach for hotel managers is to improve the satisfaction levels of Chinese tourists, who dedicate less percentage of their budget to hotels but are more abundant in number. They are less satisfied with soft attributes but have an identifiable method of improving satisfaction by lessening language barriers and providing a satisfactory breakfast. Another approach we discussed was focusing on the cleanliness and comfort that Western tourists expect and making the hotels tobacco-free. We favor ``long-haul'' Western tourists who spend almost half of their budget on hotels with this strategy. While Westerners are less in number than Chinese tourists, it could prove to have more substantial returns. This is because Chinese customers also favor cleanliness as a satisfaction factor, and both populations could be pleased. This paper provides results and discussion that can be utilized as a guideline for managerial decisions when considering Chinese and Western tourists in Japan. We can observe their stark differences and shared attributes. 

\end{document}
