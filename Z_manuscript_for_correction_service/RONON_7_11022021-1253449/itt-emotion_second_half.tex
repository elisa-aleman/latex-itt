\begin{document}

\section{Results}\label{results}

  \subsection{Experimental results and answers to research questions}

    Our research questions were related to two issues. Based on research questions \ref{rsq:hospitality} and \ref{rsq:hospitality_both}, the objective of this study was to determine how Chinese and Western tourists respond to the \textit{omotenashi} culture, which influences hospitality and service in Japan, and the differences in their perception of the culture. We observed the top-ranking positive factors for Chinese tourists across different price ranges, as shown in Table \ref{tab:freq_res_pos}, and specifically the word ``\begin{CJK}{UTF8}{gbsn}不错\end{CJK} (not bad)'' and its pairings, as shown in Table \ref{tab:adj_zh_pos}. These observations revealed that, while service, cleanliness, and breakfast were praised in most hotels, location was usually rated higher in importance. From the rest of the factors that are lower on the list, we inferred that the list was more populated with hard attributes such as location and transportation availability across different price ranges. The negative keyword usages in Table \ref{tab:freq_res_neg} indicate that there were complaints about the lack of a friendly Chinese environment, but there were more complaints about hard attributes such as the building's age and the distance from other convenient spots. However, most complaints were about the price of the hotel, which included all of the price ranges; therefore, the price was the main concern for Chinese customers with different travel purposes.

    The word ``staff'' is the second or third in all the price ranges in the lists of satisfaction factors in reviews written in English . This word is followed by a few other keywords that are lower in the top 10 list, such as ``helpful'' or ``friendly.'' From the pairings of the top-ranking keyword ``good'' in Table \ref{tab:adj_en_pos}, we realize that customers mainly praise the location, service, breakfast, or English availability. From the negative keyword ``poor'' and its pairings in Table \ref{tab:adj_en_neg}, we deduce that  the Western tourists are also disappointed with service-related concepts. Therefore, both Chinese and English-speaking tourists in Japan have different priorities. However, both populations consider the hotel's location and transportation availability nearby (subways and trains) as secondary yet essential points to their satisfaction. The Chinese customers are primarily satisfied with the room quality in trms of the spaciousness and cleanliness as well as the service of breakfast.

    In contrast, the English-speaking customers were easily upset by uncleanliness and smoke smell from cigarettes. Cigarette smell was an issue even in the middle- and high-class hotels, of which the rooms were priced at more than 30 000 yen per night. For hotels with rooms priced above 50 000 yen per night, however, this problem seemed to disappear from the list of top 10 concerns. Furthermore, the age of the buildings was a concern for both populations. Nevertheless, for all price ranges considered, English-speaking tourists valued staff friendliness over room quality when considering their satisfaction. In contrast, Chinese tourists considered location and transportation more often.

    We can also observe some keywords that are not considered by their counterparts. For example, English-speaking customers mentioned tobacco smell in many reviews. However, it was not statistically identified as a problem for their Chinese counterparts. On the other hand, although they appear in both English and Chinese lists, references to ``\begin{CJK}{UTF8}{gbsn}购物\end{CJK} (shopping)'' are more common in the Chinese lists across hotels of 15 000 yen to 200 000 yen per night. Meanwhile, the term ``shopping'' appeared solely in the top 10 positive keywords list for English speakers who stayed in rooms priced 20 000–30 000 yen per night.

    For research questions \ref{rsq:hard_soft} and \ref{rsq:hard_soft_diff}, we considered how customers of both cultural backgrounds evaluated the hard and soft attributes of hotels. Here, we define hard attributes as those related to the hotel's physical, structural, or environmental aspects. These are often impossible or impractical to change by the hotel management and staff, such as facilities, infrastructure, surroundings, view, and location convenience. In contrast, hotel staff and management can change soft attributes, for example, by improving the hotel's services via training or hiring specialized staff, improving the quantity or quality of amenities, bedsheets, or general cleanliness. Our study discovered that Chinese tourists attached more importance to the hotel's hard attributes, with a tendency of 53 \%, albeit more uniform than the positive evaluations. English-speaking tourists, on the other hand, were more responsive to soft attributes, either positively or negatively. In the case of negative keywords, they were more concerned about the hotel's soft attributes.

    One factor that both populations had in common is that, when perceiving the hotel negatively, the ``\begin{CJK}{UTF8}{gbsn}老\end{CJK} (old),'' ``dated,'' ``outdated,'' or ``\begin{CJK}{UTF8}{gbsn}陈旧\end{CJK} (obsolete)'' aspects of the room or the hotel were surprisingly criticized across most price ranges. However, this is a hard attribute and is unlikely to change for most hotels.
    
  \subsection{Chinese tourists: A big and clean space}\label{disc:zh}

    We found that mainland Chinese tourists were mainly satisfied by big and clean spaces in Japanese hotels. The adjectival pairings extracted with dependency parsing and POS tagging (Table \ref{tab:adj_zh_pos})  imply big and clean rooms. Other mentions included big markets nearby or a big bed. Across different price ranges, the usage of the word ``\begin{CJK}{UTF8}{gbsn}大\end{CJK} (big)'' increased with the increasing price of the hotel. However, they still reacted positively in cheaper hotels. When inspecting closer by taking random samples of the pairs of ``\begin{CJK}{UTF8}{gbsn}大 空间\end{CJK} (big space)'' or ``\begin{CJK}{UTF8}{gbsn}大 面积\end{CJK} (large area),'' we notice that there were also many references to the public bathing facilities in the hotel. Such references were also implied by a word pairing ``\begin{CJK}{UTF8}{gbsn}棒 温泉\end{CJK} (great hot spring).'' In Japan, there are the so-called ``\begin{CJK}{UTF8}{min}銭湯\end{CJK} (\textit{sent\=o}),'' which are artificially constructed public bathing facilities, including saunas and baths with unique qualities. On the other hand, there are natural hot springs, called ``\begin{CJK}{UTF8}{gbsn}温泉\end{CJK} (\textit{onsen}).'' Therefore, visitors can opt for the former or the latter. It is a Japanese custom that all customers first clean themselves in a shower and afterward use the baths naked. It could be a cultural shock for many tourists but a fundamental attraction for many others. 

    However, the size of the room or the bed is a hard attribute. Without considering rebuilding the hotel, it is not trivial to improve this attribute. In contrast, cleanliness is mostly related to soft attributes when we observe its adjectival pairings. We can observe pairs such as ``\begin{CJK}{UTF8}{gbsn}干净 房间\end{CJK} (clean room)'' at the top rank of all price ranges and thereupon ``\begin{CJK}{UTF8}{gbsn}干净 酒店\end{CJK} (clean hotel),'' ``\begin{CJK}{UTF8}{gbsn}干净 总体\end{CJK} (clean overall),'' ``\begin{CJK}{UTF8}{gbsn}干净 环境\end{CJK} (clean environment),'' and ``\begin{CJK}{UTF8}{gbsn}干净 设施\end{CJK} (clean facilities),'' among other examples. In negative reviews, there was a mention of criticizing the ``\begin{CJK}{UTF8}{gbsn}一般 卫生\end{CJK} (general hygiene)'' of the hotel, although it was an uncommon pair. Therefore, we can assert that cleanliness was an important soft attribute for Chinese customers and they were mostly pleased when their expectations were fulfilled. 

    A key component found in Chinese customer satisfaction soft factors was the inclusion of breakfast within the hotel. While other food-related words were extracted, most of them were general, such as ``food'' or ``eating,'' and were lower-ranking. In contrast, the word ``\begin{CJK}{UTF8}{gbsn}早餐\end{CJK} (breakfast),'' which possibly referred to its inclusion in the hotel commodities, was frequently used in positive texts compared to other food-related words. The word ``\begin{CJK}{UTF8}{gbsn}早餐\end{CJK} (breakfast)'' was also present in all price ranges, albeit at different priorities in each of them. For this reason, we regard it as an important factor. From the word pairs of the positive Chinese keywords in Table \ref{tab:adj_zh_pos}, we can also note that ``\begin{CJK}{UTF8}{gbsn}不错\end{CJK} (not bad)'' is paired with ``\begin{CJK}{UTF8}{gbsn}不错 早餐\end{CJK} (nice breakfast)'' in four of the seven price ranges with reviews available as part of the top four pairings. It is only slightly lower in other categories, although it is not depicted on the table. Thus, we consider that a recommended strategy for hotel management is to invest in the inclusion or improvement of hotel breakfast to increase the number of good reviews.

  \subsection{Western tourists: A friendly face and absolutely clean}\label{disc:en}

    From the satisfaction factors of English-speaking tourists, we can see that at least three words were directly related to staff friendliness and services in the general database, ``staff,'' ``helpful,'' and ``friendliness.'' The word ``staff'' is the second most frequently used word for satisfied customers across most price ranges and only third in one of them. In addition, the words ``helpful'' and ``friendly'' follow it lower in the list in most price ranges. The word ``good'' mainly refers to the location, service, breakfast, or English availability in Table \ref{tab:adj_en_pos}. Similar to Chinese customers, Western customers also seemed to enjoy the included breakfasts regarding their satisfaction keyword pairings. However, the relevant word does not appear in the top 10 list directly, in contrast to their Chinese counterparts. The words ``helpful'' and ``friendly'' are mostly paired with ``staff,'' ``concierge,'' ``desk,'' and ``service.'' By considering the negative keyword ``poor'' and its pairings in Table \ref{tab:adj_en_neg}, we realized once again that Western tourists were disappointed with service-related concepts and reacted negatively.

    Another soft attribute that is high on the list for most of the price ranges is the word ``clean''; because this is an adjective, we also explored the word pairings. Customers largely praised ``clean rooms'' and ``clean bathrooms'' and also referred to the hotel in general. When observing the negative keyword frequencies for English speakers, we can find words such as ``dirty'' and ``carpet'' as well as word pairings such as ``dirty carpet,'' ``dirty room,'' and ``dirty bathroom.'' Along with complaints about off-putting smells, we could conclude that Western tourists had high expectations about cleanliness when traveling in Japan.

    An interesting detail of the keyword ranking is that the word ``comfortable'' was high on the satisfaction factors and ``uncomfortable'' was high on the dissatisfaction factors. The words were paired with nouns such as ``bed,'' ``room,'' ``pillow,'' and ``mattress,'' when they generally referred to their sleep conditions in the hotel.
    It seems that Western tourists were particularly sensitive about the hotels’ comfort levels and whether they reached their expectations. The ranking for the negative keyword ``uncomfortable'' is similar across most price ranges except the two most expensive ones, where this keyword disappears from the top 10 list.

    Albeit lower in priority, the price range of 15 000 to 20 000 yen hotels also includes ``free'' as one of the top 10 positive keywords, mainly paired with ``Wi-Fi.'' This price range corresponds to business hotels, where users would expect this feature the most.


  \subsection{Tobacco, what is that smell?}\label{disc:tobacco}

    A concern for Western tourists was the smell of tobacco in their room, which can be regarded as a soft attribute. Tobacco was found not only as a standalone word with ``cigarette'' but also in word pairs in Table \ref{tab:adj_en_neg}, such as ``funny smell.'' By manually inspecting a sample of reviews with this keyword, we noticed that the room was often advertised as nonsmoking; however, the smell permeated the room and curtains. Another common complaint was that there were no nonsmoking facilities available. The smell of smoke can completely ruin some customers’ stay and thus lead to bad reviews, thereby lowering the number of future customers.

    In contrast, Chinese customers seemed unbothered in such a case. Previous research has stated that 49–-60 \% of Chinese men (and 2.0–2.8 \% of women) currently smoke or smoked in the past. This was derived from a sample of 170 000 Chinese adults in 2013–2014, which is high compared to many English-speaking countries \cite[][]{zhang2019tobacco,who2015tobacco}.

    Japan has a polarized view on the topic of smoking. Although it has one of the world’s largest tobacco markets, tobacco use has decreased in recent years. Smoking in public spaces is prohibited in some wards of Tokyo (namely Chiyoda, Shinjuku, and Shibuya). However, it is generally only suggested and not mandatory to lift smoking restrictions in restaurants, bars, hotels, and public areas. However, many places have designated smoking rooms to keep the smoke in an enclosed area and avoid bothering others. Nevertheless, businesses, especially those who cater to certain customers, are generally discouraged by smoking restrictions if they want to maintain their clientele. To cater to all kinds of customers, including Western and Asian, Japanese hotels must provide spaces without tobacco smell. Even if the smoke does not bother a few customers, the lack of such a smell would make it an appropriate space for all customers.


  \subsection{Location, location, location}\label{disc:location}

    The hotel's location, closeness to the subway and public transportation, and availability of nearby shops proved to be of importance to both Chinese and English-speaking tourists. In positive word pairings in Tables \ref{tab:adj_zh_pos} and \ref{tab:adj_en_pos}, we can find pairs such as ``\begin{CJK}{UTF8}{gbsn}不错 位置\end{CJK} (nice location),'' ``\begin{CJK}{UTF8}{gbsn}近 地铁站\end{CJK} (near subway station),'' ``\begin{CJK}{UTF8}{gbsn}近 地铁\end{CJK} (near subway)'' in Chinese texts and ``good location,'' ``great location,'' and ``great view'' as well as single keywords ``location'' and ``shopping'' for English speakers, and ``\begin{CJK}{UTF8}{gbsn}交通\end{CJK} (traffic),'' ``\begin{CJK}{UTF8}{gbsn}购物\end{CJK} (shopping),'' ``\begin{CJK}{UTF8}{gbsn}地铁\end{CJK} (subway),'' and ``\begin{CJK}{UTF8}{gbsn}环境\end{CJK} (environment or surroundings)'' for Chinese speakers. All of these keywords and their location in each population's priorities across the price ranges signify that the hotel's location was a secondary but still important point for their satisfaction. However, since this is a hard attribute, it is not often considered in the literature. By examining examples from the data, we recognized that most customers were satisfied if the hotel was near at least two of the following facilities: subway, train, and convenience stores. 

    Japan is a country with a peculiar public transportation system. During rush hour, the subway is crowded with commuters, and trains and subway stations create a confusing public transportation map for a visitor in Tokyo. Buses are also available, albeit less used than rail systems in metropolitan cities. These three means of transportation are usually affordable in price. There are more expensive means, such as the bullet train \textit{shinkansen} for traveling across the country and taxis. The latter is a luxury in Japan compared to other countries, especially in less developed countries, where this is a low-cost choice. In Japan, taxis provide a high-quality experience with a matching price. Therefore, subway availability and maps or GPS applications as well as a plan to travel the city are of utmost necessity for tourists. 

    Japanese convenience stores are also famous worldwide, because they offer a wide range of services and products, from drinks and snacks to full meals, copy and scanning machines, alcohol, cleaning supplies, personal hygiene items, underwear, towels, and international ATMs. If some trouble occurs, or a traveler forgot to pack a particular item, it is mostly certain that they can find it. 

    Therefore, considering that both transportation systems and nearby shops are points of interest for Chinese and Western tourists, the location of a Japanese hotel must be carefully chosen prior to construction. Although not a top priority, this is a universal factor for both customer groups and is conducive to generate positive reviews.

\section{Discussion}\label{discussion}

  In this section, we explore the possible interactions with Japanese hospitality, the differences between perceptions of Chinese and Western tourists as well as the possible causes, variations in perceptions across different price ranges, and the implications for the industry. We also discuss the differences between the hotel's hard and soft attributes and their contributions to customers' satisfaction.

  \subsection{Western and Chinese tourists in the Japanese hospitality environment}\label{disc:omotenashi}

    To date, scholars have been correcting our historical bias towards the West. Studies have determined that different cultural backgrounds lead to different expectations, which influence tourists' satisfaction. In other words, tourists of a particular culture have different leading satisfaction factors across different destinations. However, Japan presents a particular environment; the spirit of hospitality and service, \textit{omotenashi}, excels and is considered to be of the highest standard across the world. Our study explores whether such an environment can affect different cultures equally Or whether it is attractive only to certain cultures.

    Our results indicate that Western tourists are more satisfied with soft attributes, such as friendly and helpful staff in Japan, than Chinese tourists. As explained earlier in this paper, Japan is well known for its customer service. Respectful language and bowing are not exclusive to high-priced hotels or businesses; these are met in convenience stores as well. The level of hospitality, even in the cheapest convenience store, is starkly different from Western culture. Although it could be a cultural shock to some of them, it is mostly approved. The Japanese staff treats all customers respectfully, and this might be unforeseen by some customers. In higher-priced hotels, the adjectives used to praise the service ranged from normal descriptors like ``good'' to higher levels of praise like ``wonderful staff,'' ``wonderful experience,'' ``excellent service,'' and ``excellent staff.'' Furthermore, \cite{kozak2002} and \cite{shanka2004} have also proven that hospitality and staff friendliness are two determinants of Western tourists' satisfaction.

    However, the negative English keywords indicate that a large part of the dissatisfaction with Japanese hotels stemmed from a lack of hygiene and room cleanliness. Although Chinese customers had solely positive keywords about cleanliness, English-speaking customers deemed many places unacceptable to their standards, particularly hotels with rooms priced below 50 000 yen per night. The most common complaint regarding cleanliness was about the carpet, followed by complaints about cigarette smell and lack of general hygiene. \cite{kozak2002} also proved that hygiene and cleanliness were essential satisfaction determinants for Western tourists. However, in the previous literature, this was linked merely to satisfaction. In contrast, our research revealed that words related to cleanliness were mostly linked to dissatisfaction. We could assert that Westerners had a high standard of room cleanliness compared to their Chinese counterparts.

    According to previous research, Western tourists are already inclined to appreciate hospitality for their satisfaction. When presented with Japanese hospitality, this expectation is met and overcome. In contrast, according to our results, Chinese tourists were more concerned about room quality rather than hospitality, staff, or service. However, when analyzing the word pairs for ``\begin{CJK}{UTF8}{gbsn}不错\end{CJK} (not bad)'' and ``\begin{CJK}{UTF8}{gbsn}棒\end{CJK} (great),'' we can see that they praise staff, service, and breakfast. By observing the percentage of hard to soft attributes in Figure \ref{fig:hard_soft_zh}, however, we discover that Chinese customers were more satisfied with hard attributes compared to Western tourists, who seemed to be meeting more than their expectations.

    It could be considered that Chinese culture does not expect high-level service initially. When an expectation that is not held is met, the satisfaction derived is less than that if it was expected. In contrast, some tourists report a ``nice surprise'': when an unknown need is unexpectedly met, there is more satisfaction. It is necessary to note the difference between these two reactions. The ``nice surprise'' reaction fulfills a need unexpectedly. Perhaps the hospitality grade in Japan does not fulfill a need high enough for the Chinese population, thereby resulting in less satisfaction. For greater satisfaction, a need must be met. However, the word ``not bad'' is at the top of the list in most price ranges, and one of the uses is related to service. Thus, we cannot conclude that they were not satisfied with the service. Instead, they held other factors at a higher priority; thus, the keyword frequency was higher for other pairings.

    Another possibility occurs when we observe the Chinese tourists’ dissatisfaction factors. Chinese tourists may have expectations about the Chinese visitors' treatment that are not being met, even in this high-standard hospitality environment. This could be because Japan is  ia monolingual and has a relatively large language barrier to tourists \cite[][]{heinrich2012making,coulmas2002japan}. While the Japanese effort to accommodate English speakers is slowly developing, efforts for Chinese accommodations can be lagging. Chinese language pamphlets as well as Chinese texts on instructions for the hotel room and its appliances and features (e.g., T.V. channels, Wi-Fi setup, etc.), or the treatment towards Chinese people could be examples. Dissatisfaction in a foreign country is natural if the language native to that region is not known. \cite{ryan2001} also found that communication difficulty was one of the main reasons that Chinese customers would state for not visiting again. However, this issue is not exclusive to Japan.

    Our initial question was whether the environment of high-grade hospitality would affect both cultures equally. This study attempted to determine the answer. It is possible that Chinese customers had high-grade hospitality and were equally satisfied with Westerners. In that case, it appears that the difference in perception stems from a psychological source; expectation leads to satisfaction, and a lack of expectation results in lesser satisfaction. There is also a possibility that Chinese customers are not receiving the highest grade of hospitality because of cultural differences between Japan and China.

    The case representing the higher possibility is unclear from our results. However, competing in hospitality and service includes language services, especially in the international tourism industry. Better multilingual support can only improve the hospitality standard in Japan. Considering that most of the tourists in Japan come from other countries in Asia, multilingual support is beneficial. Proposals for this endeavor include hiring Chinese speaking staff, preparing pamphlets in Chinese, or having a translator application readily available with staff trained in interacting through an electronic translator.

  \subsection{Hard vs. soft satisfaction factors}\label{disc:hard_soft}

    As stated in section \ref{theory_satisfaction}, previous research has mostly focused on the hotel's soft attributes and their influence on customer satisfaction. Examples of soft attributes include staff behavior, commodities, amenities, and appliances that can be improved within the hotel \cite[e.g.,][]{shanka2004,choi2001}. However, hard attributes are not usually analyzed in satisfaction studies. Examples of hard attributes include the hotel's location relative to public transportation and shops, language immersion of the country, noise pollution, or weather. Because the satisfaction factors were decided statistically in our study via customers’ online reviews, we can see the importance of the hard or soft attributes in their priorities.

    Figure \ref{fig:hard_soft_zh} shows that, in regards to Chinese customer satisfaction, in general, 68 \% of the top 10 keywords are hard factors; in contrast, only 20 \% are soft factors. The rates are similar for most price ranges except the highest-priced hotels, where 35 \% of the keywords are undefined. However, the soft attributes are still similar at 18 \%. However, two of these managerial words are all concentrated at the top of the list (``\begin{CJK}{UTF8}{gbsn}不错\end{CJK} (not bad),'' ``\begin{CJK}{UTF8}{gbsn}干净\end{CJK} (clean)''), and the adjective pairs related to soft attributes of ``\begin{CJK}{UTF8}{gbsn}不错\end{CJK} (not bad)'' are also at the top in most price ranges. Chinese tourists may expect spaciousness and cleanliness when coming to Japan. The expectation may be due to reputation, previous experiences, or cultural backgrounds. Some scholars argue that different cultures have different room size perceptions \cite[][]{Saulton2017}. Although the subjects in (study) are German and South Korean, the study presents the results as differences influenced by Asian and Western cultures. We argue that one country is not representative of others’ cultures, and thus, there can be differences between South Korea and China in room size perception. However, a different room size perception may affect the satisfaction of Chinese tourists in contrast to Westerners because Westerners prioritize room size as the price of the hotel room rises. We can compare these results with previous literature, where traveling Chinese tourists choose their destination based on several factors, including cleanliness, nature, architecture, and scenery \cite[][]{ryan2001}. These factors found in previous literature could be linked to the keyword ``\begin{CJK}{UTF8}{gbsn}环境\end{CJK} (environment or surroundings)'' as well. This keyword was found for hotels priced at more than 20 000 yen per night. 

    In contrast, English speakers are mostly satisfied with the hotel's soft attributes. Figure \ref{fig:hard_soft_en} shows that soft attributes are above 48 \% in all price ranges, the highest being 65 \% in the price range of 15 000 to 20 000 yen per night, which corresponds to, for example, affordable business hotels. English-speaking customers also have soft attributes at the top of their list. The exception is the hard attribute that is the hotel's location, which is consistently around the middle of the top 10 lists for all price ranges. If one considers the satisfaction of both Chinese and Western tourists, a hotel can improve its services to attract more customers in the future. Otherwise, if the satisfaction was related more with hard attributes overall for 1020 both cultures, hotels should be built considering the location.

    For both customer groups, the main reason for dissatisfaction was pricing, which can be interpreted as a concern about value for money. However, English-speaking customers complained about price with a lower rank in lower-priced hotels. In contrast, Chinese customers consistently had ``\begin{CJK}{UTF8}{gbsn}价格\end{CJK} (price)'' as the first or second-most concern across all price ranges. A study on Chinese tourists found that they had this concern \cite[][]{truong2009}. However, our results indicate that this concern is less related to the cultural attribute in Japanese hotels and more related with the pricing of hotels overall. Tourists coming to Japan could be either experienced or first-time travelers. However, their expectation of the price for hotels was lower than the actual prices in Japan. In general, Japan is an expensive place to visit, thereby impacting this placement in the ranking. Space is scarce in Japan, and capsule hotels with cramped spaces of 2 x 1 meters cost around 3000 to 6000 yen per night. Bigger business hotel rooms are relatively expensive, ranging from 5000 to 12 000 yen per night. For comparison, hotels in the USA with a similar quality can charge half the price.

    Around half of the dissatisfaction factors for both Chinese and Western customers are caused by issues that could be solved with improved management; this is true for all price ranges. The improvements could be staff training (perhaps in language), hiring professional cleaning services for rooms with cigarette smoke smells, or improving the bedding; however, all of these considerations can be costly. However, this paper provides a useful guideline about the factors to be prioritized and the factors that would be the most suitable for each customer group. Hotels can also use the price range categorization to choose the appropriate strategy. However, once the hotel's location and construction are set for Chinese customers, only a few changes can be made to satisfy them further. As mentioned previously, Chinese language availability is another soft attribute that can be improved with staff and training investment.

    Western tourists are mainly dissatisfied with soft attributes. This is revealed by a low satisfaction level of 35 \% in the highest price range where undefined factors are the majority and a maximum of 78 \% in the price range of 30 000 to 50 000 yen per night in a hotel. Improvement scope for Western tourists is more extensive than that for their Chinese counterparts. As such, it presents a larger investment opportunity. As mentioned earlier in this paper, Westerners are known as ``long-haul'' customers, since they spend more than 45 \% of their budget on hotel lodging.
    Asian tourists spend only 25 \% of their budget on hotels \cite[][]{choi2000}. With bigger returns on managerial improvements, we recommend investing in improving attributes that dissatisfy Western customers, such as cleanliness and removing tobacco smell. Making more hotel facilities tobacco-free and deodorizing the rooms can be a low-cost investment, which could increase returns by several times.

    However, Chinese customers are more in number, even though they tend to spend less on lodging. Attracting a large number of Chinese customers can be a viable strategy for hotels. However, as mentioned before, they tend to focus more on hard attributes; thus, breaking the language barrier breaking is one of the few strategies to accomplish this.

    The basic premise of this study is that different cultures lead to different expectations and satisfaction factors. This premise also plays a role in the differentiation between the preferences of hard or soft attributes.

    In \cite{donthu1998cultural}, subjects from 10 different countries were compared with respect to their expectations of service quality and analyzed based on Hofstede's typology of culture \cite[][]{hofstede1984culture}. The previous study states that, although culture has no specific index, five dimensions of culture can be used to analyze or categorize a country in comparison to others. These are \textit{power distance}, \textit{uncertainty avoidance}, \textit{individualism–-collectivism}, \textit{masculinity–femininity}, and \textit{long-term-–short-term orientation}. In each of these dimensions, at least one element of service expectations was found to be significantly different for countries grouped under contrasting attributes (e.g., individualistic countries vs. collectivist countries, high uncertainty avoidance countries vs. low uncertainty avoidance countries). However, Hofstede's typology has received criticism from academics, particularly for the fifth dimension that Hofstede proposed, which was later added with the alternative name \textit{Confucian dynamic}. Academics with a Chinese background criticized Hofstede for being misinformed on the philosophical aspects of Confucianism as well as considering a difficult dimension to measure \cite[][]{fang2003critique}. Other models, such as the GLOBE model, also consider some of Hofstede's dimensions and replace them with others, making a total of nine dimensions \cite[][]{house1999cultural}. The \textit{masculinity–-femininity} dimension, for example, is proposed to be instead of two dimensions: \textit{gender egalitarianism} and \textit{assertiveness}. This addition of dimensions avoids assuming that assertiveness is either masculine or feminine, which stems from outdated gender stereotypes. Such gender stereotypes have also been the subject of critique on Hofstede's model\cite[][]{jeknic2014gender}. We agree with these critiques and thus avoid considering such stereotypes in our discussion.
    
    The backgrounds of collectivism in China and individualism in Western countries have been studied previously \cite[][]{gao2017chinese}. These backgrounds as well as the differences in these cultural dimensions could be the underlying cause for differences in expectations. Regardless of the cause, however, measures in the past have proven that such differences exist \cite[][]{armstrong1997importance}. 

    For our purposes of contrasting Western vs. Chinese satisfaction stemming from expectations, these dimensions could explain why Chinese customers are generally satisfied more often with hard factors while Westerners are satisfied or dissatisfied with soft factors. The cultural background of Chinese tourists emphasizes their surroundings and their place in nature and the environment. Chinese historical backgrounds of Confucianism, Taoism, and Buddhism permeate the thought processes of Chinese populations. However, scholars argue that the changes in generations and their economic and recent history attaches less importance to these concepts in their lives \cite[][]{gao2017chinese}. 

    Nevertheless, a Chinese cultural attribute emphasizes that the environment and the location  affect satisfaction, rather than the treatment. According to previous research, Chinese tourists are collectivists, whereas Westerners are individualists \cite[][]{kim2000}. A more anthropocentric and individualistic Western culture could correlate more of their expectations and priorities to the treatment in social circumstances, rather than the environment. According to \cite{donthu1998cultural}, highly individualistic customers, in contrast to collectivistic customers, have a higher expectation of empathy and assurance from the provider, which are aspects of service, a soft attribute of a hotel.

    Among other dimensions in both models, we can consider uncertainty avoidance. Customers of high uncertainty avoidance carefully plan their travel and thus have higher expectations towards service. In contrast, customers of lower uncertainty avoidance do not take risks in their decisions and thus face less disappointment with different expectations. However, according to \cite{xiumei2011cultural}, the difference between China and the USA in uncertainty avoidance is not clear when measuring with the Hofstede typology and the GLOBE typology. While the USA is not representative of Western society, uncertainty avoidance may not cause the difference in hard–-soft attribute satisfaction between Chinese and Western cultures. Differences in another factor, power distance, were also noted when measuring by Hofstede's method compared to the GLOBAL method; therefore, power distance was not considered for comparison.

  \subsection{Satisfaction across different price ranges}\label{disc:price}

    In previous sections of this paper, we mentioned the differences reflected in hotel price ranges. The most visible change of satisfaction factors across differently priced hotels is the change in voice when describing their satisfaction with the same topics. We noticed this by observing the adjective–-noun pairs and finding pairs with different adjectives for the same nouns. For example, in English, words describing nouns such as ``location'' or ``hotel'' are ``good'' or ``nice'' in lower-priced hotels. In contrast, the adjectives that pair with the same nouns for higher- priced hotels are ``wonderful'' and ``excellent.'' In Chinese, the change ranges from ``\begin{CJK}{UTF8}{gbsn}不错\end{CJK} (not bad)'' to ``\begin{CJK}{UTF8}{gbsn}棒\end{CJK} (great)'' or ``\begin{CJK}{UTF8}{gbsn}赞\end{CJK} (awesome).'' We can infer that the level of satisfaction is high and influences how customers write their reviews. Regarding the negative keywords, however, the change ranges from ``annoying'' or ``worst'' to ``disappointing.'' Here, we can determine how expectations influence satisfaction and dissatisfaction in different ways. 

    In this paper, we follow the definition of satisfaction by \cite{hunt1975}, where meeting or exceeding expectations produces satisfaction. Therefore, the failure of meeting expectations would cause dissatisfaction. In the aforementioned cases, we can infer that a customer that pays more for a higher-class experience has higher expectations. This is true in dissatisfaction, where their expectation is higher in a more expensive hotel. As such, any lack of cleanliness can lead to disappointment. In the case of English-speaking customers in the 30 000-–50 000 yen per night price range, cigarette smell is particularly disappointing. However, we consistently see customers with high expectations for high-class hotels reacting even more positively when satisfied. In the positive case, expectations appear to be exceeded in most cases, judging from their reactions. We argue that these are two different kinds of expectations: logical and emotional. In the first case, customers are determined that the service must not fall below a specific standard; for example, they can be disappointed with unhygienic rooms or cigarette smell.

    In the second case, in contrast, customers have a vague idea of having a positive experience but do not measure it against any standard. For example, they expect a pleasant customer service experience or a hospitable treatment by the staff at a high-class hotel. Regardless of their knowledge in advance, positive emotions offer them a perception of exceeded expectations and high satisfaction. Thus, hospitality and service enhance the experience of the customers. 

    There are interesting differences between Chinese and English-speaking tourists in their satisfaction factors to differently priced hotels. For example, Chinese tourists have ``\begin{CJK}{UTF8}{gbsn}购物\end{CJK} (shopping)'' as a top keyword in all the price ranges. In contrast, English-speaking tourists mention it only as a top keyword in the 20 000–-30 000 yen price range. It is widely known in Japan that many Chinese tourists visit Japan for shopping. \cite{tsujimoto2017purchasing} analyzed the souvenir purchasing behavior of Chinese tourists in Japan. The study shows that common products besides food and drink are: electronics, cameras, cosmetics, and medicine, among other more traditional \textit{souvenir} items, such as objects that are representative of the culture or places that they visit \cite{japan2014consumption}. Furthermore, tourists’ choice to shop in Japan is more related with the quality of the items rather than their relation to the tourist attractions. Our results suggested that Western tourists were engaging more in tourist attractions rather than shopping activities. Another interesting difference is that English-speaking tourists start using negative keywords about the hotel's price only if it concerns hotels of 15 000 yen or more; thereafter, the more expensive the hotel, the higher the ranking. In contrast, for Chinese customers, this keyword is the top keyword across all price ranges. Previous research suggests that value for money is a key concern for Chinese and Asian tourists \cite[][]{choi2000,choi2001,truong2009}, whereas Western customers are more concerned about hospitality \cite[][]{kozak2002}.  

    While some aspects of satisfaction and dissatisfaction change depending on the hotel's price range, some other factors remain constant for each culture's customers. For example, appreciation for staff from English-speaking tourists is ranked close to the top satisfaction factor in all the price ranges. Satisfaction for cleanliness by both cultures constantly remains part of the top 10 keywords, except for the most expensive one, where other keywords replace keywords related to satisfaction or cleanliness in the ranking; however, they remain still high on the list. Chinese tourists have a high ranking for the word ``\begin{CJK}{UTF8}{gbsn}早餐\end{CJK} (breakfast)'' across all price ranges as well. As discussed in section \ref{disc:location}, transportation and location are also important for hotels of all classes and prices. While the ranking of attributes might differ between price ranges, hard and soft attribute proportions also appear to be constant within a 13 \% margin of error per attribute. This suggests that, from a cultural aspect, the customers have a particular bias to consider some attributes more than others.

  \subsection{Implications for hotel managers}\label{disc:implications}

    Our study reached two important conclusions: one about hospitality and cultural differences and another about managerial decisions towards two different populations. Overall, Chinese tourists did not attach much importance to hospitality and service factors. Instead, they focused on the hard attributes of a hotel. In particular, they were not satisfied with hospitality as much as Western tourists were; otherwise, they felt that basic language and communication needs were not met, thereby they were not much satisfied. Western tourists were highly satisfied with Japanese hospitality and preferred soft attributes to hard ones. The other conclusion is that managerial decisions could mostly benefit Western tourists, except that Chinese language improvements and breakfast inclusion could satisfy Chinese customers to a greater extent. Recently, Japan has been facing an increase in Chinese students as well as students of Western universities. Hiring students as part-time workers could increase the language services of a hotel.

    To satisfy both customer types, hotel managers need to invest in cleanliness, deodorizing, and making hotel rooms tobacco-free. It could also be recommended to invest in breakfast inclusion and multilingual services and staff preparedness to deal with Chinese and English speakers. Western tourists were also observed to have high comfort standards, which could be managerially improved for better reviews. Perhaps it could be suggested to perform surveys of the bedding that is most comfortable for Western tourists. However, not all hotels can invest in all of these factors simultaneously. Our results suggest that satisfying cleanliness needs could satisfy both customer types. A low-cost investment could be to make the facilities tobacco-free. Our results are also divided by price ranges, thereby a hotel manager could consider which analysis suits their hotel the most.

    Albeit not manageable after a hotel is completely constructed, hard attributes should be considered by managers. As stated before, transportation systems and nearby shops are points of interest for both Chinese and Western tourists. Japanese hotel managers should consider the location and surroundings before the hotel is constructed. A suggestion could be to purchase land and start the construction at places where public plans for new subway lines have already been made. 

    The managers must consider their business model for implementing the next strategy. One option could be attracting more Chinese customers with their observed low budgeting. Another could be attracting more big-budget Western customers. For example, investing more in cleanliness could improve Western customers looking for high-quality lodging satisfaction, even for an increased price per night. On the other hand, hotels might be deemed costly by Chinese customers wherever such an investment is made.

\section{Limitations and Future Work}\label{limitations}

  In this study, we analyzed satisfaction and dissatisfaction keywords based on whether they appeared on satisfied reviews or dissatisfied ones. Following that, we attempted to understand the context in which these words were used by using a dependency parser and observing the related nouns. However, our study is limited because it analyzed solely the words directly related to each keyword and did not follow the upstream or downstream path down to further connections. This means that if the words were used in combination with other keywords, we did not trace the effects of multiple contradicting statements. For example, in the sentence ``The room is good, but the food is lacking,'' we extracted ``good food'' and ``lacking food'' but did not consider the fact that both occurred in the same sentence.

  This study analyzed the differences in customers' expectations at different levels of hospitality and service factors by dividing our data into price ranges. However, in the same price range, for example, the highest one, we can find both a Western-style five-star resort and a high-end Japanese style \textit{ryokan}. Services offered in these hotels are of high quality, albeit very different. However, most of our database was focused on the middle range priced hotels, the services of which are comparably less varied. However, there is still a division between Western and Japanese style hotels, which constitutes another limitation of our study.

  An essential aspect of this study is that we focused on the satisfaction and dissatisfaction towards the expectations of the individual aspects of the hotels. This gave us insight into the factors that can be considered by hotel managers. However, the overall satisfaction of each customer was not measured. This measurement can be performed by rating the volume of text used to describe satisfaction and dissatisfaction factors. However, this would impose a few difficulties, which are out of the scope of this study. 

  Another limitation is that a large portion of Asian tourists coming to Japan are Taiwanese and Korean. We could not analyze these populations because our team members did not know those languages. Moreover, further typology analysis could not be made because of the nature of the data collected (for example, Chinese men and women of different ages or their Westerner counterparts).

  In future work, we plan to investigate these topics further. We plan to extend our data to research of different trends and regions of Japan, different kinds of hotels, and customers traveling alone or in groups, whether for fun or for work. Another point of interest in this study's future work is to use word clusters with similar meanings instead of single words. 

\section{Conclusion}\label{conclusion}

  In this study, our objective was to analyze the differences in satisfaction and dissatisfaction between Chinese and English-speaking customers of Japanese hotels, particularly in the context of Japanese hospitality, \textit{omotenashi}. To answer our research questions \ref{rsq:hospitality} and \ref{rsq:hospitality_both}, we extracted keywords from their online reviews uploaded to the portal sites \textit{Ctrip} and \textit{TripAdvisor} using Shannon's entropy calculations. We used these keywords for sentiment classification via an SVC. We then used dependency parsing and part of speech tagging to extract commonly found pairs of adjectives and nouns as well as single words. We divided these data by sentiment and hotel price range by considering the most expensive room for one night. 

  In the context of Japanese hospitality, we found that Western tourists were most satisfied with staff behavior, cleanliness, and other attributes related to the hotel's services and hospitality. However, we found that Chinese customers had concerns other than hospitality when studying their satisfaction; they were more inclined to praise the room, location, and hotel's convenience. We found that the two cultures had different reactions to the hospitality environment and the prices. Thus, we discussed two possible theories on the reasons why Chinese tourists responded differently from Westerners in the environment of \textit{omotenashi}. One theory is that, although they were treated well and thus reacted positively, the environment was not compatible with them because of language or culture barriers, which deteriorated their experience. The second possible theory is that they reacted to hospitality differently since they did not have the same expectations. We theorized that a lack of expectations could result in lessened satisfaction, even if the same service was present. On the other hand, even when they held high expectations in a high-priced hotel, Japanese hospitality exceeded Western tourists’ expectations, judging by their vocabulary for expressing their satisfaction. We considered that Western tourists were more reactive to hospitality and service factors than their Chinese counterparts.

  Lastly, we measured the satisfaction and dissatisfaction factors, that is, a hotel's hard and soft attributes. Soft attributes can be changed via management and staff by an improvement in services. In contrast, hard attributes are physical and impractical elements to change, such as the size of a room that has already been constructed, the location of a hotel, closeness to convenient spots, or elements out of the control of the hotel managers. We found that, for satisfaction, Western tourists favored soft attributes in contrast to Chinese tourists, who were more interested in the hard attributes of hotels across all the price ranges consistently. For dissatisfaction, Western tourists were also highly inclined to criticize soft attributes, such as cleanliness or cigarette smell in rooms. In contrast, Chinese tourists' dissatisfaction derived from both hard and soft attributes evenly.

  One possible approach for hotel managers is to improve the satisfaction levels of Chinese tourists, who dedicate lower percentage of their budget to hotels but are more abundant in number. They are less satisfied with soft attributes but have an identifiable method for improving satisfaction by lessening language barriers and providing a satisfactory breakfast. Another approach was focused on the cleanliness, comfort, and tobacco-free space expected by Western tourists. ``Long-haul'' Western tourists, who spend almost half of their budget on hotels with this strategy, were favored. Although Westerners are less in number than Chinese tourists, it could be proven that they have more substantial returns. This is because Chinese customers also favor cleanliness as a satisfaction factor, and both populations could be pleased. This paper provides results and discussion that can be utilized as a guideline for managerial decisions when considering Chinese and Western tourists in Japan. We can observe their stark differences as well as shared attributes. 

\end{document}

