\begin{document}

\title{Differences in Chinese and Western tourists faced with Japanese hospitality: A natural language processing approach}

\begin{abstract}

  The Japanese spirit of hospitality and service \textit{Omotenashi} is known worldwide for its excellence. Recent years show a steady increase in international tourists coming to Japan. Chinese tourists, especially, have been steadily increasing. However, before the shift that has brought a global perspective in recent years, most tourist behavior studies were biased for the Western world. Previous research shows that different cultural backgrounds result in different expectations and, arguably, different satisfaction factors. Knowing this, a cross-cultural study of differences between Chinese and Western cultures after the current boom in the Chinese economy in the high standard Japanese hospitality environment is fascinating. Will the top-grade hospitality of Japan influence both populations equally, or will their cultural differences set them apart? Will they be satisfied with the soft attributes like service or be more concerned with hard attributes like location and facilities? We bring light to these questions and the differences in each population's satisfaction and dissatisfaction factors in different price ranges. Taking advantage of Web 2.0, we applied Shannon's entropy to extract these factors automatically and then use them in an SVM to classify a more extensive data set. We then used dependency parsing and part of speech tagging to extract which nouns were tied to praising adjectives. We found that Chinese tourists are less concerned with hospitality and more with room quality than Western tourists. The latter were delighted by the staff behavior. We also found that Chinese tourists are concerned with the lack of a Chinese friendly environment, and Western customers are unsatisfied with dirty rooms or the smell of cigarettes.

  \keywords{Sentiment Analysis\and Hotels and Lodging\and Text Mining\and Chinese\and English\and Satisfaction and Dissatisfaction Factors}

\end{abstract}

\linenumbers

\section{Introduction}\label{intro}

  Japan has been known historically for its hospitality being the highest grade. The spirit of Japanese hospitality is celebrated worldwide in a single Japanese word: \textit{Omotenashi}. With roots in Japanese history and tea ceremony, their hospitality is famous around the world \cite[][]{al2015characteristics}. Therefore it would stand to reason that tourists visiting Japan would have this hospitality as their first and foremost satisfaction factor. However, it is known that customers from different countries and cultures hold different expectations \cite[][]{engel1990}. Thus, it could be theorized that their satisfaction factors should be different. How will different cultures react and perceive hotels and their hospitality in this context? Our study attempts to bring light to this with two essential tourist populations that differ in culture to Japan: Chinese and Western tourists. 

  In the last couple of decades, the Japanese economy has been more and more affected by an increase in inbound international tourism \cite[][]{jones2009}. There was a Year-on-Year Growth Rate of 19.3\% in 2017, with a total of \num[group-separator={,}]{28691073} inbound tourists that year \cite[][]{jnto2003-2019}. From this total, the tourist population was mostly Asian (86.14\%), and approximately a fourth of the total (25.63\%) came from China. Western countries, counting English-speaking countries and Europe, make for 11.4\% of the total, with a 7.23\% of the total being countries where English is the official or the de facto national language. The effect of Chinese tourists on international economies is increasing. From that, the number of researchers interested in this phenomenon has been increasing as well. \cite[][]{sun2017}. With these and other multicultural tourist populations, the tourist market is more and more diverse. Diversity in customers' cultural backgrounds means that their expectations when staying at a hotel will also be varied. Therefore, Hotel management needs to cater to these needs and expectations to increase customer satisfaction, maintain a good reputation, and generate positive word-of-mouth.

  However, many tourist behavior analyses have been performed with Western subjects. As such, a gap in knowledge and discussion existed until recent decades. Those studies that do include Asian populations in their analysis most commonly study Chinese tourist behavior \cite[e.g.][]{liu2019, chang2010, dongyang2015}. The few that compare Asian to Western tourist behavior \cite[e.g.][]{choi2000} are commonly survey or interview-based studies with small samples. These, while valid, can have their limitations. This gap in research creates a need for cross-cultural studies for the increasing Asian and Western tourist populations. It could be said that Westerners make for a smaller portion of the tourist population compared to Asians. However, according to \cite{choi2000}, Westerners are known as ``long-haul" customers, spending more than 45\% of their budget on hotel lodging. In comparison, their Asian counterparts only spend 25\% of their budget on hotels. Therefore, it is essential to study Asian and Western tourist populations, their differences, and contrast with the existing literature results.

  With the advent of Web 2.0 and customer review websites, researchers realized the benefits of online reviews for research, and their importance for sales  \cite[][]{ye2009, basuroy2003}, customer consideration \cite[][]{vermeulen2009} and perception of services and products \cite[][]{browning2013}, among other effects of online interactions between customers \cite[e.g.][]{xiang2010, ren2019}. Consequentially, tourism research also began to use information collected online for data mining analysis, such as opinion mining \cite[e.g.][]{hu2017436}, predicting hotel demand from online traffic \cite[][]{yang2014}, recommender systems \cite[e.g.][]{loh2003}, and more. Data mining, machine learning, and big data methodologies can increase the number of manageable samples per study. The increase can be from the hundred samples manually analyzed by researchers to the hundreds of thousands automatically analyzed by machines. This technology can not only help confirm existing theories but also lead to finding new patterns and to knowledge discovery \cite[][]{fayyad1996data}. 

  In this study, we take advantage of the availability of enormous amounts of online reviews of Japanese hotels by both Mainland Chinese tourists posting in \textit{Ctrip} and Western English-speaking tourists populations posting in \textit{TripAdvisor}. With this data, we can confirm existing theories about their differences in behavior and explore the data to discover factors that could have been overlooked in the past. To do this, we use machine learning to automatically classify review sentences as positive or negative opinions of the hotel. We then perform a statistical extraction of the topics that concerns the customers of each population the most.

\section{Research objective}\label{research_objective}

  This study's objective is to determine the difference in factors driving satisfaction and dissatisfaction between Chinese and English-speaking tourists in the context of high-grade hospitality of Japanese hotels using text-mining techniques. We aim to contrast customer groups' satisfaction and dissatisfaction factors across several price ranges. We use machine learning to classify texts' sentiment and natural language processing to study commonly used word pairings. More importantly, we also intend to measure how hard and soft attributes influence customer groups' satisfaction and dissatisfaction. We define hard attributes as relating to physical aspects and environmental aspects, such as the hotel's facilities, the hotel's location, infrastructure, and real estate nearby. In contrast, soft attributes are the hotel's non-physical attributes related to services, staff, or management.

  Our proposal includes using large scale data from online hotel reviews in Chinese and English to study their differences in a statistical manner. In the past, survey-based studies have provided a theoretical background for a few specific tourist populations of a single culture or that travel with a single purpose. These studies' short scope often leads to difficulties in observing cultural and language differences in a single study.

  Our study attempts to uncover the difference in satisfaction and dissatisfaction factors between different cultures. These factors can become the focal point for improving the tourism and service industries and increasing customer satisfaction. Satisfied customers will then write more positive online reviews that will, in turn, increase sales and attract new customers.

\section{Theoretical background and hypothesis development}\label{theory_hypothesis}

  \subsection{Japanese hospitality and service: \textit{Omotenashi}}\label{theory_omotenashi}

    The spirit of Japanese hospitality, or \textit{Omotenashi}, has roots in the countries history. However, to this day, it is regarded as the highest standard \cite[][]{ikeda2013omotenashi, al2015characteristics}. There is even a famous phrase in customer service in Japan: \textit{okyaku-sama wa kami-sama desu}, or translated ``The customer is god''. Some say that \textit{omotenashi} originated from the old Japanese art of the tea ceremony in the 16th century. However, other scholars found that its roots come from even earlier, in the form of formal banquets in the 7th-century \cite[][]{aishima2015origin}. The practice of high standards in hospitality has survived throughout the years. Today, it permeates all business practices in Japan, from the cheapest convenience stores to the most expensive ones. Manners, service, and respect towards the customer are taught to workers in their training. High standards are always followed as to not fall behind in the competition. In Japanese businesses, hotels included, staff members are trained to speak in \textit{sonkeigo}, or ``respectful language", one of the most formal of the Japanese formality syntaxes. They are also trained to bow with different depths depending on the situation, where a light bow could be used to say: ``Please, allow me to guide you". Deep bows are also used to apologize for any inconvenience the customer could have, followed by a very respectful apology. In fact, despite the word \textit{omotenashi} being translated directly as ``hospitality'', it includes both the concepts of hospitality and service \cite[][]{Kuboyama2020}. This hospitality culture permeates every kind of business with customer interaction in Japan. A simple convenience shop could express all of these hospitality and service standards, which are not exclusive to hotels.  

    It stands to reason that this cultural aspect of hospitality would be a positive aspect that would be at the top of satisfaction for any customer. However, in many cases, other factors such as proximity to a convenience store, transport availability, or room quality might be more critical to a customer.  In this study, we cannot determine whether or not a hotel is practicing with the cultural standards of \textit{omotenashi} directly. Instead, we consider it as a cultural factor that influences all businesses in Japan. We then observe the customers' evaluations regarding service and hospitality factors in general, which can be compared to other places and business practices in the world. In summary, our study considers the influence of the cultural aspect of \textit{omotenashi} while analyzing the evaluations of service and hospitality factors that are universal to all hotels in any country.

    Therefore we pose a research question for our study:

    \begin{subrsq}
    \begin{rsq}
    \label{rsq:hospitality}
    To what degree are Chinese and Western tourists satisfied with Japanese hospitality factors such as staff behavior or service?
    \end{rsq}

    However, Japanese hospitality comes from Japanese culture. Different cultures interacting with it could have a different evaluation of it. While some might be impressed by it, some might consider other factors more important to their stay in a hotel. This point leads us to a derivative of the above research question:

    \begin{rsq}
    \label{rsq:hospitality_both}
    Do Western and Chinese tourists have a different evaluation of Japanese hospitality factors such as staff behavior or service?
    \end{rsq}
    \end{subrsq}

  \subsection{Customer satisfaction and dissatisfaction towards individual factors during hotel stay}\label{theory_satisfaction}

    Customer satisfaction in tourism has been analyzed since decades past, \cite{hunt1975} having defined customer satisfaction as the realization or overcoming of expectations towards the service. \cite{oliver1981} defined it as an emotional response to the provided services in retail and other contexts, and \cite{oh1996} reviewed the psychological processes of customer satisfaction for the hospitality industry. It is generally agreed upon that satisfaction and dissatisfaction stem from the individual expectations of the customer. As such, \cite{engel1990} states that each customer's background, therefore, influences satisfaction and dissatisfaction. Previous studies on the dimensions of culture that influence differences in expectations have been performed in the past \cite{donthu1998cultural}. Western and Chinese customers can then have very different satisfaction and dissatisfaction factors since they have different backgrounds and cultures. These varying backgrounds will lead to varying expectations of the hotel services, the experiences they want to have while staying at a hotel, and the level of comfort that they will have. These expectations will be there from the moment that they choose the hotel throughout their stay. In turn, these different expectations will determine the distinct factors of satisfaction and dissatisfaction for each kind of customer and the order in which they prioritize them. 

    Because of their different origins, expectations, and cultures, it stands to reason Chinese and Western tourists could have completely different factors to one another. Therefore, it could be that some factors do not appear in the other reviews at all. For example, between different cultures, it can be that a single word can express some concept that would take more words in the other language. So we must measure their differences or similarities at their common ground as well.

    However, in this study, we study not overall customer satisfaction but the satisfaction and dissatisfaction that stem from individual-specific expectations, be they conscious or unconscious. For example, suppose a customer has a conscious expectation of a comfortable bed and a wide shower, and it is realized during their visit. In that case, they will be satisfied with this matter. However, suppose that same customer with a conscious expectation of a comfortable bed experienced loud noises at night. In that case, they can be dissatisfied with a different aspect, regardless of the satisfaction towards the bed. Then, the same customer might have packed their toiletries, thinking that the amenities might not include those. They can then be pleasantly surprised with good quality amenities and toiletries, satisfying an unconscious expectation. This definition of satisfaction does not allow us to examine overall customer satisfaction. However, it will allow us to examine the factors that a hotel can revise individually and how a population perceives them as a whole. In our study, we consider the definitions by \cite{hunt1975} that satisfaction is a realization of an expectation, and posit that customers can have different expectations towards different service aspects. Therefore, in our study, we define satisfaction as the emotional response to the realization or overcoming of conscious or unconscious expectations towards an individual aspect or factor of a service. On the other hand, dissatisfaction is the emotional response to the lack of a realization or under-performance of these conscious or unconscious expectations towards specific service aspects.

    Studies on customer satisfaction \cite[e.g.][]{truong2009, romao2014, wu2009} commonly use the Likert scale \cite[][]{likert1932technique} (e.g. 1 to 5 scale from strongly dissatisfied to strongly satisfied) to perform statistical analysis of which factors relate most to satisfaction on the same dimension as dissatisfaction \cite[e.g.][]{chan201518, choi2000}. The Likert scale's use leads to correlation analyses where one factor can lead to satisfaction, implying that the lack of it can lead to dissatisfaction. However, a binary distinction (satisfied or dissatisfied) could allow us to analyze the factors that correlate to satisfaction and explore factors that are solely linked to dissatisfaction. There are fewer examples of this approach, but studies have done this in the past \cite[e.g.][]{zhou2014}. This method can indeed decrease the extent to which we can analyze degrees of satisfaction or dissatisfaction. However, it has the benefit that it can be applied to a large sample of text data via automatic sentiment detection techniques using artificial intelligence. 

    Previous research has also focused more on soft attributes, with little focus on hard attributes, like location or infrastructure, mostly focusing only on facilities \cite[e.g.][]{shanka2004, choi2001}. However, hard factors, which are uncontrollable by the hotel staff, can play a part in the customers' choice behavior and satisfaction. Examples of these factors include the hotel's surroundings, location, language immersion of the country as a whole, or touristic destinations, and the hotel's integration with tours available nearby, among other factors. 

    This leads to another couple of research questions:

    \begin{subrsq}
    \begin{rsq}
    \label{rsq:hard_soft}
    To what degree does satisfaction and dissatisfaction stem from hard and soft attributes of the hotel?
    \end{rsq}

    \begin{rsq}
    \label{rsq:hard_soft_diff}
    How differently do Chinese and Western customers perceive hard and soft attributes of the hotel?
    \end{rsq}
    \end{subrsq}

    The resulting proportions of hard attributes to soft attributes for each population could measure how much the improvement of management in the hotel can increase future satisfaction in customers. 

  \subsection{Chinese and Western tourist behavior}\label{theory_zh_en}

    % Asians vs. Western behavior
    In the past, social science and tourism studies focused extensively on Western tourist behavior in other countries. Recently, however, with the rise of Chinese outbound tourism, both academic researchers and businesses have decided to study Chinese tourist behavior. Explaining this increase, \cite{sun2017} analyzed the number of studies related to Chinese tourists from 2001 to 2012 and found a steady increase until 2007, followed by the rapid growth of Chinese tourism studies. This increase in Chinese tourist behavior research resulted in several studies focusing on only the behavior of this subset of tourists. To this day, studies and analyses specifically comparing Asian and Western tourists are scarce, and even less the number of studies explicitly comparing Chinese and Western tourists. One example is a study by \cite{choi2000}, who found that Western tourists visiting Hong Kong are satisfied more with room quality, while Asians are satisfied with the value for money. Another study by \cite{bauer1993changing} found that Westerners prefer hotel health facilities. At the same time, Asian tourists were more inclined to enjoy the Karaoke facilities of hotels. Both groups tend to have high expectations for the overall facilities. Another study done by \cite{kim2000} found that American tourists were found to be individualistic and motivated by novelty. In contrast, Japanese tourists were collectivist and motivated by prestige and family, with an escape from routine and increased knowledge as a common motivator. 

    One thing to note with the above Asian vs. Western analyses is that they were performed before 2000 and that they are not Chinese specific but study Asian people in general. Meanwhile, the current Chinese economy boom is increasing the influx of tourists of this nation. The resulting increase in marketing and the creation of guided tours for Chinese tourists could have created a difference in tourists' perceptions and expectations. In turn, if we follow the definition of satisfaction by \cite{hunt1975}, that change in expectations could have influenced their satisfaction factors when traveling. Another note is that these studies were performed with questionnaires in places where it would be easy to locate tourists, i.e., airports. However, our study of online reviews takes the data that the hotel customers uploaded themselves. This data makes the analysis unique in exploring their behavior compared with Western tourists via factors that are not considered in most other studies. Furthermore, our study is unique in observing the customers in the specific environment of high-level hospitality in Japan.

    More recent studies have surfaced as well. A cross-country study \cite[][]{FRANCESCO201924} using posts from U.S.A. citizens, Italians, and Chinese tourists, determined using a text link analysis that customers from different countries indeed have a different perception and emphasis of a few predefined hotel attributes. According to their results, U.S.A. customers perceive cleanliness and quietness most positively. In contrast, Chinese customers perceive budget and restaurant above other attributes. Another couple of studies \cite[][]{JIA2020104071, HUANG2017117} analyze differences between Chinese and U.S. tourists using text mining techniques and more massive datasets, although in a restaurant context. 

    These last three articles focus on U.S.A. culture, while our study focuses on Western culture. Another difference with our study is that of the context of the study. The first study \cite[][]{FRANCESCO201924} was done with the context of tourists from three countries staying in hotels across the world. The second one chose restaurant reviews from the U.S.A. and Chinese tourists eating in three countries in Europe. The third is analyzing restaurants in Bejing.

    On the other hand, our study focuses on Western culture, instead of a single Western country, and Chinese culture clashing with the hospitality environment in Japan, specifically. Japan's importance in this analysis comes from the unique environment of high-grade hospitality that the country presents. In this environment, do customers hold their satisfaction to this hospitality regardless of their culture, or are other factors more relevant to the customers? Our study measures this at a large scale across different hotels in Japan. 

    % Universal behavior
    Other studies have gone further and studied people from many countries in their samples and performed a more universal and holistic (not cross-culture) analysis. \cite{choi2001} analyzed hotel guest satisfaction determinants in Hong Kong with surveys in English, Chinese and Japanese translations, with people from many countries in their sample. \cite{choi2001} found that staff service quality, room quality, and value for money were the top satisfaction determinants. As another example, \cite{Uzama2012} produced a typology for foreigners coming to Japan for tourism, without making distinctions for their culture, but their motivation in traveling in Japan. In another study, \cite{zhou2014} analyzed hotel satisfaction using English and Mandarin online reviews from guests staying in Hangzhou, China coming from many different countries. The general satisfaction score was noticed to be different in those countries. However, a more in-depth cross-cultural analysis of the satisfaction factors was not performed. As a result of their research, \cite{zhou2014} thus found that customers are universally satisfied by welcome extras, dining environments, and special food services. 

    % Western behavior 
    Regarding Western tourist behavior, a few examples can tell us what to expect when analyzing our data. \cite{kozak2002} found that British and German tourists' satisfaction determinants while visiting Spain and Turkey were hygiene and cleanliness, hospitality, the availability of facilities and activities, and accommodation services. \cite{shanka2004} found that English-speaking tourists in Perth, Australia were most satisfied with staff friendliness, the efficiency of check-in and check-out, restaurant and bar facilities, and lobby ambiance. 

    % Chinese behavior
    Regarding outbound Chinese tourists, academic studies about Chinese tourists have increased \cite[][]{sun2017}. Different researchers have found that Chinese tourist populations have several specific attributes. According to \cite{ryan2001} and their study of Chinese tourists in New Zealand, Chinese tourists prefer nature, cleanliness, and scenery in contrast to experiences and activities. \cite{dongyang2015} studied Chinese tourists in the Kansai region of Japan and found that Chinese tourists are satisfied mostly with exploring the food culture of their destination, cleanliness, and staff. Studying Chinese tourists in Vietnam, \cite{truong2009} found that Chinese tourists are highly concerned with value for money. According to \cite{liu2019}, Chinese tourists tend to have harsher criticism compared with other international tourists. Moreover, as stated by \cite{gao2017chinese}, who analyzed different generations of Chinese tourists and their connection to nature while traveling, Chinese tourists prefer nature overall. However, the younger generations seem to do so less than their older counterparts. 

    % Studies are not universal.
    Although the studies focusing only on Chinese tourists or Western tourists have a narrow view, their theoretical contributions are valuable. We can see that depending on the study and the design of questionnaires and the destinations, the results can vary greatly. Not only that, but while there seems to be some overlap in most studies, some factors are completely ignored in one study but not in the other. Since our study uses data mining, each factor's definition is left for hotel customers to decide en masse via their reviews. This means that the factors will be selected through statistical methods alone, instead of being defined by the questionnaire. Our method allows us to find factors that we would not have contemplated. It also avoids enforcing a factor on the mind of study subjects by presenting them with a question that they did not think of by themselves. This large variety of opinions in a well-sized sample, added to the automatic findings of statistical text analysis methods, gives our study an advantage compared to others with smaller samples. This study could also help us analyze the satisfaction and dissatisfaction factors cross-culturally and compare them with the existing literature.

    % Reviewer comparison table
    Undoubtedly previous literature has examples of other cross-culture studies of tourist behavior and to further highlight our study and its merits. A contrast is shown in Table \ref{tab:lit-rev}. This table shows that older studies were conducted with surveys and had a different study topic. These are changes in demand \cite[][]{bauer1993changing}, tourist motivation \cite[][]{kim2000}, and closer to our study, satisfaction levels \cite[][]{choi2000}. However, our study topic is not the levels of satisfaction but the factors that drive it and dissatisfaction, which is overlooked in most studies. Newer studies with larger samples and similar methodologies have emerged, although two of these study restaurants instead of hotels \cite[][]{JIA2020104071, HUANG2017117}. One important difference is the geographical focus of their studies. While \cite{FRANCESCO201924} , \cite{JIA2020104071} and \cite{HUANG2017117} have a multi-national focus, we instead focus on Japan. The focus on Japan is important because of its top rank in hospitality across all types of businesses. This raises the question: in such an environment, will the customers be universally satisfied with this factor, or will they have differing views within their cultures? Our study brings light to the changes, or lack thereof, in different touristic environments where an attribute can be considered excellent. The number of samples in other text-mining studies is also smaller than ours in comparison. Apart from that, every study has a different text mining method.

  \subsection{Data mining, machine learning, knowledge discovery and sentiment analysis}\label{theory_data}

    % Explain the discovery of theory inside the data
    In the current world, data is presented to us in larger and larger quantities. Today's data sizes were commonly only seen in very specialized large laboratories with supercomputers a couple of decades ago. However, they are now standard for market and managerial studies, independent university students, and any scientist connecting to the Internet. Such quantities of data are available to study now more than ever. Nevertheless, it would be impossible for researchers to parse all of this data by themselves. As \cite{fayyad1996data} summarizes, data by itself is unusable until it goes through a process of selection, preprocessing, transformation, mining, and evaluation. Only then can it be established as knowledge. With the tools available to us in the era of information science, algorithms can be used to detect patterns that would take researchers too long to recognize. These patterns can, later on, be evaluated to generate knowledge. This process is called Knowledge Discovery in Databases. 

    % Text mining 
    Now, there are, of course, many sources of numerical data to be explored.  However, perhaps what is most available and interesting to managerial purposes is the resource of customers' opinions in text form. Since the introduction of Web 2.0, a never before seen quantity of valuable information is posted to the Internet at a staggering speed. Text mining has then been proposed more than a decade ago to utilize this data \cite[e.g.][]{rajman1998text,nahm2002text}. Using Natural Language Processing, one can parse language in a way that translates to numbers so that a computer can analyze it. Since then, text mining techniques have improved over the years. This has been used in the field of hospitality as well for many purposes, including satisfaction analysis from reviews \cite[e.g][]{berezina2016, xu2016, xiang2015, hargreaves2015, balbi2018}, social media's influence on travelers \cite[e.g.][]{xiang2010}, review summarization \cite[e.g.][]{hu2017436}, perceived value of reviews \cite[e.g][]{FANG2016498}, and even predicting hotel demand using web traffic data \cite[e.g][]{yang2014}.

    % Sentiment Analysis
    More than only analyzing patterns within the text, researchers have found how to determine the sentiment behind a statement based on speech patterns, statistical patterns, and other methodologies. This method is called sentiment analysis or opinion mining. A precursor of this method was attempted decades ago \cite[][]{stone1966general}. With sentiment analysis, one could use patterns in the text to determine whether a sentence was being said with a positive opinion, a critical opinion. This methodology could even determine other ranges of emotions, depending on the thoroughness of the algorithm. Examples of sentiment analysis include ranking products through online reviews \cite[e.g][]{liu2017149, zhang2011}, predicting political poll results through opinions in Twitter \cite[][]{oconnor2010}, and so on. In the hospitality field, it has been used to classify reviewers' opinions of hotels in online reviews \cite[e.g.][]{kim2017362, alsmadi2018}. 

    % Machine Learning
    The algorithm used for sentiment analysis in our study is called a Support Vector Machine. It is a form of supervised machine learning used for binary classification. This means a sample of labeled training data is given to the algorithm to detect patterns in the data and use those patterns to establish a method for classifying other unlabeled data automatically. Machine learning is a general term used for algorithms that, when given data, will automatically use that data to "learn" from its patterns and apply them for improving upon a task. Learning machines can be supervised, as in our study, where the algorithm has manually labeled data to know the correct task result template. Machine learning can also be unsupervised, where without any pre-labeled data. In this latter case, the machine will analyze the structure and patterns of the data and perform a task based on its conclusions. Our study calls for a supervised machine since text analysis can be intricate. Many patterns might occur, but we are only interested in satisfaction and dissatisfaction labels. Consequently, we teach the machine through previously labeled text samples. 

    Machine learning and data mining are two fields with a significant overlap since they can use each other's methods to achieve the task at hand. Machine learning methods focus on predicting new data based on known properties and patterns of the given data. Data mining, on the other hand, is discovering new information and new properties of the data. Our machine learning approach will learn the sentiment patterns of our sample texts showing satisfaction and dissatisfaction and using these to label the rest of the data. We are not exploring new patterns in the sentiment data. However, we are using sentiment predictions for knowledge discovery in our database. Thus our study is a data mining experiment based on machine learning.

    % Data exploration
    Because the methodology for finding patterns in the data is automatic and statistical, it is both reliable and unpredictable. Reliable in that the algorithm will find a pattern by its nature. Unpredictable in that since it has no intervention from the researchers in making questionnaires, it can result in anything that the researchers could expect. These qualities determine why, much like actual mining, data mining is mostly exploratory. One can never be sure that one will find a specific something. However, we can make predictions and estimates about finding knowledge and what kind of knowledge we can uncover. The exploration of large opinion datasets with these methods is essential. The reason being we can discover knowledge that could be missed by looking at a localized sample rather than a holistic view of every users' opinion. In other words, a machine algorithm can find the needles in a haystack that we did not know were there from taking small bundles of hay at a time.

    In this study, we can predict that several things might occur. Our data could show satisfaction and dissatisfaction factors that are universal, and it could also find strictly cultural factors. However, we expect that both of these options will present themselves. We can also assert that we could arrive at very similar results to previous literature if they are correct in their findings. However, we are using a database of several orders of magnitude larger. We can also expect to discover patterns that researchers previously had not noticed because of the lack of questionnaire design and users' freedom to record their pleasures and grievances.

\section{Data Analysis}\label{dataanalysis}

  \subsection{Frequent keywords in differently priced hotels}\label{svmresults}

    To understand Chinese tourists and English-speaking tourists' satisfaction and dissatisfaction factors when lodging in Japan, we study both the frequency of the words they use. Following that, to know the relevance of a keyword as a preference for each group, we observed each entropy-based keyword's frequencies in our complete data set and in each price range. The frequency of the keywords in the database shows the level of priority it has for customers.

    We observed the top 10 words with the highest frequencies for keywords linked by entropy to satisfaction and dissatisfaction in emotionally positive and negative statements to study. The keywords are the quantitative rank of the needs of Chinese and English speaking customers. We show the top 10 positive keywords for each price range comparing English and Chinese in Table \ref{tab:freq_res_pos}. For the negative keywords, we show the results in Table \ref{tab:freq_res_neg}.

    We can observe that the most used keywords for most price ranges in the same language are similar, with a few changes in priority for the keywords involved. For example, in Chinese, we can see that the customers praise cleanliness first in cheaper hotels, whereas the size of the room or bed is praised more in hotels of higher class. Another example is that in negative English reviews, complaints about price appear only after 10,000 yen hotels. After this, it climbs in importance following the increase in the hotel's price.

  \subsection{Frequently used adjectives and their pairs}\label{adjresults}

    Some keywords in these lists are adjectives, such as the word ``\begin{CJK}{UTF8}{gbsn}大\end{CJK} (big)'' mentioned before. To understand those, we performed the dependency parsing, and part of speech tagging explained in section \ref{textprocessing}. While many of these connections, we only considered the top 4 used keyword connections per adjective per price range. We show the most used Chinese adjectives in positive keywords in Table \ref{tab:adj_zh_pos}, and for negative Chinese adjective keywords in Table \ref{tab:adj_zh_neg}. Similarly, for English adjectives used in positive sentences we show the most common examples in Table \ref{tab:adj_en_pos}, and for adjectives used in negative sentences in Table \ref{tab:adj_en_neg}.

  \subsection{Determining hard and soft attribute usage}\label{det_hard_soft}

    To further understand the differences in satisfaction and dissatisfaction in Chinese and Western customers of Japanese hotels, we classified these factors as either hard or soft attributes of a hotel. We define hard attributes as matters regarding the hotel's physical or environmental aspects, such as facilities, location, or infrastructure. Some of these aspects would be impractical for the hotel to change, such as its surroundings and location. Others can be expensive to change, such as matters requiring construction costs, which are possible but regarding infrastructure investment. On the other hand, soft attributes are the non-physical attributes of the hotel service and staff behavior that are practical to change through management. For example, the hotel's services or the cleanliness of the rooms are soft attributes. For our purposes, amenities, clean or good quality bedsheets or curtains, and other physical attributes that are part of the service and not of the hotel's physical structure are also considered soft attributes. Thus, we can observe the top 10 satisfaction and dissatisfaction keywords and determine whether they are soft or hard attributes.

    We manually labeled each language's top keywords into either hard or soft by considering how the word would be used when writing a review. If the word is describing unchangeable physical factors by the staff or management, we consider them hard. If the word implies an issue that could be solved or managed by the hotel staff or management, we consider it soft. For adjectives, we looked at the top 4 adjective and noun pairings used in the entire dataset. We counted the percentage of usage in each context. If it is not clear from the word or the pairing alone, we declare it undefined. Then, we added the counts of these words in each category. A single word with no pairing is always 100\% in the category it corresponds to. We add the partial percentages for each category when an adjective includes various contexts. The interpretation of these keywords is shown in the Tables \ref{tab:zh_hard_soft_keywords} and \ref{tab:en_hard_soft_keywords}. We can see the summarized results for the hard and soft percentages of positive and negative Chinese keywords in Figure \ref{fig:hard_soft_zh}. For the English keywords, see Figure \ref{fig:hard_soft_en}.

\end{document}
