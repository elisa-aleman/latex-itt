\documentclass{letter}

% To insert commas into large numbers automatically
\usepackage{siunitx}

\begin{document}
\pagenumbering{gobble}



\textbf{Reviewer \#3 Response}

Thank you for your review observations. I will quote each observation and respond to it while indicating my corrections in the text.
Aside from these corrections, we also performed a general grammar check and proofread to the new manuscript, including unchanged sections. Because of comments by reviewers, we redid the experiment with more balanced data, so the results shown are different in their quantitative values. However, our conclusions and hypothesis testing remained largely unchanged by this. We also added a new research rationale and focus, which meant new hypotheses were added to the manuscript. The new manuscript focuses on how these populations interact with the high-standard spirit of Japanese hospitality and service called Omotenashi. In this environment, will the cultural differences be strong enough that their satisfaction stems from different factors?, or will both populations be largely satisfied by service and staff related terms?

\begin{quotation}
\#1 - Research rationale
The authors mentioned "Up until recently, most studies in tourist behavior have been biased for the Western world". Yet, to the best of my knowledge, there are fruitful studies examined Chinese consumers' hotel booking behavior

I searched and found that several similar studies have already been published recently like:
*  Francesco, G., \& Roberta, G. (2019). Cross-country analysis of perception and emphasis of hotel attributes. Tourism Management, 74, 24-42.
*  Huang, J. (2017). The dining experience of Beijing Roast Duck: A comparative study of the Chinese and English online consumer reviews. International Journal of Hospitality Management, 66, 117-129.
*  Jia, S. S. (2020). Motivation and satisfaction of Chinese and US tourists in restaurants: A cross-cultural text mining of online reviews. Tourism Management, 78, 104071.

The authors are encouraged to review and indicate the existence of the above articles. To highlight the uniqueness of this work, the authors can highlight how the current study differs from other published works in terms of "research focus", "methodological approach", "study context" or/and "data size"
\end{quotation}


I think the misunderstanding comes from the phrasing of our sentence. "Up until recently" does not mean "Until now". We mean to say that in the last recent years the focus has changed. We changed the phrase to reflect this in a more understandable manner. However, we are glad to add these articles to our manuscript literature review and highlight the differences.

In addition, we found a new rationale for our manuscript that makes it stand against the rest: the study context of Japanese hospitality (omotenashi). The different cultures react differently to this high standard hospitality environment, which is interesting in its own regard. We added a new section 3.1  Japanese hospitality: Omotenashi to explain it and developed hypotheses from that context.

We added a paragraph in the beginning of our introduction, modified the abstract, and added the literature review in this context in previously 3.2, now section 3.3 Chinese and Western tourist behavior. We also changed the title to emphasize this context as well. 

Because of the change in rationale, new hypotheses were added as well. The main problem in our refocused paper is: In such a high-standard hospitality environment, will the cultural backgrounds react similarly or differently to them? Will they both be satisfied more with hospitality and service related topics than with other topics, or will they ignore this environment?

We found that Chinese tourists don't react as positively to this hospitality as Westerners did. We have two theories which we discussed in section 7. One is that they don't have the same expectations. Getting what they don't expect leads to less satisfaction than if they were expecting it. Another theory is that the language barrier poses an obstacle to the enjoyment of this hospitality.


\begin{quotation}
\#2 - Research objectives
*  Section 2 shows that the objective of this study is to determine the difference in preferences between Chinese- and English-speaking customers of Japanese hotels using text-mining techniques. Yet, according to my review, this study solely examines what English- and Chinese-speaking customers concern most when they evaluate Japanese hotels

*  I can accept the assertion that the findings can reflect the differences in factors driving satisfaction and dissatisfaction between Chinese- and English-speaking customers. But the findings cannot reflect their preferences
\end{quotation}


In a previous version of our manuscript, we had another reviewer point out that "satisfaction and dissatisfaction factors" was a long way to say preferences, which is why we had it corrected this way. Based on this more recent review, however, we happily reverted it to the previous terminology. This is reflected in the manuscript and title. We also modified the research objective to match the new research rationale that was encouraged in a previous point.



\begin{quotation}
\#3 - Literature review
*  Since H1, H2 and H3 has already been extensively tested in prior studies, I do not see the worthiness and necessity to re-examine them. 

*  For Section 3.2, the authors may consider reviewing more cross-country studies and presenting the findings in a tabular format (with the column showing study topic, industry, samples' nationality …) This comparison table shall be helpful for the authors to justify if the research gaps truly exist
\end{quotation}


As to the first point, we believe there is always a necessity in science to re-examine and avoid making assumptions. In addition to this thought, other reviewers have found the hypotheses helpful in the flow of the manuscript, so we decide to keep them as part of the manuscript. As part of the rationale change, we also added new hypotheses.

For the second point, we agree that adding literature was necessary as per a previous point. We also agree that displaying them such a table would give a helpful overview of the current research in this topic for our paper and have included it in the previously 3.2, now section 3.3.



\begin{quotation}
\#4 - Method
*  To provide a fair comparison, the two sets of texts should review the same hotels within a similar time period. The authors should provide this background information in order to demonstrate the fairness of comparing those two sets of texts

*  Please report more information about the review samples, including time period, hotel and hotel attributes involved in this study and etc
\end{quotation}


In regards to this point, we apologize and would like to explain. In the original manuscript, the experiments were done while the data was still being collected further. After finishing the manuscript, the original data had changed. Not only this, but we realized our mistake from this and another comment from a different reviewer. Indeed the data should reflect the same kinds of hotels, the same kinds of categories, the same time period and attributes. Since we did not data on the attributes of each hotel, we decided to match both English and Chinese comments to come from exactly the same hotels, matching by name. This reduced the number of comments on each language, but it made the data fairer. We don't have data on what type of hotel or star rating they had since this data wasn't available to us, but we can be sure they are the same 580 hotels for both kinds of reviews. 

Because of this change in datasets, we also redid the entire experiment, calculating the frequencies and RBO again accordingly. 

The time period resulted to be from July 2014 to July 2017. The price range was from 2000 yen per night, to 188,000 yen a night.

We added this paragraph to the Data collection section:

"However, in order to make the data and comparisons we draw from each of these datasets fair, we filtered both databases to only contain reviews from hotels that were in both datasets, using their English names to do a search match. We also filtered them to be in the same date range and cut off reviews outside of each other's date ranges. After filtering, we found that the number of hotels in common in the data collected was \num[group-separator={,}]{580}, and that the common date range for reviews was from July 2014 to July 2017. We also found that the price for a night in these hotels ranges from low priced capsule hotels at 2000 yen per night, to high end hotels \num[group-separator={,}]{188000} yen a night as the far ends of the bell curve. Within this range, from \textit{Ctrip} there was \num[group-separator={,}]{49561} reviews comprised of \num[group-separator={,}]{105076} sentences, and from \textit{TripAdvisor} there was \num[group-separator={,}]{41611} reviews comprised of \num[group-separator={,}]{352022} sentences.

We found that after filtering the data, the number of reviews was similar for both English and Chinese reviews, but that English reviews tend to be longer in general."



\begin{quotation}
\#5 - Discussions and Conclusions
*  Regarding the discussion part, the authors discussed the differences (in sections 7.1 and 7.2) and the similarities (in section 7.3 and 7.4) of Western and Chinese consumers' satisfaction and dissatisfaction factors.  However, the authors discussed the difference in section 7.5 again. Also, the smell of Tobacco was already mentioned in the section 7.2 ("the most common complaint regarding cleanliness was about the carpet, followed by complaints about stains, and cigarette stench in the curtains"). This is a little bit confusing, and I kindly suggest the authors to re-organize the discussion part

*  The manuscript does not include much discussions about its theoretical contributions. Again, how what new knowledge was brought by this study? Any contrasting findings were found?
\end{quotation}

Because of our change in research rationale, we added new content to the introduction, changed the abstract and titles, as well as adding a new section in the discussion with theoretical perspectives on the context of Japanese hospitality. It is right that our study did not measure "culture", so we can't be sure of the explanations to certain aspects, although we do explain in the literature review that it has been previously studied how different backgrounds and cultures result in different expectations and as such, different satisfaction factors. 

In summary, our expanded discussion is about the differences between Chinese and Western tourists in the context of Japanese hospitality. Western tourists had hospitality and service related satisfaction keywords at the top of their priorities. However, Chinese tourists did not. We theorize this can be from two different causes. One theory is that the Chinese customers have different expectations based on their culture background, and that these expectations lead to different reactions. Specifically, not expecting something and then getting it could be less satisfying than getting an expectation met. Another theory is that despite being in the environment of Omotenashi, their language barriers are enough to disregard hospitality because of their unease or disappointment with the staff and communication availability.

We also expanded on the discussion section by adding more to the previously 7.5, now section 7.2 Managerial vs. Environmental satisfaction, as well as add reasoning to each discussion section as well. In this section we added potential ramifications of the distribution of these factors. While Westerners can be satisfied even more by improving on managerial factors, a big portion of Chinese satisfaction comes from environmental factors. This means that it could be harder for managers to achieve an improvement in their satisfaction. However, we also claim that improving on cleanliness and making hotels tobacco-free could be a solution that benefits both populations. If this is done in addition to improving Chinese language services, both populations can be satisfied further.

We added a section 7.3 Implications for hotel managers, where we discuss different strategies that managers could follow to improve on customer satisfaction for these populations. We propose two possible strategies, which are appropriate or not depending on a hotel business model. A hotel can aim to increase the numbers of Chinese customers by investing on Chinese language services, which are a larger population but spend less percentage of their budget on hotels. On the other hand, a hotel can aim to increase cleanliness and comfort, which are bigger investments, but would satisfy Western customers, who spend almost half their budget on hotels.

We added most of what was once in the discussion section to the results section, and added more in-depth discussion in section 7 as explained above. The subsections that moved up to the Results section were 6.2 Chinese tourists - A big and clean space, 6.3 Western tourists - A friendly face, and absolutely clean, 6.4 Tobacco, what’s that smell?, and 6.5 Location, location, location. The price section was deleted since it was short enough to be discussed elsewhere.

We also revised the tables in what previously was 6.1 to present them side to side for better following the differences in keyword ranking. They are now tables 5 and 6.



\begin{quotation}
\#6 - Minor
*  The manuscript title should be revised. The phrase "drivers of satisfaction and dissatisfaction" should be added. Also, instead of mentioning "with machine learning" in the manuscript title, the authors may consider highlighting "using Rank-biased Overlap measure"

\end{quotation}

We reconsidered the title from this and another point in this review as follows:

"Cross-culture differences in tourists faced with Japanese hospitality: A text mining and rank-biased overlap measure study of satisfaction and dissatisfaction factors in Chinese and Western cultures"

This reflects both points in this comment and focuses on the change in research rationale, while also highlighting the method.




\end{document}